\section{Kan-Quillen Model structure}

For references for this section see \cite[Section 2.2]{Cisinski_2019}, \cite[Section I.9]{GoerSimp1999} and \cite{Model_category_Hovey}.

Lecture 21.1

\underline{Aim} We prove that $(\SetD,$ Weq, Mono's, Kanfib.$)$ is a model structure.

Let us examine what we know so far.
\begin{enumerate}
    \item 
    Weak equivalences satisfy the 2 out of 3 property.
    \item 
    (Mono's, Triv Kan Fib.) and (An, Kanfib.) form a weak factorisation system.
    Thus for all $f\colon X \to Y$ we have factorisations
    \[
    \begin{tikzcd}
        &
        Z
        \ar[rd, "\in \text{ Triv. KanFib. }"]
        &
        \\
        X
        \ar[ru, " \in \text{ Mono's }"]
        \ar[rr, "f"]
        &&
        Y
    \end{tikzcd}
    \qquad
    \begin{tikzcd}
        &
        Z'
        \ar[rd, "\in \text{ KanFib. }"]
        &
        \\
        X
        \ar[ru, " \in \An "]
        \ar[rr, "f"]
        &&
        Y
    \end{tikzcd}
    \]
    Remember that the Triv. KanFib. $= \{ \partial\Delta^n \hookrightarrow \Delta^n \mid n \geq 0 \}^{\lift{}}=\text{ Mono's }^{\lift{}}$, Mono's $=\prescript{\lift{}}{}{\text{ Triv. KanFib. }} = \prescript{\lift{}}{}{(\{ \partial\Delta^n \hookrightarrow \Delta^n \mid n \geq 0 \}^{\lift{}})}$ and that $\KanFib\coloneqq \{ \Lambda_k^n \hookrightarrow \Delta^n \mid n \geq 1, 0 \leq k \leq n \}^{\lift{}}= \An^{\lift{}}, \An= \prescript{\lift{}}{}{(\KanFib)^{\lift{}}}= \prescript{\lift{}}{}{(\{ \Lambda_k^n \hookrightarrow \Delta^n \mid n \geq 1, 0 \leq k \leq n\}^{\lift{}})}$.
\end{enumerate}

What we need is $\An=$ Weq $\cap$ Mono's and TrivKanFib$=$ Weq $\cap$ KanFib and we already know $\An \subseteq$ Weq $\cap$ Mono's and Triv.KanFib. $\subseteq$ Weq $\cap$ KanFib.

\begin{prop}
    Assume that TrivKanFib = Weq $\cap$ KanFib, then $\An \supseteq$ Weq $\cap$ Mono's.
\end{prop}

\begin{proof}
    Let $f\colon X \to Y $ be in Weq $\cap$ Mono's.
    Choose a factorisation 
    \[
    \begin{tikzcd}
        &
        Z
        \ar[rd, "p"]
        &
        \\
        X
        \ar[ru, "i"]
        \ar[rr, "f"]
        &&
        Y
    \end{tikzcd}
    \]
    where $i \in \An$ and $p \in \KanFib$ since $i \in \An \subseteq$ Weq $\cap$ Mono's and $f\in$ Weq $\cap$ Mono's.
    By the 2 out of 3 property $p \in$ Weq $\cap \KanFib = $TrivKanFib.
    Take the square 
    \[
    \begin{tikzcd}
        X
        \ar[r, "i"]
        \ar[d, "f \in \text{ Mono's}"]
        &
        Z
        \ar[d,"p \in \text{ TrivKanFib. }"]
        \\
        Y 
        \ar[r, equal]
        &
        Y
    \end{tikzcd}
    \]
    since we have that $f \lift{} p$ we get by the retract argument \cref{Retract_argument} that $f$ is a retract $i \in \An$.
\end{proof}

We are now reduced to prove Triv. KanFib. $\supseteq$ Weq $\cap$ KanFib.

\begin{prop}
\label{Property A}
Let us call the following property, property A.
Let now $p \colon X \to Y$ be a $\KanFib$ and take the square
\[
\begin{tikzcd}
    X'
    \ar[d, "q"]
    \ar[r, "f"]
    &
    X
    \ar[d, "p \in \KanFib"]
    \\
    Y'
    \ar[r, "g"]
    &
    Y
\end{tikzcd}
\]
If $p$ is a weak equivalence, then $q \in$ weak eq. $\cap$ KanFib. 
\end{prop}

\begin{thm}
\label{Ex_infinity properties}
    Let $f\colon X \to Y$ be a Kan Fibration.
    The following are equivalent
    \begin{enumerate}
        \item 
        $f$ is a trivial Kan Fibration,
        \item 
        $f$ is a homotopy equivalence,
        \item 
        $f$ is a weak equivalence,
        \item 
        for all $y \in Y$
        \[
        \begin{tikzcd}
            X_y
            \ar[r]
            \dar
            &
            X
            \dar["f \in \text{ Weq $\cap$ KanFib}"]
            \\
            \Delta^0
            \rar["y"]
            &
            Y
        \end{tikzcd}
        \]
        $X_y$ is a contractible Kan complex, that is $X_y \to \Delta^0 \in$ TrivKanFib..
    \end{enumerate}
\end{thm}

\begin{proof}
    The only implication that is missing to be shown is 3. to 4. which relies on \cref{Property A}.
    Assume that $f\colon X \to Y$ is in Weq. $\cap$ KanFib.
    Then the following is a pullback
    \[
    \begin{tikzcd}
        X_y
        \rar
        \dar
        &
        X
        \ar[d, "f \in \text{ Weq $\cap$ KanFib.}"]
        \\
        \Delta^0 
        \rar
        &
        Y
    \end{tikzcd}
    \]
    and with \cref{Property A} it follows that $X_y \to \Delta^0$ in Weq $\cap$ KanFib. is a trivial KanFib..
    We are reduced to proving \cref{Property A}.
\end{proof}

\begin{thm}
    There is a functor $\Exinfty \colon \SetD \to \SetD$ with the following properties 
    \begin{enumerate}
        \item 
        For all $X \in \SetD, \Exinfty$ is a Kan complex.
        \item 
        There exists a natural transformation $\mathds{1} \xrightarrow{\beta} \Exinfty$ such that for all $X \in \SetD$ $X \to \Exinfty$ is a weak equivalence.
        \item   
        The functor $\Exinfty \colon \SetD \to \SetD$ preserves finite limits, weak equivalences and Kan Fibrations.
    \end{enumerate}
\end{thm}

\begin{proof}{\cref{Property A}}    
    Take the square 
    \[
    \begin{tikzcd}
        X' 
        \ar[r,"f"]
        \ar[d, "q"]
        &
        X
        \ar[d, "p \in \text{ Weq $\cap$ KanFib.}"]
        \\
        Y'
        \ar[r, "g"]
        &
        Y
    \end{tikzcd}
    \]
    Apply Kan's $\Exinfty$-functor to obtain:
    \[
    \begin{tikzcd}
        X'
        \ar[d, "g"]
        \ar[rrr, "\circ" marking, "\beta_{X'}"]
        \ar[rrrd, "\circ" marking, "\in \text{ Triv. KanFib. }"]
        &&&
        \Exinfty(X')
        \ar[r,"\Exinfty(f)"]
        \ar[d, "\circ" marking, "\Exinfty(g)"]
        &
        \Exinfty(X) 
        \ar[d, "\Exinfty(p) \in \text{Weq $\cap$ KanFib}"]
        \\
        Y'
        \ar[rrr, "\circ" marking, "\beta_{Y'}"]
        &&&
        \Exinfty(Y')
        \ar[r, "\Exinfty(g)"]
        &
        \Exinfty(Y)
    \end{tikzcd}
    \]
    Note that the arrows marked with a circle are weak equivalences.
    By the 2 out of 3 property and \cref{Ex_infinity properties}, we get that $g$ is a weak equivalence.
    So we just have to show the theorem above.
\end{proof}   

Lecture 23.1

