\section{Adjunctions}

Lecture 22.10

Everytime you encounter some free object, in the sense that it is freely generated from some other object of some other category, like a free group on some set, you are most likely going to use the properties of the object stemming from an adjunction.

\begin{defi}
    \begin{tikzcd}
        L \colon  \mathcal{C} 
        \arrow[r, shift left]
        &
        \mathcal{D} \colon R
        \arrow[l, shift left]
    \end{tikzcd}
    An adjunction $L \dashv R$ is a natural isomorphism $\phi_{c,d} \colon \Hom_\mathcal{D} (Lc,d) \isomorphism \Hom_\mathcal{C}(c,Rd)$ of functors $\mathcal{C}^{\op} \times \mathcal{D} \to \Set$.
    \[
    \begin{tikzcd}
        \mathcal{C}^{\op} \times \mathcal{D}
        \arrow[bend left]{rr}[above]{\Hom_{\mathcal{D}}(L(-),-)}[below,name=U]{}
        \arrow[bend right]{rr}[below]{\Hom_{\mathcal{C}}(-,R(-))}[above,name=D]{}
        \arrow[Rightarrow, from=U, to=D, right, "\phi"]
        &
        &
        \Set
    \end{tikzcd}
    \]
    That is explicitely, we have commutative squares for all pairs of morphisms:
    \[
    \begin{tikzcd}
        (c,d) 
        \arrow[d, shift left, "\id_d"]
        &
        \Hom_{\mathcal{D}}(Lc,d) 
        \arrow[r, "\phi_{c,d}"]
        \arrow[d, "g \circ ?"']
        &
        \Hom_{\mathcal{C}}(c,Rd)
        \arrow[d, "R(g) \circ ?"]
        \\
        (c',d)
        \arrow[u, shift left , "f"]
        &
        \Hom(Lc,d')
        \arrow[r, "\phi_{c,d'}"']
        &
        \Hom_{\mathcal{C}}(c , Rd')
    \end{tikzcd}
    \]
    which means that for all $f \colon Lc \to d$ and all $g \colon d \to d'$ we have that $R(g) \circ \Bar{f} = \overline{g \circ f}$, where the closure operator denotes the image of an element under $\phi$
    \[
    \begin{tikzcd}
        (c,d) 
        \arrow[d, shift left, "\id_g"]
        &
        \Hom_{\mathcal{D}}(Lc,d) 
        \arrow[r, "\phi_{c,d}"]
        \arrow[d, "? \circ Lf"']
        &
        \Hom_{\mathcal{C}}(c,Rd)
        \arrow[l, "\Bar{(-)}"]
        \arrow[d, "? \circ f ?"]
        \\
        (c',d)
        \arrow[u, shift left , "f"]
        &
        \Hom(Lc',d)
        &
        \Hom_{\mathcal{C}}(c' , Rd)
        \arrow[l, "\Bar{(-)}"]
    \end{tikzcd}
    \]
    which means that for all $k \colon c \to Rd$ and all $f \colon c' \to c$ we have that $\Bar{k} \circ Lf = \overline{k \circ f}$.
\end{defi}

\begin{rmk}
    For $c \in \mathcal{C}$ consider $\eta_c$ given as follows
    \[
    \begin{tikzcd}
        \Hom_{\mathcal{D}}(Lc,Lc)
        \arrow[r ,"\sim"]
        &
        \Hom_{\mathcal{C}}(c,RLc)
        \\
        \id_{Lc}
        \arrow[r , mapsto]
        &
        \eta_c \coloneqq \overline{\id_{Lc}}
    \end{tikzcd}
    \]
    as well as for any $d \in \mathcal{D}$ consider $\epsilon_d$ given as follows
    \[
    \begin{tikzcd}
        \Hom_{\mathcal{C}}(Rc,Rc)
        \arrow[r ,"\sim"]
        &
        \Hom_{\mathcal{D}}(LRd , d )
        \\
        \id_{Rc}
        \arrow[r , mapsto]
        &
        \epsilon_d \coloneqq \overline{\id_{Rd}}
    \end{tikzcd}
    \]
\end{rmk}

\begin{prop}
    Notice that $\eta=(\eta_c \colon c \to RLc \mid c \in \mathcal{C})$ is a natural transformation $\eta \colon \mathds{1}_{\mathcal{C}} \ RL $ we call this the \underline{unit} of the adjunction and similarly $\epsilon=(\epsilon_c \colon c \to LRd \mid d \in \mathcal{C})$ is a natural transformation $\epsilon\colon LR  \to \mathds{1}_{\mathcal{D}}$ and is a called the \underline{counit} of the adjunction.
\end{prop}

\begin{proof}
    Consider the following square:
    \[
    \begin{tikzcd}
        c 
        \arrow[d, "f"]
        &
        c
        \arrow[r, "\eta_{\mathcal{C}}"]
        \arrow[d, "f"']
        &
        RLc
        \arrow[d, "RLf"]
        \\
        c'
        &
        c'
        \arrow[r, "\eta_{c'}"']
        &
        c'
    \end{tikzcd}
    \]
    and the resulting equation
    \begin{align}
        & RLf \circ \eta_c 
        = \eta_c' \circ f
        \\
        \iff& \overline{RLf \circ \eta_c} 
        = \overline{\eta_{c'} \circ f}
        \\
        \iff& \overline{\overline{Lf}}
        =\overline{\id_{c'} \circ Lf} 
        \\
        \iff&
        Lf = \id_f \circ Lf
    \end{align}
\end{proof}

\begin{prop}
    The unit $\eta \colon \mathds{1}_{\mathcal{C}} \Rightarrow RL$ aand the counit $\epsilon \colon LR \Rightarrow \mathds{1}_{\mathcal{D}}$ satisfy the triangle identities:
    \[
    \begin{tikzcd}
        &
        LRLc
        \arrow[rd, "\epsilon_{Lc}"]
        &
        \\
        Lc
        \arrow[ru, "L(\eta_c)"]
        \arrow[rr, "\id_{Lc}"']
        &
        &
        Lc
    \end{tikzcd}
    \qquad
    \begin{tikzcd}
        &
        RLRd
        \arrow[rd, "R(\epsilon_{d})"]
        &
        \\
        Rd
        \arrow[ru, "\eta_{Rd}"]
        \arrow[rr, "\id_{Rd}"']
        &
        &
        Rd
    \end{tikzcd}
    \]
    For all objects $c\in \mathcal{C}$ and $d \in \mathcal{D}$.
\end{prop}

\begin{proof}
    If we take the triangle for the unit above and apply the bar-operator we obtain the following
    \[
    \begin{tikzcd}
        &
        LRLc
        \arrow[rd, "\epsilon_{Lc}"]
        &
        \\
        Lc
        \arrow[ru, "L(\eta_c)"]
        \arrow[rr, "\id_{Lc}"']
        &
        &
        Lc  
    \end{tikzcd}
    \xleftrightarrow{\overline{(-)}}
    \begin{tikzcd}
        &
        RLc
        \arrow[rd, "\id_{RLc}"]
        &
        \\
        c
        \arrow[ru, "\eta_c"]
        \arrow[rr, "\overline{\id_{c}}=\eta_c"']
        &
        &
        RLc
    \end{tikzcd}
    \]
    The second triangle clearly commutes, the argument for the counit is analogous.
\end{proof}

\begin{prop}
    Let \begin{tikzcd}
        L \colon  \mathcal{C} 
        \arrow[r, shift left]
        &
        \mathcal{D} \colon R
        \arrow[l, shift left]
    \end{tikzcd}
    be two functors between categories and suppose there exist natural transformations $\eta \colon \mathds{1}_{\mathcal{C}} \Rightarrow RL$ and $\eta\colon LR \Rightarrow \mathds{1}_{\mathcal{D}}$ that satisfy the triangle identities.
    Then the following defines an adjunction $L \dashv R$
    \[
    \begin{tikzcd}
        \phi \colon \Hom_{\mathcal{D}}(Lc,d)
        \arrow[r, shift left]
        &
        \Hom_{\mathcal{C}}(c,Rd) \colon \psi
        \arrow[l, shift left]
        \\
        Lc \xrightarrow{g} d
        \arrow[r, mapsto ,shift left]
        &
        R(g)\circ \eta_c
        \arrow[l, mapsto ,shift left]
    \end{tikzcd}
    \]
\end{prop}

\begin{proof}
    For $g \colon Lc \to d$ we have that 
    \[
        g=(\psi\circ\phi)(g) = \psi(R(g) \circ \eta_c) = \epsilon_d \circ LR(g) \circ L(\eta_c)
    \]
    Now we have the following the diagram
    \[
    \begin{tikzcd}
        &
        LRLc
        \arrow[rd, "\epsilon_{Lc}"]
        \arrow[rr, "LRg"]
        &
        &
        LRd
        \arrow[rd, "\epsilon_d"]
        \\
        Lc 
        \arrow[ru,"L(\eta_c)"]
        \arrow[rr,"\id_{Lc}"']
        &
        &
        Lc 
        \arrow[rr,"g"']
        &
        &
        d
    \end{tikzcd}
    \]
    The triangle commutes due to the triangle identities and the square commutes by the naturality of the counit. thus the whole diagram commutes, which means that $\psi$ is an inverse to $\phi$ which yields the statement.    
\end{proof}

\begin{defi}
    For $c \in \mathcal{C}$ and $R\colon \mathcal{D}\to\mathcal{C}$ a functor, define the category $c/R$ with objects given by tuples $(c,f)$
    where $f\colon c \to R(d)$ is a morphism for some $d\in \mathcal{D}$ and morphisms are given by 
    \[
    \begin{tikzcd}
        d \in \mathcal{D} 
        \arrow[d,"g"]
        &
        c
        \arrow[r,"f"]
        \arrow[d, equal]
        &
        Rd
        \arrow[d,"R(g)"]
        \\
        d' \in \mathcal{D}
        &
        c
        \arrow[r,"f'"']
        &
        Rd'
    \end{tikzcd}
    \]
    Dually for $L \colon \mathcal{C} \to \mathcal{D}$ and $d \in \mathcal{D}$ we define $L/d$ as tuples $(d,g)$ where $g\colon Lc \to g$ is a morphism and morphisms are given by 
    \[
    \begin{tikzcd}
        c \in \mathcal{C} 
        \arrow[d,"f"]
        &
        Lc
        \arrow[r,"g"]
        \arrow[d, "Lf"']
        &
        d
        \arrow[d, equal]
        \\
        c' \in \mathcal{C}
        &
        Lc'
        \arrow[r,"g'"']
        &
        d
    \end{tikzcd}
    \]
    Notice that given an adjunction $ L \dashv R$ we have that $\forall c \in \mathcal{C} (c,c \xrightarrow{\eta_c} Rc)$ is in $c/R$ and $\forall \in \mathcal{D}(d, LR \xrightarrow{\epsilon_d} d)$ is in $L/d$.
\end{defi}

\begin{prop}
\label{initial_final_comma_category}
    The object $(c,c \xrightarrow{\eta_c} Rc)$ is initial in $c/R$ and
    $(d, LR \xrightarrow{\epsilon_d} d)$ is final in $L/d$.
\end{prop}

\begin{proof}
    Consider the following commutative triangle
        \[
        \begin{tikzcd}
            c 
            \arrow[r, " \eta_c "] 
            \arrow[rd, "\overline{\overline{f}}=f"']
            &
            RLc
            \arrow[d,"R(\overline{f})"]
            \\
            &
            Rd
        \end{tikzcd}
        \]
    where $\overline{\overline{f}}=\overline{R(\overline{f})\circ \eta_c}$. 
    Thus the morphism $f$ of the object $(c,f)$ uniquely determines the morphism $R(\overline{f})$. 
    The argument for final object is dual
\end{proof}

Lecture 24.10

\begin{prop}
    Let 
    \[
    \begin{tikzcd}
        \mathcal{C}
        \arrow[r, shift left, "L_1"]
        &
        \mathcal{D}
        \arrow[l, shift left, "R_1"]
        \arrow[r, shift left, "L_2"]
        &
        \mathcal{E}
        \arrow[l, shift left, "R_2"]
    \end{tikzcd}
    \]
    be adjunctions then their composition
    \[
    \begin{tikzcd}
        \mathcal{C}
        \arrow[r, shift left, "L_1\circ L_2"]
        &
        \mathcal{D}
        \arrow[l, shift left, "R_1\circ R_2"]
    \end{tikzcd}
    \]
    is an adjunction as well.
\end{prop}

\begin{proof}
Consider the following transformations, given by the adjunction isomorphisms
    \[
    \Hom_{\mathcal{E}}(L_2L_1c,e) \isomorphism \Hom_{\mathcal{D}}(L_1c,R_2e) \isomorphism \Hom_{\mathcal{C}}(c,R_1R_2e)
    \]
\end{proof}

\begin{rmk}
    Given an adjunction 
    \[
    \begin{tikzcd}
        \mathcal{C}
        \arrow[r, shift left, "L"]
        &
        \mathcal{D}
        \arrow[l, shift left, "R"]
    \end{tikzcd}
    \]
    then the following is an adjunction as well
    \[
    \begin{tikzcd}
        \mathcal{C^{\op}}
        \arrow[r, shift left, "L^{\op}"]
        &
        \mathcal{D^{\op}}
        \arrow[l, shift left, "R^{\op}"]
    \end{tikzcd}
    \]
    and the unit $\eta_c\colon c \to RLc$ in $\mathcal{C}$ corresponds to the counit $c \xleftarrow{}RLc$ in $\mathcal{C}^{\op}$
\end{rmk}

\begin{prop}
    Let $L_1 ; L_2 \colon \mathcal{C} \to  \mathcal{D}$ and $\mathcal{C} \xleftarrow{}\mathcal{D}:R$ be functors.
    Suppose that $L_1 \dashv R$ and $L_2 \dashv R$ are adjunctions, then it follows that $L_1 \cong L_2$.
\end{prop}

\begin{proof}
    We need to construct a natural isomorphism $\phi \colon L_1 \isomorphism L_2$.
    Let $\eta^{(1)}\colon \mathds{1} \to RL_1$ and $\eta^{(2)} \colon \mathds{1}_{\mathcal{C}} \to RL_2$.
    By the uniqueness of initial objects in $c/R$, for $c \in \mathcal{C}$ \cref{initial_final_comma_category} we obtain 
    \[
    \begin{tikzcd}
        &
        c
        \arrow[ld, "\eta^{(1)}_c"']
        \arrow[d, "\eta^{(2)}_2"]
        \arrow[rd, "\eta^{(1)}_c"]
        &
        \\
        RL_1c
        \arrow[r, dashed, "Rg"]
        &
        RL_2c
        \arrow[r, dashed, "Rh"]
        &
        RL_1c
    \end{tikzcd}
    \]
    By the uniqueness of a morphism out of an initial object we obtain for the composition that 
    $L_1c \xrightarrow{g}L_2c \xrightarrow{h}L_1c$ 
    is given by $h \circ g = \id_{L_1c}$. 
    Similarly one obtains $g \circ h = \id_{L_2c}$.
    Thus $g \eqqcolon U_c \colon L_1c \isomorphism L_2c$.
    We now have to check that $U_c$ is actually a natural transformation of functors. Consider the following diagram
    
    \[
    \begin{tikzcd}
        c 
        \arrow[d, "f"]
        &
        L_1c
        \arrow[r, "U_c"]
        \arrow[d, "L_1f"']
        &
        L_2c
        \arrow[d, "L_2f"]
        \\
        c'
        &
        L_1c'
        \arrow[r, "U_c'"']
        &
        L_2c'
    \end{tikzcd}
    \]
    apply $R$ to it 
    \[
    \begin{tikzcd}
        &
        c
        \arrow[dl, "\eta^{(1)}_c"']
        \arrow[dr, "\eta^{(2)}_c"]
        \arrow[dd]
        &
        \\
        RL_1c
        \arrow[rr, crossing over, "R(U_c)", pos=0.3]
        \arrow[dd, "RL_1(f)"']
        &
        &
        RL_2c
        \arrow[dd,"RL_2(f)"]
        \\
        &
        c'
        \arrow[ld, "\eta_{c'}^{(1)}"']
        \arrow[rd, "\eta_{c'}^{(2)}"]
        &
        \\
        RL_1c'
        \arrow[rr, "R(U_{c'})"]
        &
        &
        RL_2(c')
    \end{tikzcd}
    \]
    This yields the following equations
    \begin{align}
        R(U_{c'} \circ L_1(f)) \circ \eta^{(1)}_c
        &= R (U_{c'}) \circ \eta_c^{(1)} \circ f
        \\
        &= \eta_c^{(2)} \circ f
        \\
        &=RL_2(f) \circ \eta_c^{(2)}
        \\
        &=RL_2(f) \circ R(U_c) \circ \eta_c^{(1)}
        \\
        &=R(L_2(f) \circ U_c) \circ \eta_c^{(1)}
    \end{align}
    And thus results in the following commutative triangle 
    \[
    \begin{tikzcd}
        c 
        \arrow[r, "\eta_c^{(1)}"]
        &
        RL_1c
        \\
    \end{tikzcd}
    \]
    \todo{why are we done here ??}
\end{proof}

\begin{prop}
    Let $\begin{tikzcd}
        \mathcal{C}
        \arrow[r, shift left, "L"]
        &
        \mathcal{D}
        \arrow[l, shift left, "R"]
    \end{tikzcd}$
    be an adjunction then $L$ preserves colimits that exist in $\mathcal{C}$ and $R$ preserves limits that exist in $\mathcal{D}$.
\end{prop}

\begin{proof}
    Let $X\colon A \to \mathcal{C}$ be a diagram that admits a colimit in $\mathcal{C}$, $\colim X_a \in \mathcal{C}$ and $ a \in A$.
    \begin{align}
        \Hom_{\mathcal{D}}(L(\colim_{a \in A} X_a),d) 
        &\isomorphism \Hom_{\mathcal{C}}(\colim_{a \in A} X_a, Rd)
        \\
        &\isomorphism \lim_{a \in A} \Hom_{\mathcal{C}}(X_a,Rd)
        \\
        &\isomorphism \lim_{a \in A} \Hom_{\mathcal{C}}(LX_a,d)
        \\
    \end{align}
    This exhibits $L(\colim_{a \in A} X)$ as a colimit of $X\colon A \xrightarrow{X} \mathcal{C} \xrightarrow{L} \mathcal{D}$.
\end{proof}

Let $A$ be a small category and $\mathcal{C}$ a category.
Consider the functor $\const_A \colon \mathcal{C} \to \Fun(A , \mathcal{C})$
that maps each object of $\mathcal{C}$ to the functor $F_c(a)=c$ for all and $a \in A$ and each morphism to $\id_c$.

\begin{prop}
    Suppose that there exists $L \colon \Fun(A,\mathcal{C}) \to \mathcal{C}$ a left adjoint to $\const_A$. 
    Then for all $X \colon A  \to \mathcal{C}$ the unit $\eta_X \colon X \to \const_A(LX)$ exhibits $LX$ as a colimit of $X$.
\end{prop}

\begin{proof}
    We know that $\eta_X \colon X \to \const_A(LX)$ is initial in $X / \const_A$.
    Notice that the objects of $X/\const_A$ are pairs
    $(c \in \mathcal{C}, \rho \colon X \Rightarrow \const_A(c))$.
    Thus $\Bar{\rho} = ( \rho_a \colon X_a \to c \mid a \in A )$ is such that 
    \[
    \begin{tikzcd}
        a
        \arrow[d,"u"]
        &
        X_a 
        \arrow[r, "\rho_a"]
        \arrow[d, "X_u"']
        &
        c
        \arrow[d, "\id_c"]
        \\
        b
        &
        X_b
        \arrow[r, "\rho_b"]
        &
        c
    \end{tikzcd}
    \]
    Let furthermore $f\colon c \to c'$ be a morphism inducing a morphism in $X/\const_A$
    \[
    \begin{tikzcd}
        X 
        \arrow[r, Rightarrow, "\Bar{\rho}"]
        \arrow[d, equal]
        &
        \const_A(c)
        \arrow[d,equal, "\const_A(f)"]
        \\
        X 
        \arrow[r, Rightarrow, "\Bar{\sigma}"]
        &
        \const_A(c')
    \end{tikzcd}
    \]
    given evaluated on objects $a \in A$ by 
    \[
    \begin{tikzcd}
        X_a
        \arrow[r, "\phi_a"]
        \arrow[rd, " \sigma"]
        \arrow[d, equal]
        &
        c
        \arrow[d, "f"]
        \\
        X_a
        \arrow[r,"\sigma"]
        &
        c'
    \end{tikzcd}
    \]
\end{proof}

\todo{why are we done here ? Don't see it.}


\begin{prop}
    Let $\mathcal{C} \xleftarrow{}\mathcal{D}:R$ be such that for all $c \in \mathcal{C}$ the functor $\Hom_{\mathcal{C}}(c,R(-))\colon \mathcal{D} \to \Set$ is corepresentable by an object $L(c) \in \mathcal{D}$ via $\phi_c \colon \Hom_{\mathcal{D}}(L(c),-) \isomorphism \Hom_{\mathcal{D}}(c,R(-))$.
    Then the association $c \mapsto L(c)$ can be promoted to a functor $L \colon \mathcal{C} \to \mathcal{D}$ that is adjoint to $R$ via $\phi$.
\end{prop}

\begin{proof}
    We need to define $L$ on morphisms.
    For $c \xrightarrow{f}c'$ in $\mathcal{C}$ consider the commutative square
    \[
    \begin{tikzcd}
        \Hom_{\mathcal{D}}(L(c),-)
        \arrow[r, "\phi_c"]
        &
        \Hom_{\mathcal{C}}(c,R(-))
        \\
        \Hom_{\mathcal{D}}(L(c'),-)
        \arrow[u, dashed, "L(f)^*"]
        \arrow[r, "\phi_{c'}"]
        &
        \Hom_{\mathcal{C}}(c',R(-))
        \arrow[u, "f^*"']
    \end{tikzcd}
    \]

    By Yoneda we obtain an object $d \in\mathcal{D}$ such that
    \[
    \begin{tikzcd}
        L(c)
        \arrow[d, "L(f)"']
        \arrow[rd]
        &
        \\
        L(c')
        \arrow[r]
        &
        d
    \end{tikzcd}
    \]
    Now we need to prove that this actually defines a functor $L\colon \mathcal{C} \to \mathcal{D}$.
    For $c \in \mathcal{C}$ we have that $L(\id_c)=\id_{L(c)}$ by construction.
    Let $c \xrightarrow{f}c' \xrightarrow{g}c''$ be in $\mathcal{C}$.
    \[
    \begin{tikzcd}
        \Hom_{\mathcal{D}}(L(c),-)
        \arrow[r, "\phi_c"]
        &
        \Hom_{\mathcal{C}}(c, R(-))
        \\
        \Hom_{\mathcal{D}}(L(c'),-)
        \arrow[r, "\phi_c'"]
        \arrow[u, "L(f)^*"]
        &
        \Hom_{\mathcal{C}}(c', R(-))
        \arrow[u, "f^*"']
        \\        
        \Hom_{\mathcal{D}}(L(c''),-)
        \arrow[r, "\phi_c''"]
        \arrow[u, "L(g)^*"]
        \arrow[uu, bend left=90, "L(g \circ f)"]
        &
        \Hom_{\mathcal{C}}(c', R(-))
        \arrow[u, "g^*"']
        \arrow[uu,bend right=90,  shift right ,"g \circ f"']
    \end{tikzcd}
    \]
    The uniqueness given by Yoneda, implies
    \[
    L(g \circ f) = L(g) \circ L(f)
    \]
    Let $\phi_{c,d}=(\phi_c)_d$
\end{proof}