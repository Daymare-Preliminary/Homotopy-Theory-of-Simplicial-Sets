\section{Motivation}

Let $\mathcal{C}$ be a category and $W$ a class of morphisms.

\begin{defi}
\label{localisation_morphisms}
    A localisation of $\mathcal{C}$ at $W$ is a category $\mathcal{C}[W^{-1}]$ together with a functor $\gamma\colon \mathcal{C} \to \mathcal{C}[W^{-1}]$ such that $\forall f \in W$, we get that $\gamma(f)$ is an isomorphism in $\mathcal{C}[W^{-1}]$.
\end{defi}

\begin{exmp}
    \begin{itemize}
        \item 
        A ring considered as category and the localisation at an ideal.
        Can this be extended to rings with more than one object?
        \item 
        The derived category of an abelian category is the localisation with respect to the quasi isomorphisms.
    \end{itemize}
\end{exmp}

\begin{prop}
    Let $\mathcal{C}[W^{-1}]$ as in \Cref{localisation_morphisms}. 
    For any category $\mathcal{D}$ the functor 
        \begin{align*}
            j^*\colon\Fun(\mathcal{C}[W^{-1}], \mathcal{D}) 
            \to
            &
            \Fun(\mathcal{C},\mathcal{D})
            \\
            (\mathcal{C}[W^{-1}] \to 
            \mathcal{D})\longmapsto
            &
            (\mathcal{C} \xrightarrow{\gamma}
            \mathcal{C}[W^{-1}]
            \xrightarrow{F}
            \mathcal{D})
        \end{align*}
    is an equivalence.
\end{prop}

\begin{thm}%[{\cite{Gabriel1967CalculusOF}}]
    Set-theoretic issues aside, loacalisations always exist. 
\end{thm}
    
%\todo{where}

\begin{exmp}
    \begin{itemize}
        \item 
        Let $\Top$ be the category with objects given by topological spaces and morphisms given by continuous maps.
        Let $W$ be the weak homotopy equivalences, that is morphisms $f:X \to Y$ such that the induced maps on path components 
        \[
        \pi_0(f) \colon\pi_0(X) \isomorphism \pi_0(Y)
        \]
        and for all points $x \in X$ and for all $n\geq1$ with $n \in \mathbb{N}$ the morphism $f$ induces an an isomorphism on homotopy groups
        \[
        \pi_n(f,x) \colon \pi_n(X,x) \isomorphism \pi_n(Y,f(x))
        \]
        in $\Grp$.
        The result of the localisation is called the homotopy category $\mathcal{H}: \Top[W^{-1}]$.
        \item 
        The localisation at all morphism of the category $\mathcal{C}[\mathcal{C}^{-1}]$ is a groupoid.
    \end{itemize}
\end{exmp}

\begin{rmk}
    The takeaway is the general paradigm, that the localisation is the \underline{truncation} of a richer mathematical structure.
\end{rmk}

\subsection{Exercises}

\begin{Exercise}
    For a group $ G $ let $ \set_G $ be the category of sets with a right $ G $ action i.e. 
    \begin{itemize}
        \item 
        the objects are tuples $ ( X , \rho_X ) $ where $ X $ is a set and $ \rho \colon X \times G \to X $ is a map satisfying for all $ g , h \in G $ and $ x \in X $ that $ \rho_X ( x , g h ) = \rho_X ( \rho_X ( x , g ) , h ) $ and furthermore $ \id_X = \rho_X ( - , e ) $ for $ e $ the neutral element of $ G $, and
        \item 
        a morphism $ \phi \colon ( X , \rho_X ) \to ( Y , \rho_Y ) $ is given by a map $ \phi \colon X \to Y $ satisfying $ \phi \circ \rho_X = \rho_Y \circ ( \phi \times \id_G ) $ .
    \end{itemize}
    Recall that we view $ G $ as a category $ B G $ with one object $ \star $ and  $ \Hom_{ B G } ( \star, \star ) = G $. 
    Show that there is and isomorphism of categories between $ \set_G $ and $ \widehat{ BG } $, the category of presheaves over $ B G $.
\end{Exercise}

\begin{Exercise}
    For a set $ Y $, show that there is an isomorphism of functors $ \set^{ \op } \times \set \to \set $
    \[
        \Hom_{ \set } ( - \times Y , ? ) \cong  \Hom_{ \Set } ( - , \Hom_{ \set } ( Y , ? ) ).
    \]
\end{Exercise}

\begin{Exercise}
    Let $ \eta \colon F \to G $ be a natural transformation of two functors $ F , G \colon \mathcal{ A } \to \mathcal{ B } $.
    Show that $ \eta $ is a natural isomorphism, i.e. there exists a natural transformation $ \eta' \colon G \to F $ such that $ \eta' \circ \eta = \id_F $ and $ \eta \circ \eta' = \id_G $, if and only if for every $ a \in \mathcal{ A } $ the morphism $ \eta_a \colon F ( a ) \to G ( a ) $ is an isomorphism.
\end{Exercise}

\begin{Exercise}
    Fix an object $ x \in \mathcal{ C } $ of a category $ \mathcal{ C } $. 
    The slice category $ \mathcal{ C } / x $ of $ \mathcal{ C } $ over $ x $ is defined as follows.
    \begin{itemize}
        \item 
        The objects are tuples $ ( a , \pi ) $ with $ a  \in \mathcal{ C } $ an object and a morphism $ \pi \colon a \to x $. 
        \item 
        A morphism $ f \colon ( a , \pi ) \to ( b , \rho ) $ is given by a morphism $ f \colon a \to b $ such that $ \pi = \rho \circ f $.
    \end{itemize}

    After convincing yourself that this defines a category, do the following.
    
    \begin{enumerate}[label=(\alph*)]
    
        \item 
        Show that there exists a final object in $ \mathcal{ C } / x $, i.e. an object $ ( e , \rho ) $ such that for any object $ ( a , \pi ) $ there is a unique morphism $ f \colon ( a , \pi ) \to ( e , \rho ) $.
        \item 
        
        Define the coslice category $ x / \mathcal{ C } $ of elements under $ x $.
        
        \item 
        Describe $ ( \mathcal{ C } / x )^{ \op } $ as a slice or coslice category.
        
    \end{enumerate}
\end{Exercise}
