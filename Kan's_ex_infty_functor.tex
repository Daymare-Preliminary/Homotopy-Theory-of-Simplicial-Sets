\section{Kan's Ex functor}

Lecture 23.1

Consider the adjunction 
$\begin{tikzcd}
    \Sd\colon \SetD
    \ar[r, shift left]
    &
    \SetD\colon\Ex
    \ar[l, shift left]
\end{tikzcd}$
induced by the functor 
\[
\Delta \to \SetD, [n] \mapsto N(S([n]))
\]
where $S[n]$ is the poset of non-empty subsets of $[n]$ and to $\sigma \colon [m] \mapsto [n]$ we associate.
$(U \subseteq [m] \mapsto \sigma(U) \subseteq [n])$.
For $X \in \SetD$ we have $\Ex(X)_n \coloneqq \Hom_{\SetD}(\Sd(\Delta^n),X)$ with $\Sd(\Delta^n)=N(S[n])$

\begin{exmp}
    \begin{itemize}
        \item 
        $(n=0) \Sd(\Delta^0) = N(S[0])=\{0\}$
        \item 
        $(n=1) \Sd(\Delta^1)=N(S[1]) \colon \{0\} \to \{ 0,1 \} \xleftarrow{}\{1\}$
        \item 
        $ (n=2) \Sd( \Delta^1) = N(S[2])$
        \[
        \begin{tikzcd}
            &&
            \{1\}
            \ar[rd]
            \ar[dd]
            \\
            &
            \{0,1\}
            \ar[rd]
            \ar[ru]
            &
            &
            \{1,2\}
            \ar[ld]
            \\
            &&
            \{0,1,2\}
            &&
            \\
            \{0\}
            \ar[ruu]
            \ar[rr]
            \ar[rru]
            &&
            \{0,1\}
            \ar[u]
            &&
            \{2\}
            \ar[ll]
            \ar[llu]
            \ar[luu]
        \end{tikzcd}
        \]
    \end{itemize}    
\end{exmp}

\begin{exmp}
    Let $T \in \Top$. Then we can consider $\Ex(\Sing(X))$ with $\Ex(\Sing(X))_n=\Hom_{\SetD}(\Sd(\Delta^n),\Sing(X))_n \cong \Hom_{\Top}(\lvert\Sd(\Delta^n)\rvert, T)$, where $\lvert\Sd(\Delta^n)\rvert$ is the barycentric subdivision of $\lvert\Delta^n\rvert$.
\end{exmp}    
    
    We wish to define a natural map 
    
    $\beta_X \colon X \to \Ex(X)$. 
    For this we define $a_n \colon S[n] \to [n]$ and $U \subseteq [n] \mapsto \max (U)$ and the induced map 
    $\alpha_n \colon \Sd(\Delta^n) N(S[n]) \to N([n]) = \Delta^n$ as well as the induced map 
    $(\beta_X)_n \colon X_n \cong \Hom_{\SetD}(\Delta^n, X) \xrightarrow{\alpha_n^*} \Hom_{\SetD}(\Sd(\Delta^n),X)=\Ex(X)_n$

\begin{lem}
    Take $\beta_X\colon X \to \Ex(X)$ is a morphism of simplicial sets.
\end{lem}

\begin{proof}
    The proof is just an exercise in backtracking the definitions of the constructions.
    Let $\sigma\colon [m] \to [n]$ be a morphism in $\Delta$, we get the following commutative square
    \[
    \begin{tikzcd}
        X_m \cong \Hom_{\SetD}( \Delta^m,X) 
        \ar[r, "{(\beta_X)_m=\alpha_m^*}"]
        &
        \Hom(\Sd(\Delta^m),X)=\Ex(X)_m
        \\
        X_n \cong \Hom_{\SetD}(\Delta^n,X)
        \ar[r, "(\beta_X)_m=\alpha^*"]
        &
        \Hom_{\SetD}(\Sd(\Delta^n),X)=\Ex(X)_n
    \end{tikzcd}
    \]
    Let $\Delta^n \xrightarrow{x} X$, it is enough to show that the following commutes:
    \[
    \begin{tikzcd}
        \Delta^m
        \ar[r, "\sigma"]
        &
        \Delta^n
        \ar[r, "x"]
        &
        X
        \\
        \Sd(\Delta^m)
        \ar[u, "\alpha_m"]
        \ar[r, "\Sd(\sigma)"]
        &
        \Sd(\Delta^n)
        \ar[u, "\alpha_n"]
        \ar[ru]
    \end{tikzcd}
    \]
    which is the case if the following commutes
    \[
    \begin{tikzcd}
        {[m]}
        \ar[r, "\sigma"]
        &
        {[n]}
        \\
        S([m])
        \ar[u, "a_m"]
        \ar[r, "S(\sigma)"]
        &
        S([n])
        \ar[u, "a_n"]
    \end{tikzcd}
    \qquad
    \begin{tikzcd}
        \max(U)
        \ar[r, mapsto]
        &
        \sigma(\max U)
        \\
        U
        \ar[u,mapsto]
        \ar[r,mapsto]
        &
        \sigma(U)
        \ar[u, mapsto]
    \end{tikzcd}
    \]
\end{proof}

\begin{lem}
    The map $\beta \colon \mathds{1}_{\SetD} \to \Ex$, induced by $\beta_X$, is a natural transformation.   
\end{lem}

\begin{proof}
    Let $f\colon X \to Y$ be a morphism in $\SetD$, we obtain a square 
    \[
    \begin{tikzcd}
        X 
        \ar[d, "f"]
        \ar[r, "\beta_X"]
        &
        \Ex(X)
        \ar[d, "\Ex(f)"]
        \\
        Y 
        \ar[r, "\beta_Y"]
        &
        \Ex(Y)
    \end{tikzcd}
    \qquad
    \begin{tikzcd}
        (\Delta^n \xrightarrow{x} X)
        \ar[d]
        \ar[r, mapsto]
        &
        (\Sd(\Delta^n)\xrightarrow{\alpha_n}\Delta^n\xrightarrow{x}X)
        \ar[d, mapsto]
        \\
        (\Delta^n \xhookrightarrow{x} X \xrightarrow{f} Y)
        \ar[r, mapsto]
        &
        (\Sd(\Delta^n) \xrightarrow{\alpha_n} \Delta^n \to X \xrightarrow{f} Y)
    \end{tikzcd}
    \]
    Now one can check with the definitions of the morphisms that objects (right square) are mapped accordingly and everything commutes.
\end{proof}

We now discuss some properties of $X \mapsto \Ex(X)$

\begin{prop}
    The functor $\Ex\colon \SetD \to \SetD$ preserves filtered colimits.
\end{prop}

\begin{proof}
    Notice that $\Sd(\Delta^n)=N(S[n])$ is compact,that is $\Hom_{\SetD}(\Sd(\Delta^n),-)$ preserves filtered colimits.
    Let $J \to \SetD$ be a filtered diagram where $j \to X^{(j)}$.
    Let 
    \begin{align*}
        \Ex(\colim_{j\in J} X^{(j)})_n &= \Hom_{\SetD}(\Sd(\Delta^n), \colim_{j \in J} X^{(j)}) \\
        &\cong \colim_{j \in J}\Hom_{\SetD}(\Sd(\Delta^n), X^{(j)}) \\
        &= \colim_{j \in J} \Ex(X^{(j)})_n
    \end{align*}
\end{proof}

\begin{rmk}
    Let $\alpha_X = \overline{\beta_X} \in \Hom_{\SetD}(\Sd(X),X) \cong \Hom_{\SetD}(X, \Ex(X)) \ni \beta_X$
\end{rmk}

\begin{prop}
    For all $i\colon K \hookrightarrow L$ anodyne, $\Sd(i) \colon \Sd(K) \hookrightarrow \Sd(L)$ is anodyne.
\end{prop}

\begin{proof}
    see cisinski 3.1.18
\end{proof}

\begin{cor}
\label{Ex preserves Kan fibrations}
    Let $p\colon X \to Y$ be a Kan fibration and $\Ex(p) \colon \Ex(X) \to \Ex(Y)$ is a Kan fibration.
\end{cor}

\begin{proof}
    Take the square
    \[
    \begin{tikzcd}
        \Lambda_k^n
        \ar[d, hook, "\in \An"']
        \ar[r]
        &
        \Ex(X)
        \ar[d, "\Ex(p)"]
        \\
        \Delta^n 
        \rar
        &
        \Ex(Y)
    \end{tikzcd}
    \]
    which gives by applying the definition of $\Ex$
    \[
    \begin{tikzcd}
        \Sd(\Lambda_k^n)
        \ar[d, hook, "\in \An"']
        \ar[r]
        &
        X
        \ar[d, "\Ex(p)"]
        \\
        \Sd(\Delta^n )
        \ar[ru, dashed]
        \rar
        &
        Y
    \end{tikzcd}
    \]
    where we can find a lift since $\An$ and Kan fibrations are a weak factorisation system.
\end{proof}

\begin{cor}
    Let $X \in \SetD$ be a Kan complex, then $\Ex(X)$ is a Kan complex.
\end{cor}

\begin{proof}
    Since $X$ is a Kan complex, $(X \to \Delta^0) \in \KanFib$, thus by the previous corollary $\Ex(Y) \to \Ex(\Delta^0) \in \KanFib$ and $\Ex(\Delta^0)\cong \Delta^0$ since $\Ex$ preserves coolimits. 
\end{proof}

\begin{thm}
    For all $X \in \SetD$ we have that $\beta_X \colon X \to \Ex(X)$ is a weak equivalence. 
\end{thm}

\begin{proof}
    We are only going to give a sketch of the proof.
    Let $X$ be a Kan complex. 
    Take the square induced by $\pi_0$
    \[
    \begin{tikzcd}
        \pi_0(\underline{\Hom}(\Ex(X),K)) 
        \ar[d, "\beta_K \circ ?"]
        \ar[r, "? \circ \beta_X"]
        &
        \pi_0(\underline{\Hom}(X,K))
        \ar[d, "\beta_K \circ ?"]
        \ar[ld, "f\mapsto \Ex(f)"']
        \\
        \pi_0(\underline{\Hom}(\Ex(X),\Ex(K))
        \ar[r, "? \circ \beta_X"]
        &
        \pi_0(\underline{\Hom}(X, \Ex(K)))
    \end{tikzcd}
    \]
    proof can be found in Goerss-Jardine III thm 4.6.
\end{proof}

\begin{cor}
\label{Ex preserves and reflects weak eq.}
    The morphism $(f\colon X \to Y)$ is a weak equivalence if and only if
    $(\Ex(f)\colon \Ex(X) \to \Ex(Y))$ is a weak equivalence.
\end{cor}

\begin{proof}
    Take the square
    \[
    \begin{tikzcd}
        X
        \ar[r, "\circ" marking, "\beta_X"]
        \ar[d, "f"]
        \ar[dr]
        &
        \Ex(X)
        \ar[d, "\Ex(f)"]
        \\
        Y
        \ar[r, "\circ" marking, "\beta_Y"]
        &
        \Ex(Y)
    \end{tikzcd}
    \] 
    Now the assertion follows by applying \cref{2 out of 3 weak equivalences}.
\end{proof}

\begin{defi}
    For $X \in \SetD$ we let $\Exinfty = \colim ( X \xrightarrow{\beta_X} \Ex(X) \xrightarrow{\beta_{\Ex(X)}}\Ex^2(X) \to \dotsc)$ this defines an Endofunctor of $\SetD$ and a natural transformation $\beta^{\infty}\colon \mathds{1} \to \Exinfty$.
\end{defi}

\begin{prop}
    The functor $\Exinfty\colon \SetD \to \SetD$ preserves filtered colimits since for all $n\geq 0$ the functor $\Ex^n$ preserves filtered colimits and $\Exinfty$ is defined as a filtered colimit of these.
    Furthermore $\Exinfty$ preserves finite limits since for all $n \geq 0$, $\Ex^n$ preserves all limits and finite limits commute with filtered colimits in $\SetD$ hence also in $\SetD$.
\end{prop}

\begin{prop}
Trivial Kan Fibrations are closed under filtered colimits in $\Fun([1], \SetD)$.
\end{prop}

\begin{prop}
    Let $f \colon X \to Y$ be a Kan fibration then $\Exinfty(f)$ is a Kan fibration.
\end{prop}

\begin{proof}
    This is an application of \cref{Ex preserves Kan fibrations} to all factors of the colimit.
\end{proof}

\begin{cor}
    The morphism $\beta_X^\infty \colon X  \to \Ex(X)$ is a weak equivalence.
\end{cor}

\begin{proof}
    This is an application of \cref{Ex preserves and reflects weak eq.} to all factors of the colimit.
\end{proof}

%\begin{thm}
%%    For all $X \in \SetD, \Exinfty(X)$ is a Kan complex.
%\end{thm}

%\todo{add proof, goerss jardine III.4.7.}