\section{Quillens small object argument}

For references for this section see \cite[Section 12.2]{Riehl_2014}.

Let $\mathcal{C}$ be a category that has all small colimits, let furthermore $\mathcal{C}$ be a cocomplete category (e.g. $\SetD$)
and $J$ a set of morphisms in $\mathcal{C}$

\underline{Aim}: For all $f$ in $\mathcal{C}$ construct a factorisation under some assumption in $J$.
\[
\begin{tikzcd}
    &
    \bullet
    \ar[rd, "{\in J^{\lift{}}}"]
    &
    \\
    \bullet
    \ar[ru, " \prescript{\lift{}}{}{(J^{\lift{}})} \ni"]
    \ar[rr, "f"']
    &&
    \bullet
\end{tikzcd}
\]

That is $J= \{ \Lambda_k^n \xhookrightarrow{} \Delta^n \mid n\geq 1, 0 \leq k \leq n\}$ and then $J^{\lift{}}=$Kan fibrations, or $J=\{ \partial \Delta^n \hookrightarrow \Delta^n \mid n \geq 0 \}$ and then $J^{\lift{}}=$Trivial Kan fibrations.
But we also want this factorisation to be functorial.

\begin{defi}
    A functorial factorisation in $\mathcal{C}$ is a section
    $\Fun([1], \mathcal{C}) \to \Fun([2], \mathcal{C})$
    of the composition functor $\Fun([2], \mathcal{C}) \xrightarrow{? \circ d_1}\Fun([1], \mathcal{C})$
    where $([1] \xrightarrow{d_1} [2] \to \mathcal{C})$.
\end{defi}

Let us unravel the definition:
For all morphisms $x \xrightarrow{f}y$ in $\mathcal{C}$ we get a $2$-simplex
\[
    \begin{tikzcd}
        &
        U(f) \ar[rd,"Rf"]
        &
        \\
        x
        \ar[ru, "Lf"]
        \ar[rr, "f"']
        &&
        y
    \end{tikzcd}
\]
in $\mathcal{C}$.
For all commutative squares 
\[
\begin{tikzcd}
    x
    \ar[r, "f"]
    \ar[d, "a"']
    &
    y
    \ar[d, "b"]
    \\
    x'
    \ar[r, "f'"']
    &
    y'
\end{tikzcd}
\]
in $\mathcal{C}$, we get a diagram
\[
\begin{tikzcd}
    &
    U(f)
    \ar[dd, dashed, "{U(a,b)}" pos=0.7]
    \ar[dr,"R(f)"]
    \\
    x
    \ar[dd,"a"']
    \ar[rr,"f" pos=0.3]
    \ar[ru,"Lf"]
    &&
    y
    \ar[dd,"b"]
    \\
    &
    U(f')
    \ar[rd,"Rf'"]
    \\
    x'
    \ar[ru,"Lf'"]
    \ar[rr,"f'"']
    &&
    y'
\end{tikzcd}
\]

\begin{defi}
\label{weak factorisation system}
    A weak factorisation system $(\mathcal{L},\mathcal{R})$ in $\mathcal{C}$ is a pair of classes of morphisms such that the following properties hold:
    \begin{enumerate}
        \item 
        (Factorisation) For all morphism $f:x\to y$ in $\mathcal{C}$ there exists a $2$-simplex
        \[
        \begin{tikzcd}
            &
            z
            \ar[rd,"\in \mathcal{R}"]
            \\
            x
            \ar[ru, " \mathcal{L} \ni"]
            \ar[rr, "f"']
            &&
            y
        \end{tikzcd}
        \],
        \item 
        (Lifting) $\mathcal{L} \lift{} \mathcal{R}$,
        \item 
        (Closure) $\mathcal{L} = \prescript{\lift{}}{}{\mathcal{R}}$ and $\mathcal{L}^{\lift{}} = \mathcal{R}$.
    \end{enumerate}
\end{defi}

\begin{lem}
\label{The retract argument}
Suppose 
$
\begin{tikzcd}
    \bullet
    \ar[d, "f"]
    \ar[r, "l"]
    &
    \bullet
    \ar[d, "r"]
    \\
    \bullet 
    \ar[r, equal]
    &
    \bullet
\end{tikzcd}
$
and $f \lift{} r$. Then $f$ is a retract of $l$ as objects in the arrow category, that is $\Fun([1], \mathcal{C})$.
\end{lem}

\begin{proof}
Since $f$ has the left lifting property with respect to $r$, we get a lift
\[
    \begin{tikzcd}
        \bullet
        \ar[d, "f"]
        \ar[r, "l"]
        &
        \bullet
        \ar[d, "r"]
        \\
        \bullet 
        \ar[ru, dashed, "w"]
        \ar[r, equal]
        &
        \bullet
    \end{tikzcd}
\]
We can rewrite this diagram
\[
    \begin{tikzcd}
        \bullet 
        \ar[r, equal]
        \ar[d, "f"]
        &
        \bullet
        \ar[r, equal]
        \ar[d, "l"]
        &
        \bullet
        \ar[d, "f"']
        \\
        \bullet
        \ar[rr, bend right, "\id"]
        \ar[r, dashed, "w"']
        &
        \bullet
        \ar[r, "r"]
        &
        \bullet
    \end{tikzcd}
\]
to obtain the result.
\end{proof}

\begin{lem}{Retract argument}
\label{Retract_argument}
    Suppose $(\mathcal{L}, \mathcal{R})$ satisfy the properties Factorisation and Lifting from \cref{weak factorisation system}.
    Then the property Closure holds if and only if $\mathcal{L}, \mathcal{R}$ are closed under retracts.
\end{lem}

\begin{proof}
    ($\implies$) Exercise
    ($\impliedby$) The inclusion $\mathcal{L} \subseteq \prescript{\lift{}}{}{\mathcal{R}}$ holds by assumption. 
    Let $k \in \prescript{\lift{}}{}{\mathcal{R}}$, by factorisation there exists a square 
    \[
    \begin{tikzcd}
        \bullet 
        \ar[r, "l \in \mathcal{L}"]
        \ar[d, "k"]
        &
        \bullet
        \ar[d, "r \in \mathcal{R}"]
        \\
        \bullet
        \ar[r, equal]
        &
        \bullet
    \end{tikzcd}
    \]
    By the retract argument \cref{Retract_argument} $k$ is a retract of $l \in \mathcal{L}$, hence $k \in \mathcal{L}$ since $\mathcal{L}$ is closed under retracts $\prescript{\lift{}}{}{\mathcal{R}}$. 
    Dually $\mathcal{L}^{\lift{}} \subseteq \mathcal{R}$.
\end{proof}

\begin{thm}{Quillen's small object argument}
\cref{Riehl_2014}
\label{Quillen's small object argument}
    Let $\mathcal{C}$ be a cocomplete category, $J$ a set of morphisms in $\mathcal{C}$ suppose that for all $j \in J$ we have that $\Hom_{\mathcal{C}}(\dom j,-)\colon \mathcal{C} \to \Set$ preserve (countable) sequential colimits (that is colimits of shape $(\mathbb{N}, \leq )$).
    Then there exists a functorial factorisation in $\mathcal{C}$ turning $(\prescript{\lift{}}{}{(J^{\lift{}})},J^{\lift{}})$ into a weak factorisation system.
    Moreover $\prescript{\lift{}}{}{(J^{\lift{}})}$ is the saturation of $J$.
\end{thm}

\begin{proof}
    Let $f$ be a morphism in $\mathcal{C}$.
    For $j \in J$, let $S_q(j,f)=
    \left\{ 
        \begin{tikzcd}
        \bullet
        \ar[d, "j"]
        \ar[r]
        &
        \bullet
        \ar[d, "f"]
        \\
        \bullet 
        \ar[r]
        &
        \bullet
        \end{tikzcd}
    \right\}$
    Take the following coproduct in the category of arrows 
    \[
    \coprod_{j\in J}\coprod_{S_q(j,f)}
    \begin{tikzcd}
        \bullet
        \ar[d, " \coprod_{j\in J}\coprod_{S_q(j,f)}j"']
        \ar[r, "d_f"]
        &
        \bullet
        \ar[d, "f"]
        \\
        \bullet 
        \ar[r, "c_f"']
        &
        \bullet
    \end{tikzcd}
    \]
    Consider now the pushout:
    \[
    \begin{tikzcd}
        \bullet
        \ar[r, "d_f"]
        \ar[d, " \coprod_{j\in J}\coprod_{S_q(j,f)}j"']
        \ar[r, phantom, shift right =3.4ex, "\PO" ]
        &
        \bullet
        \ar[d, "Lf"]
        \ar[r, equal]
        &
        \bullet
        \ar[d, "f"]
        \\
        \bullet
        \ar[rr, bend right,"c_f"']
        \ar[r,"b_f"']
        &
        x_1 
        \ar[r, "Rf"']
        &
        \bullet
    \end{tikzcd}
    \]
    By construction $Lf \in \prescript{\lift{}}{}{(J^{\lift{}})}$, since all of this is in the saturation of $ J $. but we have no guarantee that $Rf \in J^{\lift{}}$.
    Apply now the above construction to $Rf$ and thus we obtain a square:
    \[
    \begin{tikzcd}
        x_1 
        \ar[r, equal]
        \ar[d, "LRf"']
        &
        x_1
        \ar[d, "Rf"]
        \\
        x_2
        \ar[r, "R^2f"']
        &
        x_2
    \end{tikzcd}
    \]
    Repeating the construction iteratively we obtain a diagram:
    \[
    \begin{tikzcd}
        x_0
        \ar[r, "Lf"]
        \ar[rrd, bend right, "f"]
        \ar[rrrrr, bend left, "{L^wf \in \prescript{\lift{}}{}{(J^{\lift{}})}}" ]
        &
        x_1
        \ar[r,"LRf"]
        \ar[rd, bend right, "Rf"]
        &
        x_2
        \ar[r, "LR^2f"]
        \dar["R^2f"]
        &
        x_3
        \ar[r]
        \ar[ld, bend left,"R^3f"]
        &
        \dotsc 
        \rar
        &
        x_w=\colim_n x_n
        \ar[llld, bend left, "\exists ! R^wf"]
        \\
        &&
        \bullet
    \end{tikzcd}
    \]
    Where $L^wf$ is the transfinite composition of the $LR^if$ for $i \in \mathbb{N}$ and $R^{i+1}f \circ LR^if=R^if$, with $R^0f=f$.

    We claim that $R^wf \in J^{\lift{}}$.
    Consider the following lifting problem
    \[
    \begin{tikzcd}
        j_0
        \ar[r,"u"]
        \ar[d,"j"]
        &
        x_w
        \ar[d, "R^wf"]
        \\
        \bullet 
        \ar[r,"v"]
        &
        \bullet
    \end{tikzcd}
    \]
    Per definition we have that $ \Hom_{\mathcal{C}} ( j_0 , x_w \cong \colim_n \Hom ( j_0 , x_n ))$ and with the square above we obtain the following commutative diagram
    \[
    \begin{tikzcd}
    	j_0
    	\ar[r, "u"]
    	\ar[d, "j"]
    	\ar[rr, bend left, "u_w"]
    	&
    	x_n
    	\ar[d, "R^nf"]
    	\ar[r]
    	&
    	x_w
    	\ar[d,"R^wf"]
    	\\
    	\bullet
    	\ar[r, "v"]	
    	&
    	\bullet
    	\ar[r, equal]
    	&
    	\bullet
    \end{tikzcd}
	\]
	We now toke the leftmost square in the diagram above, then include it into the coproduct of squares $\amalg_{j \in J} \amalg_{S_q(j',R^nf)}$, which leads to the following.  
	\[
	\begin{tikzcd}
		j_0
		\ar[r]
		\ar[d,"j"']
		\ar[rr, bend left, "u"]
		&
		\bullet
		\ar[d, "\amalg_{j \in J} \amalg j'"']
		\ar[r]
		\ar[rd, phantom, "\PO"]
		&
		x_n
		\ar[d, "LR^nf"]
		\ar[r, equal]
		&
		\bullet
		\ar[d, "R^nf"]
		\\
		\bullet
		\ar[rrr, bend right=60, "v"']
		\ar[r]
		\ar[rr, bend right, red]
		& 
		\bullet
		\ar[r]
		&
		x_{n+1}
		\ar[r, "R^{n+1}f"']
		&
		\bullet
	\end{tikzcd}
	\]
	We can now construct the diagram below, where the red arrow corresponds to the red arrow above, which is our desired lift.
	\[
	\begin{tikzcd}
		j_0
		\ar[r, "u"]
		\ar[d]
		&
		x_n
		\ar[r, "LR^nf"]
		&
		x_{n+1}
		\ar[d, "R^{n+1}f"]
		\ar[r]
		&
		x_w
		\ar[d, "R^wf"]
		\\
		\bullet
		\ar[rru, red]
		\ar[r, "v"]
		&
		\bullet
		\ar[r, equal]
		&
		\bullet 
		\arrow[r, equal]
		&
		\bullet
	\end{tikzcd}
	\]
	Remains to show that $\prescript{\lift{}}{}{(J^{\lift{}})}$ is the saturation of $ J $. Let for that $c \in \prescript{\lift{}}{}{(J^{\lift{}})}$ and 
	\[
	\begin{tikzcd}
		\bullet
		\ar[r, " L^wc"]
		\ar[d, "c"]
		&
		\bullet
		\ar[d,"R^wc \in J^{\lift{}}"]
		\\
		\bullet
		\ar[r, equal]
		&
		\bullet
	\end{tikzcd}
	\] 
 	be a lifting diagram, that is $ c \lift{} R^wc$.
	Now by the retract argument $c$ is a retract of $ L^w c $.
\end{proof}