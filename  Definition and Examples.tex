\section{Modelcategories: Defintion and Examples}

Lecture 16.01

For references see \cite{Cisi}, \cite{Goers}

\begin{defi}{Quillen}
    A \underline{model category structure} on $\mathcal{C}$ is a triple $(W, \Cofib, \Fib)$ consisting of three classes of morphisms in $\mathcal{C}$.
    \begin{itemize}[label={}]
        \item 
        $W$: weak equivalences
        \item 
        $\Cofib$: Cofibrations
        \item  
        $\Fib$: Fibrations
        \item 
        $W \cap \Cofib$: trivial Cofibrations
        \item 
        $W \cap \Fib$: trivial Fibrations
    \end{itemize}
    subject to the following axioms:
    \begin{enumerate}
        \item 
        $W$ has the 2 out of 3 property,
        \item
        the triples $(\Cofib, W \cap \Fib)$ and $( \Cofib \cap W, \Fib)$ are weak factorisation systems.
    \end{enumerate}
    We call $(\mathcal{C}, W, \Cofib, \Fib )$ a \underline{model category}.
\end{defi}

\begin{rmk}
    Let us introduce some more convenient terminology.
    \begin{itemize}
        \item 
        We call $\mathcal{C}_{\Cofib} \coloneqq \{ A \in \mathcal{C} \mid \emptyset \to A \in \Cofib\}$ the \underline{cofibrant objects},
        \item 
        $\mathcal{C}_{\Fib} \coloneqq \{ A \in \mathcal{C} \mid A \to * \in \Fib\}$ the \underline{fibrant objects} and 
        \item   
        $\mathcal{C}_{\Bifib} \coloneqq \mathcal{C}_{\Cofib} \cap \mathcal{C}_{\Fib}$ the \underline{bifibrant objects}.
    \end{itemize}
\end{rmk}

\begin{thm}{Quillen}
    The quadruple $( \SetD ,$ Weq, Monos, Kanfib $)$ is a model category, it is called the Kan-Quillen model structure.
    We have the equalities $\An=$Weq $\cap$ Monos and TrivKan$=$ Weq $\cap$ KanFib. 
\end{thm}

\begin{rmk}
    Notice that for an $X \in \SetD$ is fibrant is per defintion equivalent to it being a Kan complex.
\end{rmk}

\begin{thm}
    Let $f\colon X  \to Y$ be a morphism in $\SetD$ then the following are equivalent:
    \begin{enumerate}
        \item 
        for all Kan complexes $K$ the induced morphism $f^*\colon \pi_0(\underline{\Hom}(Y,K)) \isomorphism \pi_0(\underline{\Hom}(X,K))$ is an isomorphism,
        \item 
        the morphism $f$ is a homotopy equivalence after applying the geometric realisation functor
        $\lvert f \rvert : \lvert X \rvert \to \lvert Y \rvert$,
        \item 
        for all $x \in X$ and all $n \geq 0$ the induced morphism on homotopy groups of the geometric realisation is an isomorphism $\lvert f \rvert^* \colon  \pi_n(\lvert X \rvert, x) \isomorphism \pi_n(\lvert Y \rvert, f(x))$.
    \end{enumerate}
\end{thm}

\underline{Motivation}

Let $(\mathcal{C},W, \Cofib, \Fib)$ be a model category.
Our aim is to study the homotopy category of $\mathcal{C}$, given as $\Ho(\mathcal{C})=\mathcal{C}[W^{-1}]$

\begin{defi}
    A \underline{Cylinder object} for $A \in \mathcal{C}$ is a factorisation 
    \[
    \begin{tikzcd}
        A \coprod A 
        \ar[rr, "{(1_A,1_A)}"]
        \ar[rd, "{(\partial_0, \partial_1)}"']
        &&
        A
        \\
        &
        I_A
        \ar[ru, "\sigma \in W"']
    \end{tikzcd}
    \]
    Dually a \underline{path object} for $A$ is a factorisation:
    \[
    \begin{tikzcd}
        &
        A^I
        \ar[rd, "{(d^0,d^1)\in \Fib}"]
        \\
        A
        \ar[rr, "{(1_A,1_A}"']
        \ar[ru, "s \in W"]
        &&
        A \times A
    \end{tikzcd}
    \]
\end{defi}

\begin{defi}
    Let $f\colon A \to B$ be a morphism in $\mathcal{C}$. 
    A left homotopy $f\sim_Lg$ is a choice of cylinder object 
    \[
    \begin{tikzcd}
        A
        \ar[d, "\partial_0"']
        \ar[rd, "f"]
        \\
        I_A
        \ar[r, "h"]
        &
        B
        \\ 
        A
        \ar[u, "\partial_1"]
        \ar[ru, "g"]
    \end{tikzcd}
    \]
    There is the dual notion of right homotopy $f \sim_R g$ given by the choice of a path object $B^I$.
\end{defi}

\begin{thm}{Quillen}
The following statements hold:
    \begin{enumerate}
        \item 
        Left homotopy is an equivalence relation on $\Hom_{\mathcal{C}}(A,B)$ when $A$ is cofibrant.
        \item 
        Let $A \in \mathcal{C}_{\Cofib}$ and $ \in \mathcal{C}_{\Fib}$ then $f \sim_L g \iff f \sim_R g$ for morphisms $f$ and $g$ from $A$ to $B$.
        \item 
        The map 
        \begin{align*}
            [-,-] \colon & \mathcal{C}^{\op}_{\Cofib} \times \mathcal{C}_{\Fib} \to \Set
            \\
            &(A,B) \mapsto \Hom_{\mathcal{C}}(A,B)/\sim
        \end{align*}
        is a functor.
        \item 
        The inclusions $\mathcal{C}_{\Cofib} \hookrightarrow \mathcal{C} \hookleftarrow\mathcal{C}_{\Fib}$ induce equivalences of homotopy categories
        $ \Ho( \mathcal{C}_{\Cofib}) \isomorphism \Ho(\mathcal{C})$ and $\Ho( \mathcal{C}_{\Cofib} )$.
        \item 
        The functor in 3. induces a functor 
        \[
        [-,-]\colon \Ho( \mathcal{C_{\Cofib}})^{\op} \times  \Ho(\mathcal{C}_{\Fib}) \to \Set.
        \]
        \item 
        For all $A \in  \mathcal{C}_{\Cofib}$ and all $B \in \mathcal{C}_{\Fib}$ we have that $[A,B] \cong  \Hom_{\Ho(\mathcal{C})}(A,B)$.
        Furthermore we write $\pi ( \mathcal{C}_{\Bifib})$ for the category with $\Ob(\pi ( \mathcal{C}_{\Bifib}))=\mathcal{C}_{\Bifib}$ and $\Hom_{\pi ( \mathcal{C}_{\Bifib})}=[A,B]$ for all $A,B \in \pi ( \mathcal{C}_{\Bifib})$.
        %\item 
        %Let $A,B \in \mathcal{C}_{\Bifib}$ and $f\colon A  \to B$ is a homotopy %equivalence, that is an isomorphism in $\pi(\mathcal{C}_{\Bifib})$
        %\todo{can not read my handwriting}
        \item 
        $\pi( \mathcal{C}_{\Bifib}) \cong \Ho(\mathcal{C})$
    \end{enumerate}
\end{thm}

\begin{exmp}
    Examples of model categories
    \begin{enumerate}
        \item 
        $R=$quasi-Frobenius ring ( $\Proj R = \Inj R$), $\underline{\Mod} R\colon$ stable category of $R$-modules 
        \item 
        $\underline{\Hom}_R(M,N)=\Hom_R(M,N)/S$, where the set $S$ is given by $\{f \colon M \to N \mid 
        \begin{tikzcd}
            M 
            \ar[rr, "f"]
            \ar[rd]
            &&
            N
            \\
            &
            P 
            \ar[ru]
        \end{tikzcd}
        P$ projective $\}$
        \item 
        $W=$ stable isomorphisms $=\{ f \text{ in } \Mod R \mid [f] \text{ isomorphism in } \underline{\Mod} R\}$
        $f \sim g \iff [f] = [g]$ in $\underline{\Mod} R$, i.e. $([f-g]=[0])$
        $\pi(\Mod R) \cong \Ho(\Mod(R)) \cong \underline{\Mod}_R$
        \item
        Let $R$ be a ring and $\Ch(\Mod R)$ the cochain complexes of $R$-modules and $W=$ quasi-isomorphisms and the fibrations are given by epimorphisms.
        
        Then $(\Ch(\Mod R), \text{ Quasi-iso's }, \prescript{\lift{}}{}{(\text{ Quasi-iso's } \cap \text{ epi's })}, \text{ epi's })$ is a model category called the projective model structure.
        Every object is fibrant. 
        If $X$ is cofibrant then for all $n \in \mathbb{Z}$ the object $X^n$ in degree $n$ is projective and it is bounded below.
        Note that then 
        \[
        K(\Cofib) \cong \Ho(\Ch(\Mod R)) \cong D ( \Mod R)
        \]
        \item 
        Let $R$ be a ring and $\Ch/( \Mod R)$ the cochain complexes of $R$-modules, let $W=$ quasi-isomorphisms and $\Cofib=$ monomorphisms.
        
        Then $(\Ch(\Mod R), \text{ Quasi-iso's }, \prescript{\lift{}}{}{(\text{ Quasi-iso's } \cap \text{ mono's })}, \text{ mono's })$ is a model category (injective model structure).
        For all $X \in  \Ch( \Mod R)$ the morphism $0 \to X$ is a cofibration.
        Furthermore if $X$ is a fibrant object, then for all $n \in  \mathbb{Z}$ $X^n$ is an injective module and bounded below.
        \item 
        Let $\SetD$ be the category of simplicial sets.
        Let $W=$ weak cat. equivalences, that is $f\colon X\to Y$ is in $W$ if for every $\mathcal{C} \in \SetD$ the induced morphism $f^*\colon \tau( \underline{\Hom}(Y, \mathcal{C}) \isomorphism \tau(\underline{\Hom}(X, \mathcal{C}))$ is an isomorphism.

        Then $(\SetD, \text{ Weak cat eq. , mono's, (Weak cat eq. $\cap$ mono's)$^{\lift{}}$})$ is called the Joyal model structure.
        Where $X$ is fibrant if and only if $X$ is an $\infty$-category.  
    \end{enumerate}
\end{exmp}