\section{Modelcategories: Defintion and Examples}

Lecture 16.01

\begin{defi}{Quillen}
    A \underline{model category structure} on $\mathcal{C}$ is a triple $(W, \Cofib, \Fib)$ consisting of three classes of morphisms in $\mathcal{C}$.
    \begin{itemize}[label={}]
        \item 
        $W$: weak equivalences
        \item 
        $\Cofib$: Cofibrations
        \item  
        $\Fib$: Fibrations
        \item 
        $W \cap \Cofib$: trivial Cofibrations
        \item 
        $W \cap \Fib$: trivial Fibrations
    \end{itemize}
    subject to the following axioms:
    \begin{enumerate}
        \item 
        $W$ has the 2 out of 3 property,
        \item
        the triples $(\Cofib, W \cap \Fib)$ and $( \Cofib \cap W, \Fib)$ are weak factorisation systems.
    \end{enumerate}
    We call $(\mathcal{C}, W, \Cofib, \Fib )$ a \underline{model category}.
\end{defi}

\begin{rmk}
    Let us introduce some more convenient terminology.
    \begin{itemize}
        \item 
        We call $\mathcal{C}_{\Cofib} \coloneqq \{ A \in \mathcal{C} \mid \emptyset \to A \in \Cofib\}$ the \underline{cofibrant objects},
        \item 
        $\mathcal{C}_{\Fib} \coloneqq \{ A \in \mathcal{C} \mid A \to * \in \Fib\}$ the \underline{fibrant objects} and 
        \item   
        $\mathcal{C}_{\Bifib} \coloneqq \mathcal{C}_{\Cofib} \cap \mathcal{C}_{\Fib}$ the \underline{bifibrant objects}.
    \end{itemize}
\end{rmk}

\begin{thm}{Quillen}
    The quadruple $( \SetD ,$ Weq, Monos, Kanfib $)$ is a model category, it is called the Kan-Quillen model structure.
    We have the equalities $\An=$Weq $\cap$ Monos and TrivKan$=$ Weq $\cap$ KanFib. 
\end{thm}

\begin{rmk}
    Notice that for an $X \in \SetD$ is fibrant is per defintion equivalent to it being a Kan complex.
\end{rmk}

\begin{thm}
    Let $f\colon X  \to Y$ be a morphism in $\SetD$ then the following are equivalent:
    \begin{enumerate}
        \item 
        for all Kan complexes $K$ the induced morphism $f^*\colon \pi_0(\underline{\Hom}(Y,K)) \isomorphism \pi_0(\underline{\Hom}(X,K))$ is an isomorphism,
        \item 
        the morphism $f$ is a homotopy equivalence after applying the geometric realisation functor
        $\lvert f \rvert : \lvert X \rvert \to \lvert Y \rvert$,
        \item 
        for all $x \in X$ and all $n \geq 0$ the induced morphism on homotopy groups of the geometric realisation is an isomorphism $\lvert f \rvert^* \colon  \pi_n(\lvert X \rvert, x) \isomorphism \pi_n(\lvert Y \rvert, f(x))$.
    \end{enumerate}
\end{thm}