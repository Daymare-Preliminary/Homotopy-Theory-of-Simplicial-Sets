\section{Weak equivalences of simplicial sets}

Recall that for $X \in \SetD$ and $K \in \SetD$ a Kan complex then $\underline{\Hom}(X,K)$ is a Kan complex.
Furthermore $f\colon X \to Y$ in $\SetD$ is a homtopy equivalence if there exists $g\colon Y \to X$ and homotopies $g \circ f \to \id_X$ and $ f \circ g \to \id_Y$.

\begin{exmp}
    Let $\begin{tikzcd}
            L \colon  \mathcal{C} 
            \arrow[r, shift left]
            &
            \mathcal{D} \colon R
            \arrow[l, shift left]
        \end{tikzcd}$
        be an adjunction $L \dashv R$.
        Then $N(L)\colon N( \mathcal{C} \to \mathcal{D})$ is a homotopy equivalence.
        Let $\eta \colon \mathds{1}_{\mathcal{C}} \to RL$ be the unit and $\epsilon\colon LR \to \mathds{1}_{\mathcal{D}}$ the counit of the adjunction, thus $\eta$ is a morphism in $\Fun( \mathcal{C} , \mathcal{C})$ and $\epsilon$ on $\Fun( \mathcal{D} , \mathcal{D} )$.
        Thus we can also consider $\eta \colon [1] \to \Fun(\mathcal{C},\mathcal{C})$ and $\epsilon \colon [1] \to \Fun(\mathcal{D},\mathcal{D})$.
        This can again be rephrased as $\eta \in \Fun( [1] , \Fun( \mathcal{C} , \mathcal{C})) \cong \Fun ( [1] \times \mathcal{C}, \mathcal{C}) \ni \overline{\eta}$ and $\epsilon \in \Fun( [1] , \Fun( \mathcal{D} , \mathcal{D})) \cong \Fun ( [1] \times \mathcal{D}, \mathcal{D}) \ni \overline{\epsilon}$.
        Now $N( \overline{\eta}) \colon N( [1] \times \mathcal{C}) \to N (\mathcal{C})$ and we have isomorphisms $N( [1] \times \mathcal{C}) \cong N([1]) \times N(\mathcal{C}) \cong \Delta^1 \times N(\mathcal{C})$.
        Take the nerve $N(\overline{\eta}) \colon N( \mathds{1}_{\mathcal{C}}) \to N(RL)= N(R) \circ N(L)$
        \[
        \begin{tikzcd}
            \mathcal{C} \cong [0] \times \mathcal{C}
            \ar[d, "L_0 \times \mathds{1}_{\mathcal{C}}"']
            \ar[rd, bend left, "\mathds{1}_{\mathcal{C}}"]
            \\
            [1] \times \mathcal{C}
            \ar[r, "\overline{\eta}"]
            &
            \mathcal{C}
            \\
            \overline{C}\cong [0] \times \mathcal{C} 
            \ar[u, "L_1 \times \mathds{1}_{\mathcal{C}}"]
            \ar[ru, bend right, "RL"']
        \end{tikzcd}
        \]
\end{exmp}

\begin{prop}
\label{Homtopy eq. of Kan complexes are bij in pi_0(weak eq.)}
    Let $f\colon X \to Y$ be a morphism between Kan complexes, then the following are equivalent:
    \begin{enumerate}
        \item 
        $f$ is a homtopy equivalence,
        \item 
        for all Kan complexes $K \in \SetD$ the morphism $\pi_0(f^*) \colon \pi_0(\underline{\Hom}(Y,K)) \to \pi_0(\underline{\Hom}(X,K))$ is bijective.
    \end{enumerate}
\end{prop}

\begin{defi}
\label{weak equivalence}
    A morphism of simplicial sets $f\colon X \to Y$ is a \underline{weak equivalence} if for all Kan complexes $K \in \SetD$ the morphism $\pi_0(f^*) \colon \pi_0(\underline{\Hom}(Y,K)) \to \pi_0(\underline{\Hom}(X,K))$ is bijective.
\end{defi}

\begin{lem}
\label{homotopy equivalence of inner hom}
    Let $f\colon X \to Y$ be a homotopy equivalence. 
    Then for all Kan complexes $K \in \SetD$ the morphism $f^*\colon \underline{\Hom}(Y,K) \to \underline{\Hom}(X,K)$ is a homotopy equivalence.
\end{lem}

\begin{proof}
    Let $h\colon \id_X \to g \circ f$ be a homotopy $(\Delta^1 \times X \xrightarrow{h} X)$.
    Let $h^*\colon \underline{\Hom}(X,K) \to \underline{\Hom}(\Delta^1 \times X,K) \cong \underline{\Hom}(\Delta^1 , \underline{\Hom}(X,K))$ be the morphism induced by $h$ on the inner $\Hom$ simplicial sets, so $h^* \in \underline{\Hom}(\Delta^1 \times \underline{\Hom}(X,K), \underline{\Hom}(X,K))$.
    Thus $h^*$ gives a homotopy between $(\id_X)^*=\id_{\underline{\Hom}(X,K)}$ and $(g \circ f)^* = f^* \circ g^*$.
\end{proof}

\begin{cor}
    Let $f \colon X  \to  Y$ be a homotopy equivalence of simplicial sets, then $f$ is a weak equivalence.
\end{cor}

\begin{proof}
    It is an application of \cref{homotopy equivalence of inner hom} and applying the definition of weak equivalences.
\end{proof}

\begin{prop}
    Let $i \colon X \hookrightarrow K$ be an anodyne extension, then $i$ is a weak equivalence.
\end{prop}

\begin{proof}
    Let $K$ be a Kan complex, then $i^* \colon \underline{\Hom}(Y,K) \to \Hom(X,K) $ is a trivial Kan fibration, then it is a homotopy equivalence, thus $\pi_0$ is an iso.
\end{proof}

\begin{prop}{(2 out of 3)}
\label{2 out of 3 weak equivalences}
    Weak equivalences satisfy the 2 out of 3 property. 
    That is for every commutative diagram
    \[
    \begin{tikzcd}
        &
        Y
        \ar[rd, "g"]
        &
        \\
        X
        \ar[ru, "f"]
        \ar[rr,"gf"']
        &&
        Z
    \end{tikzcd}
    \]
    if 2 out of the morphisms $f,g$ and $g\circ f$ are weak equivalences, then so is the third.
\end{prop}    

\begin{proof}
    Let $K \in \SetD$ be a Kan complex. 
    Then we get a diagram induced by the morphisms $f,g$ and $gf$.
    \[
    \begin{tikzcd}
        &
        \pi_0(\underline{\Hom}(Y,K))
        \ar[ld,"{\pi_0(f^*)}"']
        &
        \\
        \pi_0(\underline{\Hom}(X,K))
        &&
        \pi_0(\underline{\Hom}(Z,K))
        \ar[ll, "{\pi_0((g \circ f)^*)}"]
        \ar[lu, "{\pi_0(g^*)}"']
    \end{tikzcd}
    \]
    now if two of the morphisms are bijections, then so is the third, since bijections fullfill the 2 out of 3 property.
\end{proof}

\begin{prop}
    Let $f^{(i)}\colon X^{(i)} \to Y^{(i)}$ be a family of weak equivalences, indexed over the set $I$.
    Then $\coprod_{i \in I} f^{(i)}$ is a weak equivalence.
\end{prop}

\begin{proof}
    Let $K$ be a Kan complex 
    \[
    \begin{tikzcd}
        \underline{\Hom}(\coprod Y^{(i)},K)
        \ar[d,sloped, auto= false, "\sim"]
        \ar[r, "(\coprod f^{(i)})^*"]
        &
        \underline{\Hom}(\coprod Y^{(i)}, K)
        \ar[d,sloped, auto= false, "\sim"]
        \\
        \prod \underline{\Hom} ( Y^{(i)}, K) 
        \ar[r, "(f^{(i)*})_i"']
        &
        \prod \underline{\Hom}(X^{(i)},K)
    \end{tikzcd}
    \]
    Then $\underline{\Hom}(Y^{(i)},K)$ is a Kan complex, since $\pi_0$ preserves all small coproducts of Kan complexes, we are done.
\end{proof}

\begin{prop}
    Let $f^{(i)}\colon X^{(i)} \to Y^{(i)}$ be a family of weak equivalences of Kan complexes, indexed over a set $I$.
    Then $\prod f^{(i)}$ is a weak equivalence.
    By \cref{Homtopy eq. of Kan complexes are bij in pi_0} all weak equivalences here are homotopy equivalences.
\end{prop}

\begin{proof}
    Let $K$ be a Kan complex, then $\prod X^{(i)}$ and $\prod Y^{(i)}$ are Kan complexes.
    Then we have the following diagram
    \[
    \begin{tikzcd}
        \pi_0(\underline{\Hom}(K, \prod X^{(i)}))
        \rar
        \ar[d,sloped, auto= false, "\sim"]
        &
        \pi_0(\underline{\Hom}(K, \prod Y^{(i)}))
        \ar[d,sloped, auto= false, "\sim"]
        \\
        \pi_0(\prod \underline{\Hom}(K, X^{(i)}))
        \rar
        \dar
        &
        \pi_0(\prod \underline{\Hom}(K, Y^{(i)}))
        \dar
        \\
        \prod \pi_0(\underline{\Hom}(K, X^{(i)}))
        \ar[r, "sim"]
        &
        \prod \pi_0(\underline{\Hom}(K, Y^{(i)}))
    \end{tikzcd}
    \]
\end{proof}

\begin{rmk}
    For a finite product we do not need homtopy equivalence neither Kan complexes for the statement to hold.
\end{rmk}

\begin{prop}
    Let $f\colon X \to Y$ be a weak equivalence.
    Then for all Kan complexes $K$ the morphism $f^*\colon \underline{\Hom}(Y,K) \to \underline{\Hom}(X,K)$ is a homotopy equivalence.
\end{prop}

\begin{proof}
    Let $ W \in \SetD$ be a Kan complex.
    Then
    \[
    \begin{tikzcd}
        \pi_0(\underline{\Hom}(W, \underline{\Hom}(Y,K)))
        \ar[r, "f^* \circ ?"]
        \ar[d,sloped, auto= false, "\sim"]
        &
        \pi_0(\underline{\Hom}(W, \underline{\Hom}(X,K)))
        \ar[d,sloped, auto= false, "\sim"]
        \\
        \pi_0(\underline{\Hom}(Y, \underline{\Hom}(W,K)))
        \ar[r, "\pi_0(f^*)"]
        &
        \pi_0(\underline{\Hom}(X, \underline{\Hom}(W,K)))
    \end{tikzcd}
    \]
    Now all the inner $\Hom$ simplicial sets are Kan complexes and we are done by applying \cref{Homtopy eq. of Kan complexes are bij in pi_0(weak eq.)}.
\end{proof}

\begin{cor}
    Suppose that $f \colon X \to Y$ admits a factorisation 
    \[
    \begin{tikzcd}
        &
        Z
        \ar[rd, "\in \text{ Triv. Kan fibrations}"]
        &
        \\
        X
        \ar[ru, "\An \ni"]
        \ar[rr, "f"']
        &&
        Y
    \end{tikzcd}
    \]
    Then $f$ is a weak equivalence.
\end{cor}

\begin{proof}
    Anodyne extensions and trivial Kan fibrations are weak equivalences, $f$ is the composition of two weak equivalences hence a weak equivalence by \cref{2 out of 3 weak equivalences}.
\end{proof}

Let $X \in \SetD$ such that
\[
    \begin{tikzcd}
        &
        \Tilde{X}
        \ar[rd, "\in \KanFib"]
        &
        \\
        X
        \ar[ru, "\An \ni"]
        \ar[rr, "f"']
        &&
        \Delta^0
    \end{tikzcd}
\]
then $X$ is weakly equivalent to the Kan complex $\Tilde{X}$.


\subsection{Exercises}

\begin{Exercise}
    Consider a Kan complex $ X $ such that $ f \colon X  \to \Delta^0 $ is a weak homotopy equivalence.
    Our aim is to show that $ f $ is a trivial Kan fibration and thus in particular a homotopy equivalence.
    \begin{enumerate}[label=(\alph*)]
        \item 
        Construct a homotopy from $ \id_{\Delta^n}$ to a constant map.

        \item 
        Show that any morphism $ \Lambda_k^n \to X $ from a horn to a Kan complex is homotopic to a constant map.
    \end{enumerate}

    We now fix a lifting problem relative to a boundary inclusion
    \[
    \begin{tikzcd}
        \partial \Delta^n
        \ar[r, "g"]
        \ar[d]
        &
        X
        \ar[d, "f"]
        \\
        \Delta^n
        \ar[r]
        &
        \Delta^0
    \end{tikzcd}
    \]

    \begin{enumerate}[resume]
        \item 
        Use the homotopy lifting property to construct a morphism $ g' \colon \partial \Delta^n \to X $ which is homotopic to $ g $ and constant when restricted to $ \Lambda_n^n$.

        \item 
        Deduce from Exercise \cref{Exercise_11.2} that $ g' $ has a filling and construct a filling for $ g $.

        \item 
        Conclude that $ f $ is a trivial fibration.
    \end{enumerate}

\end{Exercise}

\begin{Exercise}
\label{Exercise commuting finite limits with colimits in set}
    We say a category $ K $ is $ \kappa$-filtered for an infinite cardinality $ \kappa $ if for any set of objects $ \{ \alpha_i \}_{ i \in I}$ with $ \lvert I \rvert < \kappa $ there is a cocone, i.e. there exists some $ \beta \in K $ and morphisms $ \alpha_i \to \beta $ for every $ i \in I $.
    Assume that $ K $ is $ \kappa $-complete and let $ J $ be a small category such that $ \lvert \Mor ( J ) \rvert < \kappa $.
    Show that for every functor $ D \colon K \times J \to \Set $ the canonical morphism 
    \[
        \colim_K \lim_{j \in J} D ( - , j ) \to \lim_J \colim_{\alpha \in K } D ( \alpha ,  ? )
    \]
    indueced by the morphisms $ \{ D ( \gamma , j ) \to \colim_{ \alpha \in K } D ( \alpha , j ) \}_{ j \in J }$ is a bijection. Fot thi, use the explicit form of limits and colimits in $ \Set$.
    Furthermore, describe the special case where $ \kappa = \lvert \mathbb{ N } \rvert$.
\end{Exercise}

\begin{Exercise}
    Fix a small category $ A $ and consider a presheaf $ X \in \widehat{A}$.
    Our gial is to show that $ X $ is $ \kappa$-compact for any sufficiently large regular cardinal $ \kappa $, i.e. the canonical map 
    \[
        \colim_{ \alpha \in \kappa } \Hom_{ \widehat{A}} ( X , D ( \alpha ) ) \to \Hom_{\widehat{ A }} ( X , \colim_{ \alpha \in \kappa } D ( \alpha ) )
    \]
    is an isomorphism for any $ \kappa $-indexed colimit $ D : \kappa \to \widehat{ A } $.
    Here, we view the cardinal $ \kappa $ as a well ordered set representing it.
    In particular, we may view $ \kappa $ as the category induced from the underlying partial order.
    Furthermore, a cardinal $ \kappa $ is called regular if the category associated to $ \kappa $ is $ \kappa $-complete in the sense of \cref{Exercise commuting finite limits with colimits in set}.
    \begin{enumerate}[label=(\alph*)]
        \item 
        Show that if $ X $ is representable, then $ X $ is $ \kappa $-compact for arbitrary $ \kappa $.

        \item 
        Deduce that to show that $ X $ is $ \kappa $-compact for some cardinal $ \kappa $, it is sufficient to show that $ \kappa$-indexed colimits commute with $ \int^A X $-indexed limits in $\Set$.

        \item 
        Show that the category of elements $ \int^A X $ is small, i.e. $ \Mor ( \int^A X ) $ is a set and thus has a cardinality.

        \item 
        Combine the above with \cref{Exercise commuting finite limits with colimits in set} to give a sufficient condition on $\kappa $ for $ X $ to be $ \kappa $-compact.
    \end{enumerate}
\end{Exercise}

\begin{Exercise}
    Let $ A $ be a small category and $ \mathcal{  F } $ is a set of morphisms on $ \widehat{ A } $. 
    Consider the set $ \dom ( \mathcal{ F } ) \coloneqq \{ Y \mid \exists f \in \mathcal{ F } f \colon Y \to Z \} $ of domains of morphisms in $\mathcal{F}$.

    \begin{enumerate}[label=(\alph*)]
        \item 
        Deduce from the Exercise above that there exists a cardinality $ \kappa $ such that every $ X \in \dom ( \mathcal{ F } ) $ is $ \kappa $-compact. 
        Here, you may freely use the fact from set theory that for any cardinal there exists a larger cardinality which is regular.

        \item
        Conclude $ ( l ( r ( \mathcal{ F } ) ), r ( \mathcal{ F } ) )$ is a weak factorisation system.

        \item 
        Deduce that $ \overline{ \mathcal{ F } } = l ( r ( \mathcal{ F } ) ) $ where $ \overline{ \mathcal{ F }}$ is the saturated closure of $ \mathcal{ F }$.
    \end{enumerate}
\end{Exercise}