\section{Reminder Function complexes}

For all $X,Y \in \SetD$ there exists a simplicial set $\underline{\Hom}_{\SetD}(X,Y)$ where 

$\underline{\Hom}_{\SetD}(X,Y)_n = \Hom_{\SetD}(\Delta^n \times X, Y)$ and for any $\sigma : \Delta^m \to \Delta^n$ we have 
\[
(\Delta^n \times X  \xrightarrow{\alpha} Y) \mapsto (\Delta^m \times X \xrightarrow{\sigma \times \id_X} \Delta^n \times X \xrightarrow{\alpha}Y)
\]
We have furthermore that 
\[
\Hom_{\Set_{\Delta}}(W \times X, Y) \isomorphism \Hom_{\Set_{\Delta}}(W,\underline{\Hom}_{\SetD}(X,Y))
\]
as well as 
\[
\underline{\Hom}_{\SetD}(\Delta^0,Y) \cong Y.
\]
Applying the above isomorphisms together yields
\[
\Hom_{\SetD}(Z, \underline{\Hom}_{\SetD}(\Delta^0,Y)) \cong \Hom_{\SetD}(Z\times \Delta^0,Y) \cong \Hom_{\SetD}(Z,Y)
\]
as well as 
\[
\underline{\Hom}_{\SetD}(\Delta^n,X)_0 = \Hom (\Delta^0 \times \Delta^n , X ) \cong \Hom(\Delta^n,X) \cong X_n.
\]
The adjunction isomorphisms are explicitely given as follows 
\[
\begin{tabular}{c}
    $\Hom(W, \underline{\Hom}(X,Y))\isomorphism \Hom(W \times X, Y)$
    \\
    $ (W \xrightarrow{\sigma} \underline{\Hom}(X,Y)) \mapsto ((\Delta^n \xrightarrow{w} W, \Delta^n \xrightarrow{x} X) \xmapsto{\overline{\sigma}}(\Delta^n\xrightarrow{(\id,x)} \Delta^n \times X \xrightarrow{\sigma(w)}Y)$.  
\end{tabular}
\]
Furthermore we have that 
\[
\begin{tabular}{c}
    $\ev_x\colon \underline{\Hom}(X,Y) \times X \to Y$
    \\
    $(\Delta^n \times X \xrightarrow{f} Y, \Delta^n \xrightarrow{x} X) \xmapsto{\ev_x}( \Delta^n \xrightarrow{(\id,x)} \Delta^n \times X \xrightarrow{f} Y)$
\end{tabular}
\]

We can turn this into the inner Hom-functor $\underline{\Hom}(X,-): \SetD \to \SetD$.
Given $i\colon L\to K$ in $\SetD$ define 
\[
\begin{tabular}{c}
    $i^*\colon \underline{\Hom}(K,X) \to \underline{\Hom}(L,X)$
    \\
    $(\Delta^n \times K \xrightarrow{f} X) \xmapsto{i^*}(\Delta^n \times L \xrightarrow{(\id, i)} \Delta^n \times K \xrightarrow{f} X)$   
\end{tabular}
\]
such that 
\[
\begin{tikzcd}
    \Delta^n 
    &
    \Delta^n \times L
    \ar[r, "\id \times i"]
    &
    \Delta^n \times K
    \ar[r, "f"]
    &
    X
    \\
    \Delta^m
    \ar[u,"\sigma"]
    &
    \Delta^m \times L
    \ar[r, "\id \times i"]
    \ar[u,"\sigma \times \id_L"]
    &
    \Delta^m \times K
    \ar[u, "\sigma \times \id"]
    \ar[ru, "\sigma^*(f)"']
\end{tikzcd}
\]

\subsection{Exercises}  

\begin{Exercise}
    
Consider a pair of adjunctions $ L \dashv R $ and $ L' \dashv R' $ between two categories $ \mathcal{ C } $ and $ \mathcal{ D } $.
A natural transformation of adjunctions $ \lambda : ( L \dashv R ) \to ( L' \dashv R' ) $ is a tuple of natural transformations $ \lambda = ( \lambda^L , \lambda^R ) $ where $ \lambda^L : L \to L'$ and $\lambda^R : R' \to R $ such that 
\[
\begin{tikzcd}
    \Hom_{\mathcal{ D } } ( L' ( - ) , ? )
    \ar[r,"\sim","\varphi'"']
    \ar[d, " (\lambda^L)^* "]
    &
    \Hom_{\mathcal{C}}(-,R'(?))
    \ar[d,"(\lambda^R)_*"]
    \\
    \Hom_{\mathcal{D}}(L(-),?)
    \ar[r,"\varphi"',"\sim"]
    &
    \Hom_{\mathcal{C}}(-,R(?))
\end{tikzcd}
\]
commutes.
Here $\varphi$ and $\varphi'$ are the respective adjunction isomorphisms.
Fix two morphisms $ i \colon A \to B $ in $ \mathcal{ C } $ and $ p : X \to Y $ in $ \mathcal{ D } $.
Assume that $ \mathcal{ C } $ admits pullbacks and $ \mathcal{ D } $ consider the following induced morphisms.

\begin{tikzcd}
    R'X 
    \ar[rrd, bend left, "\lambda_X^R"]
    \ar[rd, dashed, "{ \exists! ( R'p , \lambda ) }"]
    \ar[rdd, bend right, "R'p"']
    &
    &
    \\
    &
    R'Y \times_RY RX
    \ar[r, "\pr_X"]
    \ar[d, "\pr_Y"']
    &
    RX
    \ar[d, "Rp"]
    \\
    &
    R'Y 
    \ar[r,"\lambda_Y^R"']
    &
    RY
\end{tikzcd}
\qquad
\begin{tikzcd}
    LA
    \ar[r, "\lambda_A^L"]
    \ar[d, "Li"]
    &
    L'A
    \ar[d, "\iota_A"]
    \ar[rdd, bend left, "L'i"]
    &
    \\
    LB
    \ar[r, "\iota_B"]
    \ar[rrd, "\lambda_B^L"]
    &
    LB \amalg_LA L'A
    \ar[rd, "{\exists ! ( \lambda , L'i )}"]
    &
    \\
    &&
    L'B
\end{tikzcd}

Our aim is to show that $ ( R'p , \lambda ) \in r ( i ) $ if and only if $ ( \lambda, L'i) \in l(p)$.

\begin{enumerate}[label=(\alph*)]
    \item 
    Show that $ \big( ( R'p , \lambda ) \in r ( i ) \implies ( \lambda , L' i ) \in l ( p ) \big) $ is dual to $ \big( ( \lambda , L'i ) \in l ( p ) \implies ( R'p , \lambda ) \in r ( i ) \big)$.
\end{enumerate}

With this it suffices to show only one of the implications. 
So consider the following lifting problem.
\[
\begin{tikzcd}
    A 
    \ar[r, "f"]
    \ar[d, "i"]
    &
    R'X
    \ar[d, "{(R'p , \lambda)} "]
    \\
    B
    \ar[r, "{(g_Y , g_X)}"']
    &
    R'Y \times_RY RX
\end{tikzcd}
\]

\begin{enumerate}[label=(\alph*), resume]
    \item 
    Show that $ \varphi^{-1} ( g_X ) $ and $ ( \varphi' )^{ - 1 } ( f ) $ induce an unique morphism $ LB \amalg_{LA} L'A \to X $ and that this map fits in the following commuting square.
    \[
    \begin{tikzcd}
        LB \amalg_{LA} L'A 
        \rar
        \ar[d, "{ (\lambda , L' i ) }"']
        &
        X
        \ar[d, "p"]
        \\
        L'B 
        \ar[r, "{(\varphi')^{-1} (g_Y)}"']
        &
        Y
    \end{tikzcd}
    \]

    \item 
    Show that if $ \rho $ is a lift in the square from (b) then $ \varphi' ( b ) $ is a solution to the original problem.

    \item 
    Conclude that $ ( R'p , \lambda ) \in r(i) $ if and only if $ ( \lambda , L'i) \in l ( p ) $.

    \item 
    Describe the result if $ Y $ is final or $ A $ is initial.
\end{enumerate}

\end{Exercise}

\begin{Exercise}
    
\begin{enumerate}[label=(\alph*)]
    \item 
    Show that given morphsim of simplicial sets $ f : L \to K $, the pair $ \lambda \coloneqq ( \id_{(-)} \times f , f^* ) $ is a morphism between the adjunctions $ - \times L \dashv \Hom ( L , - ) $ and $ - \times K \dashv \Hom ( K , - ) $ of endofunctors of $ \SetD $.

    \item 
    Show that if $ j : L \to K $ is a monomorphism, then for any horn inclusion $ i : \Lambda_k^n \to \Delta^n $, the induced morphism $ ( \id_{ \Delta^n } \times j , i \times \id_K ) $ is an anodyne extension.

    \item 
    Deduce that if $ p $ is a Kan fibration and $ j : L \to K $ is a monomorphism, then the induced morphism
    \[
        ( p_* , j^* ): \Hom( K , X ) \to \Hom ( K , Y ) \times_{ \Hom ( L , Y ) } \Hom ( L , X )
    \]
    is again a Kan fibration.

    \item  
    Describe the special case where $ Y = \Delta^0 $.
    
\end{enumerate}
\end{Exercise}

