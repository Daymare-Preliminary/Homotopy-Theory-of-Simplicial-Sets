\section{Simplicial sets}

\begin{defi}
    The simplicial category $\Delta$ has objects $[n] \coloneqq \{ 0<1<2< \dotsc <n\}$ for $n \geq 0$ and morphisms $\Delta(m,n)\coloneqq \Hom_{\Delta}([m],[n]) \coloneqq \{ f \colon[m] \to [n], \text{order preserving} \}$.
\end{defi}

\begin{defi}
    The \underline{category of simplicial sets} is given by $\Set_{\Delta}=\sSet=\widehat{\Delta}= \Fun(\Delta^{\op},\Set)$.
\end{defi}

\begin{rmk}
    An alternative definiton of a simplicial set $X$ can be given as follows:
    \begin{itemize}
        \item 
            For all $n\geq m$ a set $X_n$ called the \underline{n-simplices} of $X$.
        \item 
            For all $0 \leq i \leq n$ morphisms $d_i\colon X_n \to X_{n+1}$ called the \underline{face maps}.
        \item 
            For all $0 \leq i \leq n$ morphisms $s_i \colon X_n \to X_{n+1}$
            called the \underline{degeneracy maps}.
        \item 
            The face and degeneracy maps satisfy the following identities:
            \[
            \begin{tabular}{ccc}
                 $d_id_j=d_{j-1}d_i$ & $d_is_j=s_{j-1}d_i$ & $d_js_j=\id=d_{j+1}s_j$  
                 \\
                 $i<j$ & $i<j$ & 
                 \\
                 $d_is_j=s_jd_{i-1}$ & $s_is_j=s_{j+1}s_i$
                 \\
                 $i>j+1$ & $i \leq j$
            \end{tabular}
            \]
    \end{itemize}
\end{rmk}

\begin{exmp}
    \begin{enumerate}
        \item 
        For an arbitrary simplicial set we often write
            \[
            \begin{tikzcd}
                X\colon \dotsc 
                &
                X_3
                \arrow[r, altstackar=7]
                &
                X_2
                \arrow[r, altstackar=5]
                &
                X_1
                \arrow[r,altstackar=3]
                &
                X_0
            \end{tikzcd}
            \]
        where the arrows correspond to the face and boundary maps.
        \item 
        \begin{tabular}{@{} lcl @{}}
            [0]=\{0\}
            \\
            \relax
            [1]=\{ 
            \begin{tikzcd}
                0
                \arrow[loop left, "id"]
                \arrow[r, "10"]
                &
                1
                \arrow[loop right, "id"]
            \end{tikzcd}
            \}
            \\
            \relax
            $[2]=\left\{
            \begin{tikzcd}
                &
                1
                \arrow[rd]
                &
                \\
                0
                \arrow[ru]
                \arrow[rr]
                &
                &
                2
            \end{tikzcd}  
            \right\}$
        \end{tabular}
    \end{enumerate}
    \todo{dis all fucked up, ... dont like how this looks}
\end{exmp}

By Yoned we obtain an isomorphism $\Hom_{\Set_{\Delta}}(\Delta(-,n),X) \isomorphism X_n$. 
Let $x \in X_n$ and let $\Delta^n \coloneqq \Delta(-,n)$ we obtain the following diagram
\[
\begin{tikzcd}
    \Delta^n
    \arrow[rr, "x"]
    \arrow[rd, "\sigma_x"']
    &
    &
    X
    \\
    &
    \Delta^m
    \arrow[ru , "\mu"']
\end{tikzcd}
\]
\todo{unsure what all is happening here}

Let $d^i\colon[n-1] \to [n]$ be the unique order preserving injective map not having $i \in [n]$ in its image for all $0 \leq i \leq n$ .
\[
\begin{tabular}{cc}
     \begin{tikzcd}
         \text{[}0]=\{0\}
         \arrow[r, "d^0"]
         &
         \{0 \to \circled{1}\}=[1]
     \end{tikzcd}
     \\
     \begin{tikzcd}
         \text{[}0]=\{0\}
         \arrow[r, "d^0"]
         &
         \{\circled{0} \to 1\}=[1]
     \end{tikzcd}
     \\
     \begin{tikzcd}
         \text{[}1]=\{ \encircled{0 \to 1}\}
         \arrow[r, "d^0"]
         &
         {}
     \end{tikzcd}
     &
     $\left\{
     \begin{tikzcd}
         &
         1
         \arrow[rd]
         &
         \\
         0
         \arrow[ru]
         \arrow[rr]
         &
         &
         2
     \end{tikzcd}
     \right\}=[2]$
\end{tabular}
\]

\todo{more stuff missing cause dk yet how to tex this}

\begin{defi}
    The category $\Delta_{\txtbig}$ has as objects the finite non-empty total orders with order preserving maps between them
    \[
    \begin{tabular}{c}
         \begin{tikzcd}
             \Delta
             \arrow[r, hook, shift right]
             &
             \Delta_{\txtbig} 
             \arrow[l, shift right] \ni I = \{ i_0 < i_1 < \dotsc < i_n \}
         \end{tikzcd}      
    \end{tabular}
    \]
\end{defi}

\todo{alotta diagrams i am too lazy for currently}

\begin{exmp}{the nerve of a category}
    Let $\mathcal{C}$ be a small category. 
    We define $N(\mathcal{C}) \in \Set_{\Delta}$ as follows. 
    Let $N(\mathcal{C})_0 \coloneqq \Ob(\mathcal{C})$, $N(\mathcal{C})_1=\Mor(\mathcal{C})\colon$ set of morphisms in $\mathcal{C}$ with the face and boundary maps given as follows
    \todo{again to many arrows that I am too tired for right now}
    
\end{exmp}

\begin{thm}
    For all $X \in \Set_{\Delta}$ $\lvert X \rvert = \colim_{\substack{([n],x)\in \int_X^\Delta \\
    X\colon \Delta^n \dashv X} }\lvert \Delta^n\rvert$ is a CW complex.
\end{thm}

\begin{defi}
    An element $x$ of simplicial set $X$ is \underline{degenerate} if $x \in \im s_i$ for some $i$.
\end{defi}

\begin{thm}(Dold-Kan Correspondences)

\end{thm}