\section{Kan Fibrations}

For references for this section see \cite[Section I.3]{GoerSimp1999}.

Aim: Formalize the idea of a "family of Kan complexes parametrized by a simplicial set", i.e. suitable morphisms $X \xrightarrow{p} Y$ in $\Set_{\Delta}$ whose fibres are Kan complexes.
That is we have a pullback diagram
\[
\begin{tikzcd}
    X_y
    \arrow[d]
    \ar[r]
    &
    X
    \ar[d,"p"]
    \\
    \Delta^0
    \ar[r,"y"]
    &
    Y    
\end{tikzcd}
\]

\underline{Terminology}:
Let $ L \xrightarrow{i} K$ and $ X \xrightarrow{p} Y$ be morphisms.
We say for the notation ($ i \lift{} p$) that $i$ has the left lifting property with respect to $p$ (equivalently $p$ has the right lifting property with respect to $i$).
If there exists for every $L \xrightarrow{f} X$ and $K \xrightarrow{g}Y$ a morphism $h:K \to X$ such that the following diagram commutes
\[
\begin{tikzcd}
    L
    \arrow[d,"i"]
    \ar[r, "f"]
    &
    X
    \ar[d,"p"]
    \\
    K
    \ar[ru,"h"]
    \ar[r,"g"]
    &
    Y    
\end{tikzcd}
\]
Let $X \in \SetD$ be a Kan complex. 
Then 
\[
\{ \Lambda_k^n \to \Delta^n \mid n\geq 1 ,0 \leq k \leq n \} \lift{} (X \to \Delta^0)
\]
is given by 
\[
\begin{tikzcd}
    \Lambda_k^n
    \arrow[d,"\iota"']
    \ar[r, "\sigma"]
    &
    X
    \ar[d,"!"]
    \\
    \Delta^n
    \ar[ru,"\Tilde{\sigma}"]
    \ar[r,"!"']
    &
    \Delta^0
\end{tikzcd}
\]

\begin{defi}
    A morphism $X \xrightarrow{p} Y$ is a Kan fibration if 
    \[
    \{ \Lambda_k^n \to \Delta^n \mid n\geq 1 ,0 \leq k \leq n \} \lift{} (X \xrightarrow{p} Y)
    \]
\end{defi}

\begin{rmk}
    We may as well write $X \xrightarrow{p} Y \in \{ \Lambda_k^n \to \Delta^n \mid n\geq 1 , 0 \leq k \leq n\}^{\lift{}}$.
\end{rmk}

\begin{lem}
    Let $i \lift{} p$ and
    \[
    \begin{tikzcd}
        X'
        \arrow[d,"p'"]
        \ar[r]
        &
        X
        \ar[d,"p"]
        \\
        Y'
        \ar[r,"v"]
        &
        Y    
    \end{tikzcd}
    \]
    be a pullback-square, then $i \lift{} p'$.
\end{lem}

\begin{proof}
    Consider the diagram
    \[
    \begin{tikzcd}
        L
        \ar[r,"f"]
        \ar[d,"i"']
        &
        X'
        \arrow[d,"p'" pos=0.65]
        \ar[r, "u"]
        &
        X
        \ar[d,"p"]
        \\
        K
        \arrow[r,"g"']
        \ar[rru, dashed, "r" pos= 0.4]
        \ar[ru, dashed, "h"]
        &
        Y'
        \ar[r,"v"']
        &
        Y    
    \end{tikzcd}
    \]
    where $r$ exists since $i$ has the left lifting property with respect to $p$ and $h$ exists by the universal property of the pullback.
    Now $u \circ h \circ i = r \circ i = u \circ f$ and $p' \circ h \circ i = q \circ i = p' \circ f$, since the  morphism given by the pullback is unique we get that $h \circ i = f$.
\end{proof}

Lecture 3.12 

Formalise the notion of a locally constant family of Kan complexes relative to a base simplicial set.

\begin{defi}
    Let $X \xrightarrow{p} Y$ be a Kan fibration. 
    By the preceding Lemma, we get that for every $y\in Y_0$ the fibre $X_y$ is a Kan complex. 
    Now consider $\Delta^0$ as the simplicial set where $\Delta^0_n=\Hom_{\SetD}([n],[0])=\{[n] \to [0]\}$ is the unique map into the final object.
    Furthermore we have the constant map $c=c^n\colon Y_0 \to  Y_n$, $y \mapsto c^n(y)$.
    Let $X \xrightarrow{p} Y$ be a Kan fibration.
    Where 
    \[
    \begin{tikzcd}
        \Lambda_1^1=\Delta^{\{1\}}_0
        \ar[r,"x"]
        \ar[d]
        &
        X
        \ar[d,"p"]
        \\
        \Delta^{\{0,1\}}
        \ar[r,"f"]
        \ar[ru, dashed, "\exists \Tilde{f}"]
        &
        Y
    \end{tikzcd}
    \]
\end{defi}


\begin{prop}
\label{comp_Kan_fib}
    Let $L \xrightarrow{i} K$ be a morphism and suppose that $X \xrightarrow{p} Y$, $Y \xrightarrow{g} Z$ have the right lifting property with respect to $i$, then the composition $g \circ p$ has the right lifting property with respect to $i$.
\end{prop}

\begin{proof}
    This is just a simple matter of writing down the square for $g$ and using the obtained morphism to write down the square $p$ obtaining a morphism that fullfills the desired property.
\end{proof}

\begin{cor}
    Suppose that $X \xrightarrow{p} Y$ is a Kan fibration and $Y$ is a Kan complex, then $X$ is a Kan complex.
\end{cor}

\begin{proof}
    Since $X \xrightarrow{p} Y$ is a Kan fibration and $Y \to \Delta^0$ is a Kan fibration, we get by \cref{comp_Kan_fib} that $X \to \Delta^0$ is a Kan fibration, thus we get that $X$ is a Kan complex.
\end{proof}

\begin{prop}
    Let $X^{(i)} \xrightarrow{p^i} Y^{(i)}$ with $i \in I$ be a set indexed family of Kan fibrations.
    Then $\prod X^{(i)} \xrightarrow{\prod p^i} \prod Y^{(i)}$ is a Kan fibration.
\end{prop}

\begin{proof}
Consider the diagram arising from the assumptions
\[
    \begin{tikzcd}
        \Lambda_k^n
        \ar[r,"\sigma"]
        \ar[d,"\iota"]
        &
        \prod X^{(i)}
        \ar[d, "\prod p^i" pos=0.7]
        \ar[r, "\prod \pi_j"]
        &
        X^{(j)}
        \ar[d, "p^j"]
        \\
        \Delta^n
        \ar[ru, dashed, "\exists ?"]
        \ar[rru, dashed, "h_j" pos=0.4]
        \ar[r, "\tau"']
        &
        \prod Y^{(i)}
        \ar[r, "\pi_j"']
        &
        Y^{(j)}
    \end{tikzcd}
\]
Since $p_j$ is a Kan fibration, there exists $h_j \colon \Delta^n \to X^{(j)}$ such that 
$\begin{cases}
    p_jh_j=\pi_j\tau\\
    h_j\iota = \pi_j \sigma_j.
\end{cases}$
    The universal property of the product gives $h \colon \Delta^n \to \prod X^{(i)}$ such that $\pi_j \circ h = h_j$.
    Now we have that $\pi_j(h \circ \iota ) = h_j \circ \iota = \pi_i \circ \sigma$ and thus $h \circ \iota = \sigma$ as well as $\pi_j(\prod {p_i} \circ h) = p_j \circ \pi_j \circ h = p_j h_j = \pi_j \tau$ and thus $\prod p_i \circ h = \tau$, where the equalities follow from the uniqueness of the morphism given by the pullback.
\end{proof}

\begin{prop}
    Let $\dotsc \to X^{(n)} \xrightarrow{p_n} \dotsc \to X^{(1)} \xrightarrow{p_1}X^{(0)}$ be a "tower" of Kan fibrations, that is for all $i \in \mathbb{N}, p_i$ is a Kan fibration.
    Then $X^{\infty} \coloneqq \lim X^{(n)} \xrightarrow{\pi_0} X^{(0)}$ is a Kan fibration. 
\end{prop}

\begin{proof}
    Consider the diagram 
    \[
    \begin{tikzcd}
        \Lambda_k^n
        \arrow[d,"\iota"']
        \ar[r, "\sigma"]
        &
        X^{(\infty)}
        \ar[d,"\pi_0"]
        \\
        \Delta^n
        \ar[ru, dashed, "\exists ?"]
        \ar[r,"\tau"']
        &
        X^{(0)}
    \end{tikzcd}
    \]
    The idea is to construct a cone of the tower with apex $\Delta^n$ and then use the universal property $X^\infty$.
    \[
    \begin{tikzcd}
        &&&
        \Delta^n
        \ar[llld, dashed, bend right]
        \ar[lld, " h_2 "' ]
        \ar[ld , " h_1 " ]
        \ar[d , " h_0 \coloneqq \tau_0" ]
        \\
        \dotsc
        \ar[r, "p_2"']
        &
        X^{(2)}
        \ar[r, "p_1"']
        &
        X^{(1)}
        \ar[r, "p_0"']
        &
        X^{(0)}
    \end{tikzcd}
    \]

    Suppose that $ 0 \leq s \leq t $ and that we have constructed $ h_s \colon \Delta^n \to X^{ ( s ) } $ such that $ p_s h_s = h_{s-1} $ and consider the case for $t+1$, where the diagram following on the left side is used to construct the top morphism in the right diagram.
    \[
    \begin{tikzcd}
        &
        \Lambda_k^n
        \ar[d, " \sigma "]
        \\
        &
        X^{(\infty)}
        \ar[d, " \pi_t "]
        \ar[dl, " \pi_{t+1} "']
        \\
        X^{(t+1)}
        \ar[r, " p_{t+1} "']
        &
        X^{(t)}
    \end{tikzcd}
    \qquad
    \begin{tikzcd}
        \Lambda_k^n
        \ar[d, " \iota "]
        \ar[r, " \pi_{t+1} \circ \sigma "]
        &
        X^{t+1}
        \ar[d, " p_{t+1}"]
        \\
        \Delta^n 
        \ar[ru, " \exists h_{t+1}"]
        \ar[r, "h_t"]
        &
        X^{(t)}
    \end{tikzcd}
    \]
    Since $ p_{t+1}$ is a Kan fibration, $ h_{t+1} $ exists and $h_{t+1} \circ \iota = \pi_{ t+1 } \circ \sigma $ as well as $ p_{ t + 1 } \circ h_{ t + 1 } = h_t $.
    By the universal property of $ X^{ ( \infty ) }$ we get 
    \[
    \begin{tikzcd}
        \exists! h \colon \Delta^n 
        \ar[r]
        \ar[rd, "h_t"']
        &
        X^{(\infty)}
        \ar[d, "\pi_t"]
        \\
        &
        X^{(t)}
    \end{tikzcd}
    \]
    such that for all $t, \pi_t \circ h = h_t $.
    Also for all $ t $,
    $ \pi_t \circ  h \circ \iota  = h_t \circ \iota = \pi_t \sigma $, since this holds in particular for the case $ t = 0 $ the statement is proven.
\end{proof}


\subsection{Exercises}

\begin{Exercise}
    Recall that for a category $ \mathcal{ C } $ and a class of morphisms $ \mathcal{ F } \subseteq \Mor ( \mathcal{ C } )$ the class of morphisms with the left lifting property with respect to $ \mathcal{ F } $ is 
    \[
        l ( \mathcal{ F } )
        \coloneqq  
        \Bigg\{ i \colon x \to y \mid \forall \begin{tikzcd}
            a 
            \ar[r, "f"]
            \ar[d, "i"]
            &
            x
            \ar[d, "p \in \mathcal{ F }"]
            \\
            b
            \ar[r, "g"]
            &
            y
        \end{tikzcd}
        \exists h
        \begin{tikzcd}
            a 
            \ar[r, "f"]
            \ar[d, "i"]
            &
            x
            \ar[d, "p "]
            \\
            b
            \ar[r, "g"]
            \ar[ru, dashed, "h"]
            &
            y
        \end{tikzcd}
        \Bigg\}
    \]
    where diagrams commute, i.e. $ p \circ f = g \circ i , f = h \circ i $ and $ g = p \circ h $.
    \begin{enumerate}[label=(\alph*)]
        \item 
        Give the explicit description of the class of morphisms with the right lifting property of $ \mathcal{ F } , l ( \mathcal{ F } )$, which is defined dually, i.e.
        \[
            r ( \mathcal{ F } ) \coloneqq ( l ( \mathcal{ F^{\op} } ) )^{\op}
        \]
        where $\mathcal{ F }^{\op} $ denotes the same class of morphisms but viewed in the opposite category $ \mathcal{ C }^{\op}$.
    
        \item 
        Show that $ l ( r ( l ( \mathcal{ F } ) ) ) = l ( \mathcal{ F } ) $.
    
    \end{enumerate}
    
    From now on let $ \mathcal{ C } = \Set $ and consider the inclusion $ \iota : \emptyset \to \{ \star \} $ of the empty set into a set with one element.
    
    \begin{enumerate}[label=(\alph*), resume]
        \item 
        Compute the set of morphisms with the right lifting property for $ \iota, r ( \{ \iota \} ), $ in Set.
    
        \item 
        Show that $ l ( r ( \{ \iota \} ) ) $ is the class of injective maps.
    
        \item 
        Show that (d) is equivalent to the axiom of choice.
        \[
            \forall  X ( \emptyset \notin X \implies \exists \varphi \colon X  \to \bigcup_{ A \in X } A 
            \forall A \in C \varphi( A ) \in A )    
        \]
    \end{enumerate}
\end{Exercise}

\begin{Exercise}
    Let $ \mathcal{ F } $ be a class of morphisms in a cocomplette category $ \mathcal{ C } $.
    We say that an object $ x \in \mathcal{ C } $ is a retract of an object $ y \in \mathcal{ C } $ if there exist two morphisms $ j : x \to y $ and $ q: y \to x $ such that $ q \circ j = \id_x .$
    
    \begin{enumerate}[label=(\alph*)]
        \item 
        Describe retracts in the category of morphisms $ \Fun ( [ 1 ] , \mathcal{ C } ) $ where we assume $ \mathcal{ C } $ to be small.
    
        \item 
        Show that $ l ( \mathcal{ F } ) $ is closed under retracts in the sense (a).
    
        \item 
        Show that $ l ( \mathcal{ F } ) $ is closed under pushouts, i.e. for any pushout square
        \[
        \begin{tikzcd}
            x 
            \ar[r, "g"]
            \ar[d, "f"]
            &
            x'
            \ar[d,"f'"]
            \\
            y
            \ar[r, "g'"]
            &
            y'
        \end{tikzcd}
        \]
        we have that if $ f \in l ( \mathcal{ F } ) $, then $ f' \in l ( \mathcal{ F } ) $.
    
        \item 
        Show that $ l ( \mathcal{ F } )$ is closed under composition.
    
        \item 
        Show that $ l ( \mathcal{ F } ) $ is closed under ( countable ) transfinite composition, i.e. for any functor $ F : \mathbb{ N }_0 \to \mathcal{ C } $ with $ f_n \coloneqq F ( n < n + 1 ) \in l ( \mathcal{ F } ) $ we have that the canonical map $ F ( 0 ) \to \colim_{ \mathbb{ N } } F $ is in $ l ( \mathcal{ F } ) $.
    \end{enumerate}
    
    Note that (e) can be generalised to arbitrary well-ordered sets (ordinals) in place of $ \mathbb{ N } $ using transfinite induction.
\end{Exercise}

\begin{Exercise}
    Let $ \mathcal{ F } $ be a class of morphisms on a cocomplete category $ \mathcal{ C } $.
    Assume that $ \mathcal{ F } $ includes all identity morphisms, is closed under pushouts and (countable) transfinite composition in the sense of Exercise 9.2.
    
    \begin{enumerate}[label=(\alph*)]
        \item 
        Show that $ \mathcal{ F } $ contains all isomorphisms.
    
        \item 
        Show that $ \mathcal{ F } $ is closed under composition.
    
        \item 
        Deduce that $ \mathcal{ F } $ is closed under finite coproducts, i.e. for two morphisms $ f : x  \to y $ and $ f' : x' \to y' $ with both $ f , f' \in \mathcal{ F } $, we have that the induced map $ f \amalg f' : x \amalg x' \to y \amalg y' $ is also in $ l ( \mathcal{ F } ) $.
    
        \item 
        Show that $ \mathcal{ F } $ is closed under countable coproducts by expressing a morphism $ \coprod_{ n \in \mathbb{ N } } f_n $ as suitable transfinite composition.
        
    \end{enumerate}
    
    Again, assuming that $ \mathcal{ F } $ is closed under arbitrary transfinite compositions, we can generalise the above argument to show that $ \mathcal{ F } $ is closed under set indexed coproducts.
\end{Exercise}

\begin{Exercise}
    Let $ A $ be a small category. 
    Recall that a morphism $ f : y \to z $ in $ A $ is a monomorphism, if for any two $ g , g' : x \to y $, we have that $ f \circ g = f \circ g' $ if and only if $ g = g' $.
    
    \begin{enumerate}[label=(\alph*)]
        \item 
        Show that a morphism in $ \Set $ is a monomorphism if and only if it is injective.
    
        \item 
        Deduce from Exercise 3.1 that a morphism of presheaves $ f : X \to Y $ in $ \widehat{ A } $ is a monomorphism if and only if $ f_a $ is a monomorphism for all $ a \in A $.
    
        \item 
        Conclude that the class of monomorphism in $ \widehat{ A } $ is closed under retracts, pushouts, (countable) transfinite composition and coproducts. In particular, the class of monomoprhism in $ \widehat{ A } $ is saturated.
    \end{enumerate}
\end{Exercise}