\section{Kan Fibrations}

Aim: Formalize the idea of a "family of Kan complexes parametrized by a simplicial set", i.e. suitable morphisms $X \xrightarrow{p} Y$ in $\Set_{\Delta}$ whose fibres are Kan complexes.
That is we have a pullback diagram
\[
\begin{tikzcd}
    X_y
    \arrow[d]
    \ar[r]
    &
    X
    \ar[d,"p"]
    \\
    \Delta^0
    \ar[r,"y"]
    &
    Y    
\end{tikzcd}
\]

\underline{Terminology}:
Let $ L \xrightarrow{i} K$ and $ X \xrightarrow{p} Y$ be morphisms.
We say that ($ i \lift{} p$) $i$ has the left lifting property with respect to $p$ (equivalently $p$ has the right lifting property with respect to $i$).
If there exists for every $L \xrightarrow{f} X$ and $K \xrightarrow{g}Y$ a morphism $h:K \to X$ such that the following diagram commutes
\[
\begin{tikzcd}
    L
    \arrow[d,"i"]
    \ar[r, "f"]
    &
    X
    \ar[d,"p"]
    \\
    K
    \ar[ru,"h"]
    \ar[r,"g"]
    &
    Y    
\end{tikzcd}
\]
Let $X \in \SetD$ be a Kan complex. 
Then 
\[
\{ \Lambda_k^n \to \Delta^n \mid n\geq 1 ,0 \leq k \leq n \} \lift{} (X \to \Delta^0)
\]
is given by 
\[
\begin{tikzcd}
    \Lambda_k^n
    \arrow[d,"\iota"']
    \ar[r, "\sigma"]
    &
    X
    \ar[d,"!"]
    \\
    \Delta^n
    \ar[ru,"\Tilde{\sigma}"]
    \ar[r,"!"']
    &
    \Delta^0
\end{tikzcd}
\]

\begin{defi}
    A morphism $X \xrightarrow{p} Y$ is a Kan fibration if 
    \[
    \{ \Lambda_k^n \to \Delta^n \mid n\geq 1 ,0 \leq k \leq n \} \lift{} (X \xrightarrow{p} Y)
    \]
\end{defi}

\begin{rmk}
    We may as well write $X \xrightarrow{p} Y \in \{ \Lambda_k^n \to \Delta^n \mid n\geq 1 , 0 \leq k \leq n\}^{\lift{}}$.
\end{rmk}

\begin{lem}
    Let $i \lift{} p$ and
    \[
    \begin{tikzcd}
        X'
        \arrow[d,"p'"]
        \ar[r]
        &
        X
        \ar[d,"p"]
        \\
        Y'
        \ar[r,"v"]
        &
        Y    
    \end{tikzcd}
    \]
    be a pullbacksquare, then $i \lift{} p'$.
\end{lem}

\begin{proof}
    Consider the diagram
    \[
    \begin{tikzcd}
        L
        \ar[r,"f"]
        \ar[d,"i"']
        &
        X'
        \arrow[d,"p'" pos=0.65]
        \ar[r, "u"]
        &
        X
        \ar[d,"p"]
        \\
        K
        \arrow[r,"g"']
        \ar[rru, dashed, "r" pos= 0.4]
        \ar[ru, dashed, "h"]
        &
        Y'
        \ar[r,"v"']
        &
        Y    
    \end{tikzcd}
    \]
    where $r$ exists since $i$ has the left lifting property with respect to $p$ and $h$ exists by the universal property of the pullback.
    Now $u \circ h \circ i = r \circ i = u \circ f$ and $p' \circ h \circ i = q \circ i = p' \circ f$, since the  morphism given by the pullback is unique we get that $h \circ i = f$.
\end{proof}

Lecture 3.12 

Formalise the notion of a locally constant family of Kan complexes relative to a base simplicial set.

\begin{defi}
    Let $X \xrightarrow{p} Y$ be a Kan fibration. 
    By the preceding theorem, we get that for every $y\in Y_0$ the fibre $X_y$ is a Kan complex. 
    Now consider $\Delta^0$ as the simplicial set where $\Delta^0_n=\Hom_{\SetD}([n],[0])=\{[n] \to [0]\}$ is the unique map into the final object.
    We have the constant map 
\todo{alot missin here}
\end{defi}

\begin{prop}
\label{comp_Kan_fib}
    Let $L \xrightarrow{i} K$ be a morphism and suppose that $X \xrightarrow{p} Y$, $Y \xrightarrow{g} Z$ have the right lifting property with respect to $i$, then the composition $g \circ p$ has the right lifting property with respect to $i$.
\end{prop}

\begin{proof}
    This is just a simple matter of writing down the square for $g$ and using the obtained morphism to write down the square $p$ obtaining a morphism that fllfills the desired property.
\end{proof}

\begin{cor}
    Suppose that $X \xrightarrow{p} Y$ is a Kan fibration and $Y$ is a Kan complex, then $X$ is a Kan complex.
\end{cor}

\begin{proof}
    Since $X \xrightarrow{p} Y$ is a Kan fibration and $Y \to \Delta^0$ is a Kan fibration, we get by \cref{comp_Kan_fib} that $X \to \Delta^0$ is a Kan fibration, thus we get that $X$ is a Kan complex.
\end{proof}

\begin{prop}
    Let $X^{(i)} \xrightarrow{p^i} Y^{(i)}$ with $i \in I$ be a set indexed family of Kan fibrations.
    Then $\prod X^{(i)} \xrightarrow{\prod p^i} \prod Y^{(i)}$ is a Kan fibration.
\end{prop}

\begin{proof}
Consider the diagram arising from the assumptions
\[
    \begin{tikzcd}
        \Lambda_k^n
        \ar[r,"\sigma"]
        \ar[d,"\iota"]
        &
        \prod X^{(i)}
        \ar[d, "\prod p^i" pos=0.7]
        \ar[r, "\prod \pi_j"]
        &
        X^{(j)}
        \ar[d, "p^j"]
        \\
        \Delta^n
        \ar[ru, dashed, "\exists ?"]
        \ar[rru, dashed, "h_j" pos=0.4]
        \ar[r, "\tau"']
        &
        \prod Y^{(i)}
        \ar[r, "\pi_j"']
        &
        Y^{(j)}
    \end{tikzcd}
\]
Since $p_j$ is a Kan fibration, there exists $h_j \colon \Delta^n \to X^{(j)}$ such that 
$\begin{cases}
    p_jh_j=\pi_j\tau\\
    h_j\iota = \pi_j \sigma_j.
\end{cases}$
    The universal property of the product gives $h \colon \Delta^n \to \prod X^{(i)}$ such that $\pi_j \circ h = h_j$.
    Now we have that $\pi_j(h \circ \iota ) = h_j \circ \iota = \pi_i \circ \sigma$ and thus $h \circ \iota = \sigma$ as well as $\pi_j(\prod {p_i} \circ h) = p_j \circ \pi_j \circ h = p_j h_j = \pi_j \tau$ and thus $\prod p_i \circ h = \tau$, where the equalities follow from the uniqueness of the morphism given by the pullback.
\end{proof}

\begin{prop}
    Let $\dotsc \to X^{(n)} \xrightarrow{p_n} \dotsc \to X^{(1)} \xrightarrow{p_1}X^{(0)}$ be a "tower" of Kan fibrations, that is for all $i, p_i$ is a Kan fibration.
    Then $X^{\infty} \coloneqq \lim X^{(n)} \xrightarrow{\pi_0} X^{(0)}$ is a Kan fibration. 
\end{prop}

\begin{proof}
    Consider the diagram 
\[
\begin{tikzcd}
    \Lambda_k^n
    \arrow[d,"\iota"']
    \ar[r, "\sigma"]
    &
    X^{(\infty)}
    \ar[d,"\pi_0"]
    \\
    \Delta^n
    \ar[ru, dashed, "\exists ?"]
    \ar[r,"\tau"']
    &
    X^{(0)}
\end{tikzcd}
\]
The idea is to construct a cone of the tower with apex $\Delta^n$ and then use the universal property $X^\infty$.
\todo{finish this ya lazy bombaclat}
\end{proof}