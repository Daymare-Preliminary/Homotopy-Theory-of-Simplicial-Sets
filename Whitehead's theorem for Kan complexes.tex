\section{Whitehead's theorem for Kan complexes}

For references for this section see \cite[Section I.7]{GoerSimp1999} and \cite[Section 3.2.7]{kerodon}.

\begin{lem}
\label{homotopy inverses are pointed homotopy inverses}
    Suppose that $f\colon (X,x) \to (Y,y)$ is a morphism for pointed Kan complexes, such that $f\colon X \to Y$ admits a left inverse up to homotopy (inverse in $\hKan$).
    Then $f$ admits a pointed inverse up to homotopy (inverse in $\hKan_*$).
\end{lem}

\begin{proof}
    Let $g\colon Y \to X$ be a homotopy left inverse to $f$, so there exists a homotopy $h\colon \id_X \to g\circ f$.
    Let now $\alpha \colon \Delta^1 \times X \to X$ be a homtopy and extend the homotopy diagram by the morphism associated to the selected point:
    \[
    \begin{tikzcd}
        \Delta^{\{0\}} \times \Delta^0
        \ar[r, "x"]
        \ar[d]
        &
        \Delta^{\{0\}} \times X
        \ar[d]
        \ar[rd, bend left, "id_X"]
        \\
        e\colon\Delta^1 \times \Delta^0
        \ar[r, "\id \times x"]
        &
        \Delta^1 \times X
        \ar[r, "\alpha"]
        &
        X
        \\
        \Delta^{\{1\}} \times \Delta^0
        \ar[u]
        \ar[r, "\id \times x"]
        &
        \Delta^{\{1\}} \times X
        \ar[u]
        \ar[ru, bend right, "g \circ f"']
    \end{tikzcd}
    \]
    Where $e\colon x \to g(f(x))=g(y)$ in $X$.
    Take $\Delta^0 \xhookrightarrow{y} Y$ and the lifting diagram
    \[
    \begin{tikzcd}
        \Lambda^1_1= \Delta^{\{1\}}
        \ar[r, "g"]
        \ar[d, hook]
        &
        \underline{\Hom}(Y,X)
        \ar[d, "\ev_y"]
        \\
        \Delta^1
        \ar[ru, dashed]
        \ar[r, "e"]
        &
        \underline{\Hom}(\Delta^0,X) \cong X
    \end{tikzcd}
    \]
    The lifting morphism together with the standard adjunction gives a homotopy $\beta \colon \Delta^1 \times Y \to  X$ from $g'$ to $g$ where $g'(y)=x$.
    We can concetenate $\beta$ with $f$, to obtain $\beta_f \colon \Delta^1 \times X \xrightarrow{\id \times f} \Delta^1 \times Y \xrightarrow{ \beta}X$ which is a homotopy of the concetenation $\beta_f \colon g' \circ f \to g \circ f$.
    We can now consider the homotopy diagram of $\beta$ extended by the selected point in $Y$:
    \[
    \begin{tikzcd}
        \Delta^{\{0\}} \times \Delta^0
        \ar[r, "\id \times y"]
        \ar[d]
        &
        \Delta^{\{0\}} \times Y
        \ar[d]
        \ar[rd, bend left, "g'"]
        \\
        e\colon\Delta^1 \times \Delta^0
        \ar[r, "\id \times y"]
        &
        \Delta^1 \times Y
        \ar[r, "\beta"]
        &
        X
        \\
        \Delta^{\{1\}} \times \Delta^0
        \ar[u]
        \ar[r, "\id \times y"]
        &
        \Delta^{\{1\}} \times Y
        \ar[u]
        \ar[ru, bend right, "g"']
    \end{tikzcd}
    \]
    The next step is to take the degenerate 2-simplex $s_0(e)$ given by
    \[
    \begin{tikzcd}
        x
        \ar[rd, "e"]
        \\
        x 
        \ar[u, "1_X"]
        \ar[r, "1_X"']
        &
        g ( f ( x ) ) 
    \end{tikzcd}
    \]
    and take it as the bottom row map in the following lifting problem
    \[
    \begin{tikzcd}
        \Lambda_2^2
        \ar[d]
        \ar[r, "q"]
        &
        \underline{\Hom}  ( X , X )
        \ar[d, "\ev_x"]
        \\
        \Delta^2
        \ar[r,"s_0(e)"']
        &
        \underline{\Hom} ( \Delta^0 , X )
    \end{tikzcd}
    \]
    where $q$ is given by the following horn diagram.
    \[
    \begin{tikzcd}
        g'f
        \ar[rd, "\beta_f"]
        \\
        \id_X
        \ar[ u , dashed]
        \ar[r, " \alpha"']
        &
        gf
    \end{tikzcd}
    \]
    Since $\ev_x$ is a Kan fibration, we obtain a lift and thus a pointed homotopy $\gamma:\id_X \to g'f $, thus $g'$ is a pointed homotopy inverse to $f$.
\end{proof}

\begin{cor}
\label{htpy eq. then pointed htpy eq.}
    Let $f\colon (X,x) \to (Y,y)$ be a morphism of Kan complexes such that $f \colon X \to Y$ is a homotopy equivalence then $f$ is also a pointed homotopy equivalence.
\end{cor}

\begin{proof}
    By \cref{homotopy inverses are pointed homotopy inverses} $[f]$ admits a pointed left inverse 
    $g\colon (Y,y) \to (X,x)$ and $[g] \circ [f] = [\id_X]$ in $\hKan_*$.
    Then $g\colon Y \to X$ is a homotopy equivalence.
    There exists a homotopy left inverse $h\colon (X,x) \to (Y,y)$, $[h]=[h]([g][f])=[f]$
\end{proof}

\begin{cor}
    Let $f\colon X \to Y$ be a homotopy equivalence. 
    Then for all $x \in X$ and all $n \geq 0$, there is an isomorphism $\pi_n(f)\colon \pi_n(X,x) \xrightarrow{\sim} \pi_n(Y,f(x))$.
\end{cor}

\begin{proof}
    By \cref{htpy eq. then pointed htpy eq.} $f\colon (X,x) \to (Y,f(x))$ is a pointed homotopy equivalence.
    Now $\pi_n(X,x) = \pi_n ( \underline{\Hom}((S^n,*),(X,x))$, similarly for $(Y,f(x))$
\end{proof}

\begin{defi/prop}  
\label{contractible Kan complex}
    Let $X$ be a Kan complex, then the following are equivalent:
    \begin{enumerate}
        \item  
        $X \to \Delta^0$ is a homotopy equivalence,
        \item 
        for all $x \in X$ and all $n \geq 0$, $\pi_n(X,x) = \{ *\}$,
        \item 
        $X \to \Delta^0$ is a trivial Kan fibration.
    \end{enumerate}
    In this case we say that $X$ is \underline{contractible}.
\end{defi/prop}

\begin{proof}
    Exercise
\end{proof}

\begin{prop}
    Let $p\colon X \to Y$ be a Kan fibration between Kan complexes, then the following are equivalent:
    \begin{enumerate}
        \item 
        $p$ is a trivial Kan fibration,
        \item 
        $p$ is a homotopy equivalence,
        \item 
        for all $x \in X$ and for all $\pi_n(f)\colon \pi_n(X,x) \to \pi_n(Y,f(x))$ is bijective,
        \item 
        for all $y \in Y$ the fibre $X_y$ given by the pullback 
        \[
        \begin{tikzcd}
            X_y
            \ar[r]
            \ar[d]
            &
            X
            \ar[d,"p"]
            \\
            \Delta^0
            \ar[r,"y"']
            &
            Y
        \end{tikzcd}
        \]
        is contractible.
    \end{enumerate}
\end{prop}

\begin{proof}
\leavevmode
\begin{itemize}[label={}]
    \item 
    $1. \implies 2.$ Exercise
    \item 
    $2. \implies 3.$ homtopy inverse
    \item 
    $3. \implies 4.$ Serre long exact sequence
    \item 
    $4. \implies 1.$ Suppose that for all $y \in Y$ we have that $X_y \to \Delta^0$ is a trivial Kan fibration, which is by \cref{contractible Kan complex} equivalent to contractibility of $X_y$.
    Take a boundary inclusion 
    \[
    \begin{tikzcd}
        \partial \Delta^n
        \ar[r, "\alpha"]
        \ar[d, hook]
        &
        X
        \ar[d, "p"]
        \\
        \Delta^n 
        \ar[r, "\beta"]
        &
        Y
    \end{tikzcd}
    \]
    and consider a homotopy $H$ from the constant map to the identity on $\Delta^n$, that is $H \colon \Delta^1 \times \Delta^n \to \Delta^n$, where $H\colon c(0) \to \id_{\Delta^n}$.
    Putting together the diagram for the homotopy $H$ and the $n$-simplex $\beta$ we obtain:
    \[
    \begin{tikzcd}
        \Delta^{\{0\}} \times \partial \Delta^n
        \ar[r, "i"]
        \ar[d]
        &
        \Delta^{\{0\}} \times \Delta^n
        \ar[d]
        \ar[rd, bend left, "c(0)"]
        \ar[rrd,bend left, "c(y)"]
        \\
        e\colon\Delta^1 \times \partial \Delta^n
        \ar[r, "\id \times i"]
        &
        \Delta^1 \times \Delta^n
        \ar[r, "H"]
        &
        \Delta^n
        \ar[r, "\beta"]
        &
        Y
        \\
        \Delta^{\{1\}} \times \partial \Delta^n
        \ar[u]
        \ar[r, "i"]
        &
        \Delta^{\{1\}} \times \Delta^n
        \ar[u]
        \ar[ru, bend right, "\id_{\Delta^n}"'] 
        \ar[rru, bend right, "\beta"']
    \end{tikzcd}
    \]
    \[
    \begin{tikzcd}
    	\overline{\alpha}
    	\ar[r, "\Tilde{h}"]
    	\ar[d, mapsto,shift left=0.8cm, "p_*"]
    	&
    	\alpha
    	\\
    	c(y)
    	\ar[r, "h"']
    	&
    	p_{\alpha}
    \end{tikzcd}
	\quad
	\begin{tikzcd}
		\Delta^{\{1\}}
		\ar[r, "\alpha"]
		\ar[d]
		&
		\underline{\Hom}(\partial \Delta^n , X )
			\ar[d, "p_*"]
		\\
		\Delta^1
		\ar[ru, dashed , "\Tilde{h}"]
		\ar[r, "h"]
		&
		\underline{\Hom}(\partial \Delta^n , Y )
	\end{tikzcd}
	\]
	where $ \Tilde{h} \colon \Delta^1 \times \partial \Delta^n \to X , h \colon \overline{ \alpha } \to \alpha , \overline{\alpha} \colon \Delta^n \to X $ and $ p \circ \overline{\alpha} = c( y)$.
	\[
	\begin{tikzcd}
		\partial \Delta^n
		\ar[r, dashed, "\exists"]
		\ar[rr, bend left, "\overline{\alpha}"]
		\ar[d, "i"]
		&
		X_y
		\ar[d]
		\ar[r]
		\ar[d, shift left=0.7cm, phantom ,"\text{PB}"]
		&
		X
		\ar[d, "p"]
		\\
		\Delta^n
		\ar[r, "!"]
		\ar[rr, bend right, " c ( y )"']
		\ar[rru, "v"]
		\ar[ru, dashed]
		&
		\Delta^0
		\ar[r, "y"]
		&
		Y
	\end{tikzcd}
	\]
	\[
	\begin{tikzcd}
		\partial \Delta^n
		\ar[d, "i"]
		\ar[r, "\alpha"]
		&
		X
		\ar[d, "p"]
		\\
		\Delta^n
		\ar[r, "\beta"]
		&
		Y
	\end{tikzcd}
	\quad
		\begin{tikzcd}
			\Delta^{\{0\}} \times \partial \Delta^n
			\ar[rr, bend left, "\alpha"]
			\ar[r]
			\ar[d,hook]
			&
			( \Delta^{\{0\}} \times \Delta^n ) \cup ( \Delta^1 \times \partial \Delta^n )
			\ar[r, " { ( v , \Tilde{h} ) } "]
			\ar[d, hook, "\in \An"]
			&
			X
			\ar[d, "p"]
			\\
			\Delta^{\{0\}} \times \Delta^n
			\ar[r]
			\ar[rr, bend right, "\beta"]
			&
			\Delta^1 \times \Delta^n
			\ar[ru]
			\ar[r, " \beta \circ H"]
			&
			Y
		\end{tikzcd}
	\]
\end{itemize}
\end{proof}

\begin{thm}{Whitehead's theorem}
Let $f\colon X \to Y$ be a morphsim between Kan complexes. 
Then the following are equivalent:
    \begin{enumerate}
        \item 
        $f$ is a homotopy equivalence,
        \item 
        for all $x \in X$ and all $n \geq 0$, $\pi_n(f)\colon  \pi_n(X,x) \to \pi_n(Y,y)$ is a bijection.
    \end{enumerate}
\end{thm}

\begin{proof}
\leavevmode
    \begin{itemize}[label={}]
        \item 
        $1. \implies 2.$ This is known.
        \item 
        $2. \implies 1.$
        By the use of Quillen's small object argument \cref{Quillen's small object argument}, we obtain a factorisation of $f$
        \[
        \begin{tikzcd}
            &
            \Tilde{X}
            \ar[rd, "p"]
            &
            \\
            X
            \ar[ru, "i"]
            \ar[rr, "f"]
            &&
            Y
        \end{tikzcd}
        \]
        where $i \in \An$ and $p \in \KanFib$. 
        Since $i\colon X \to \Tilde{X}$ is anodyne, $i$ is a weak equivalence and by \cref{Homtopy eq. of Kan complexes are bij in pi_0(weak eq.)}, using that $X$ and $Y$ are Kan complexes, $i$ is also a homotopy equivalence and thus satisfies 2.
        By \cref{2 out of 3 weak equivalences} $p$ also satisfies the property 2. and by the previous proposition $p$ is a homotopy equivalence.        
    \end{itemize}    
\end{proof}
