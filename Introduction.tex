Mathematics, while deeply technical and logical is and will always be a love letter. 
A language that estranges most and can be deciphered by few, to write encrypted letters back and forth with the cosmos.
I hope that she wants to dance with me one day again, even though I have two left feet.
I sometimes think about category theory as the schematics for human dancing, the diagram chase on a commutative diagram becoming the instructions for steps. 
An infinite repetition of the same steps with slide change to give birth to a beautiful infinite dance. 
A spiraling helicial movement of lovers moving around one another, like planets orbiting one another.
Simple rules birthing the dance of reality. 
Similarly the upmost simple rules for some signs birth a story, but one needs to bring alot to the table, long breath and more faith than I had so far to dance this dance, to make it infinite dance. 
Rigid repetition until one can dance all the standards by heart, then one is as well permitted to move freely within the structure, take the lead, discover new moves, teach the art.
The prolific mathematicians have grasped alot of intuition for this process by playing piano I feel, it brings the benefit of immediate response, a harmony is for the human ear easy to discern from dissonance, hit the right keys to unravel the tones of the masters of old and anything beyond. 
These notes are based on a lecture by Gustavo Jasso held at the university of cologne in the winter semester 24/25.
