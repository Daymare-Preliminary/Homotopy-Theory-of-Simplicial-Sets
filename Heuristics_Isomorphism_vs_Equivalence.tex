\section{Heuristics: Isomorphism vs. Equivalence}

Makkai's Principle of Isomorphism (1998) says:

"All grammatically correct properties about objects in a fixed category are to be invariant under isomorphism."

Fix a category $ \mathcal{ C } $ and a small category $ A $ and take the functor category $ \Fun ( A , \mathcal{ C } )$, that is $ A $-shaped diagrams in $ \mathcal{ C } $.
For $ X \in \mathcal{ C } $ define:
\[
    \lim_{a \in A} \Hom_{\mathcal{ C }} ( X , D(a)) \coloneqq \{ ( p_a \colon X \to  D ( a ) )_{ a \in A } \mid 
    \begin{tikzcd}
    X
    \ar[d, "f_a"']
    \ar[rd, "f_b"]
    \\
    D ( a ) 
    \ar[r]
    &
    D ( b ) 
    \end{tikzcd}
    \forall f \colon a \to b \text{ in } A
    \}
\]

For $ \phi \colon Y \to X $ define the function 
\begin{align*}
    \phi^* \colon \lim_{a \in A} \Hom_{ \mathcal{ C }} ( X , D ( a ) ) 
    &\to
    \lim_{ a \in A } \Hom_{ \mathcal{ C }} ( Y , D ( a ) ) 
    \\
    p = ( p_a \colon X \to D ( a ) )_{ a \in A }
    &\mapsto
    \phi^* ( p ) = ( p_a \circ \phi \colon Y \to D ( a ) )_{ a \in A }
\end{align*}

Thus $ \lim_{a \in A } \Hom_{ \mathcal{ C } } ( - , D ( a ) ) \colon \mathcal{ C }^{\op} \to \Set $ is a presheaf of sets on $ \mathcal{ C } $.

\begin{defi}
    A limit of a diagram $ D \colon A \to \mathcal{ C } $ is a cone $ p \in \lim_{a \in A } \Hom_{ \mathcal{ C } } ( X, D ( a ) ) $ that is universal in the sense that $ \forall Y \in \mathcal{ C }, \forall q \in \lim_{a \in A} \Hom_{\mathcal{ C }} ( Y , D ( a ) ) \exists! \varphi\colon Y \to X $ such that $ \varphi^* ( p ) = q $.
    We write $ \lim_{ a \in A } D ( a ) $ for any limit of $ D $ (which may or may not exist).
\end{defi}

\begin{rmd}[Yoneda Lemma]
    Let $ X \in \mathcal{ C } $ and let 
    \begin{align*}
        \nu \colon \Nat ( \Hom_{\mathcal{ C } } ( - , X ) , \lim_{ a \in A } \Hom_{\mathcal{ C } } ( - , D ( a ) ) ) 
        &\isomorphism
        \lim_{ a \in A } \Hom_{ \mathcal{ C } }( X , D ( a ) ) 
        \\
        \eta = ( \eta_Y \colon \Hom_{ \mathcal{ C } } (  Y , X ) \to \lim_{ a \in \mathcal{ C } } ( Y , D ( a ) )_{ Y \in \mathcal{ C } } 
        &\mapsto 
        \eta_X ( \id_X )
        \\
        ( \eta^p_Y ( p ) \coloneqq \varphi^* ( p ) )_{ Y \in \mathcal{ C }}
        &\mapsfrom
        p
    \end{align*}
\end{rmd}

If $ p \in \lim_{a \in A} \Hom_{ \mathcal{ C } } ( X , D ( a ) ) $ is a limit of $ D \colon A \to \mathcal{ C } $ then 

$ \eta^p \colon \Hom_{ \mathcal{ C } } ( - ,  X ) \isomorphism \Hom_{ \mathcal{ C } } ( - , D ( a ) ) $ is a natural isomorphism. 
We furthermore obtain, that for an isomorphism $\psi \colon Y \to X $ in $\mathcal{ C } $ we have 
\[
\begin{tikzcd}
    & 
    \Hom_{\mathcal{ C } } ( - , X ) 
    \ar[rd, " \eta^p ", "\sim"']
    \\
    \Hom_{ \mathcal{ C } } ( - , Y ) 
    \ar[ru , " \psi^*" , " \sim"' ]
    \ar[rr, " \sim " ]
    &&
    \lim_{ a \in A } \Hom_{ \mathcal{ C } } ( - , D ( a ) ) 
\end{tikzcd}
\]
\begin{exmp}
    Let the following be a diagram in $ \mathcal{ C } $ 
    \[
    \begin{tikzcd}
        &
        Y
        \ar[d , "g" ]
        \\
        X
        \ar[r, "f" ]
        &
        Z
    \end{tikzcd}
    \]
    then the limit of the diagram (if it exists ) is called a pullback.
    \[
    \begin{tikzcd}
        W
        \ar[rrd, bend left, "\forall f'' " ]
        \ar[rd, " \exists! p "]
        \ar[rdd, bend right, " \forall g'' "' ]
        \\
        &
        X \times_Z Y 
        \ar[ r , " f'" ]
        \ar[d, " g' "]
        &
        Y
        \ar[d, " g "]
        \\
        &
        X
        \ar[r, "f"]
        &
        Z
    \end{tikzcd}
    \]
    For example if $ \mathcal{ C } = \Set$ then $ X  \times_Z Y = \{ ( X , Y ) \in X \times Y \mid f ( x ) = g ( y ) \}$.
\end{exmp}

The commutativity condition takes place in $ \Hom_{ \mathcal{ C } } ( X \times_Z Y , Z ) \ni g \circ f' = f \circ g' $

\underline{ Makkai's Pronciple of Equivalence }
All grammatically correct properties of objects in a fixed 2-category are to be invariant under equivalence.

\begin{rmk}
    We want to $ \Cat $ be the strict 2-category of (small) categories with functors as 1-morphisms and natural transformations as 2-morphisms.
    Now natural transformation allow for a notion of equivalence of morphisms, that is in a 1-category we only knew what it meant for two morphisms to be equal, but now we can talk about two functors being naturally isomorphic given us a notion of equivalence of 1-morphisms, via the 2-morphisms.
\end{rmk}

\begin{defi/prop}[Godement Product]
    Consider natural transformations 
    \[
    \begin{tikzcd}[column sep=huge]	
    \mathcal{C}
    	\arrow[bend left=50]{r}[name=U,label=above:$F_1$]{}
    	\arrow[bend right=50]{r}[name=D,label=below:$F_2$]{}
    &
    \mathcal{D}
    \arrow[shorten <=10pt,shorten >=10pt,Rightarrow,to path={(U) -- node[label=right:$\alpha$] {} (D)}]{}
    	\arrow[bend left=50]{r}[name=U',label=above:$G_1$]{}
    	\arrow[bend right=50]{r}[name=D',label=below:$G_2$]{}
	&
	\mathcal{E} 
    \arrow[shorten <=10pt,shorten >=10pt,Rightarrow,to path={(U') -- node[label=right:$\beta$] {} (D')}]{}    	
    \end{tikzcd}
    \]
    
    Their Godement product is the natural transformation.
    \[
    \begin{tikzcd}[column sep=huge]	
    \mathcal{ C }
    	\arrow[bend left=50]{rr}[name=U,label=above:$G_1 \circ F_1$]{}
    	\arrow[bend right=50]{rr}[name=D,label=below:$G_2 \circ F_2$]{}
    &&
    \mathcal{E}
    \arrow[shorten <=10pt,shorten >=10pt,Rightarrow,to path={(U) -- node[label=right:$\beta * \alpha$] {} (D)}]{}
    \end{tikzcd}	
    \]
    Let $ X \in \mathcal{ C } $, we obtain the following diagram
    \[
    \begin{tikzcd}
        F_1 ( X ) 
        \ar[d, " \alpha_x "]
        &
        G_1 ( F_1 ( X ) ) 
        \ar[d, " G_1 ( \alpha_X ) "']
        \ar[rd, " {( \beta * \alpha )_X} "]
        \ar[r , " \beta_{ F_1 ( X ) } "]
        & 
        G_2 ( F_1 ( X ) )
        \ar[d, " G_2 ( \alpha_X ) "]
        \\
        F_2 ( X ) 
        &
        G_1 ( F_2 ( X ) ) 
        \ar[ r, " \beta_{ F_2 ( X ) }"]
        &
        G_2 ( F_2 ( X ) ) 
    \end{tikzcd}    
    \]
    in $ \mathcal{ D } $.
\end{defi/prop}

\begin{proof}
    We show that $ \beta * \alpha \colon G_1 \circ F_1 \Rightarrow g_2 \circ F_2 $ is indeed a natural transformation. 
    For that we take the following diagram 
     \[
     \begin{tikzcd}
         X
         \ar[d, "f"]
         &
         G_1 ( F_1 ( X ) ) 
         \ar[r , " G_1 ( \alpha_X ) "]
         \ar[r, " G_1 ( \alpha_X ) "]
         \ar[d, " G_1 ( F_1 ( f ) ) "']
         &
         G_1 ( F_2 ( X ) ) 
         \ar[ r, " \beta_{ F_2 ( X ) } "]
         \ar[d, " G_1 ( F_2 ( f ) ) "]
         & 
         G_2 ( F_2 ( X ) )
         \ar[d, " G_2 ( F_2 ( f ) ) "]
         \\
         Y
         &
         G_1 ( F_1 ( Y ) )
         \ar[ r , " G_1 ( \alpha_1 ) "']
         &
         G_1 ( F_2 ( Y ) ) 
         \ar[ r, " {\beta_{ F_2 ( Y ) }}"']
         & 
         G_2 ( F_2 ( Y ) ) 
     \end{tikzcd}
     \]
\end{proof}

\begin{prop}
    Consider natural transformations 
    \[
    \begin{tikzcd}[column sep=huge]	
    	\mathcal{C}
    	\arrow[bend left=90]{r}[name=U,label=above:$F_1$]{}
    	\arrow[bend right=90]{r}[name=D,label=below:$F_3$]{}
    	\arrow{r}[name=M]{}
    	\arrow{r}[label=below:$F_2$, pos=0.2]{}
    	&
    	\mathcal{D}
    	\arrow[shorten <=5pt,shorten >=5pt,Rightarrow,to path={(U) -- node[label=right:$\alpha$] {} (M)}]{}
    	\arrow[shorten <=5pt,shorten >=5pt,Rightarrow,to path={(M) -- node[label=right:$\gamma$] {} (D)}]{}
    	\arrow[bend left=90]{r}[name=U',label=above:$G_1$]{}
    	\arrow[bend right=90]{r}[name=D',label=below:$G_3$]{}
    	\arrow{r}[name=M']{}
    	\arrow{r}[label=below:$G_2$, pos=0.2]{}
    	&
    	\mathcal{E} 
    	\arrow[shorten <=5pt,shorten >=5pt,Rightarrow,to path={(U') -- node[label=right:$\beta$] {} (M')}]{}
    	\arrow[shorten <=5pt,shorten >=5pt,Rightarrow,to path={(M') -- node[label=right:$\delta$] {} (D')}]{}    	
    \end{tikzcd}	
	\]
    Then $ ( \delta \beta ) * ( \gamma \alpha ) = ( \delta * \gamma ) \circ ( \beta * \alpha )$.
\end{prop}

\begin{proof}
    Let $ X \in \mathcal{ C } $ 
    \[
    \begin{tikzcd}
        G_1 ( F_1 ( X ) ) 
        \ar[r, " \beta_ {F_1 ( X ) }"]
        \ar[rd, " ( \beta * \alpha )_X "]
        \ar[d, " G_1 (\alpha_X ) "']
        \ar[dd, bend right=90, " G_1 ( \gamma_X \alpha_X ) "']
        &
        G_2 ( F_1 ( X ) ) 
        \ar[d, " G_2 ( \alpha_X ) "]
        \ar[r, " \delta_{ F_1 ( X ) }"]
        &
        G_3 ( F_1 ( X ) ) 
        \ar[dd, bend left =90, " G_3 ( \gamma_X \alpha_X ) "]
        \ar[ d , " G_3 ( \alpha_X ) "]
        \\
        G_1 ( F_2 ( X ) ) 
        \ar[d, " G_1 ( \gamma_X )"']
        \ar[r, " \beta_{ F_2 ( X ) }"]
        &
        G_2 ( F_2 ( X ) ) 
        \ar[r, " \delta_{ F_2 ( X ) }"]
        \ar[d, " G_2 ( \gamma_X ) "]
        \ar[rd, " ( \delta * \gamma )_X "]
        &
        G_3 ( F_2 ( X ) ) 
        \ar[d, " G_3 ( \gamma_X ) "]
        \\
        G_1 ( F_3 ( X ) ) 
        \ar[r, " \beta_{ F_3 ( X ) }"]
        &
        G_2 ( F_3 ( X ) )
        \ar[r , " \delta_{F_2 ( X ) }"]
        &
        G_3 ( F_3 ( X ) ) 
    \end{tikzcd}
    \]
    Now the long diagonal of the diagram corresponds to $ ( \delta * \gamma ) \circ ( \beta * \alpha ) $ and the outer large square to $ ( \delta \circ \beta ) * ( \gamma \circ \alpha )$.
\end{proof}

\begin{defi}
    The Godement products bellow are called whickerings
    \[
    \begin{tikzcd}[column sep = huge]
    	\mathcal{C}
    		\ar[r, " F "]
    	&
    	\mathcal{D}
    		\arrow[bend left=50]{r}[name=U',label=above:$G_1$]{}
    		\arrow[bend right=50]{r}[name=D',label=below:$G_2$]{}
    	&
    	\mathcal{E} 
    		\arrow[shorten <=10pt,shorten >=10pt,Rightarrow,to path={(U') -- node[label=right:$\beta$] {} (D')}]{} 
    \end{tikzcd}
	\mapsto
	\begin{tikzcd}[column sep = huge]
		\mathcal{ C }
			\arrow[bend left=50]{rr}[name=U',label=above:$G_1 \circ F$]{}
			\arrow[bend right=50]{rr}[name=D',label=below:$G_2 \circ F $]{}
		&&
		\mathcal{ E }
			\arrow[shorten <=10pt,shorten >=10pt,Rightarrow,to path={(U') -- node[label=right:$\beta*\id_F$] {} (D')}]{} 
	\end{tikzcd}
	\]
	\[
	\begin{tikzcd}[column sep = huge]
		\mathcal{C}
		\arrow[bend left=50]{r}[name=U',label=above:$G_1$]{}
		\arrow[bend right=50]{r}[name=D',label=below:$G_2$]{}
		&
		\mathcal{D} 
		\arrow[shorten <=10pt,shorten >=10pt,Rightarrow,to path={(U') -- node[label=right:$\alpha$] {} (D')}]{} 
		\ar[r, " G"]
		&
		\mathcal{ E }
	\end{tikzcd}
	\mapsto
	\begin{tikzcd}[column sep = huge]
		\mathcal{ C }
		\arrow[bend left=50]{rr}[name=U',label=above:$G \circ F_1$]{}
		\arrow[bend right=50]{rr}[name=D',label=below:$G \circ F_2 $]{}
		&&
		\mathcal{ E }
		\arrow[shorten <=10pt,shorten >=10pt,Rightarrow,to path={(U') -- node[label=right:$\id_G*\alpha$] {} (D')}]{} 
	\end{tikzcd}
	\]
	
\end{defi}

\begin{construction}
    Given a cospan of groupoids 
    \[
    \begin{tikzcd}
        &
        \mathcal{B}
        \ar[d, " G "]
        \\
        \mathcal{ A }
        \ar[r, " F "]
        &
        \mathcal{ C }
    \end{tikzcd}
    \]
    its 2-pullback is the diagonal of groupoids.
    \[
    \begin{tikzcd}
        \mathcal{ A } \times_{ \mathcal{ C } } \mathcal{ B } 
        \ar[r, " \pi_{ \mathcal{ B }} "]
        \ar[d, " \pi_{ \mathcal{ A }} "]
        &
        \mathcal{ B }
        \ar[ d , " G " ]
        \\
        \mathcal{ A }
        \ar[r, " F " ]
        &
        \mathcal{ C }
    \end{tikzcd}
    \]
    The objects are given as $ \Ob ( \mathcal{ A } \times_\mathcal{ C } \mathcal{ B } ) = ( a \in \mathcal{ A } , b \in \mathcal{ B }, \varphi \colon F ( a ) \isomorphism G ( b ) $ in $ \mathcal{ C } )$ and morphisms are given by tuples of morphisms $ ( u , v ) \colon ( a , b, \varphi ) \to ( a' , b' , \varphi' ) $, where $ u \colon a \to a' $ and $ v \colon b \to b '$ are morphisms in the respective groupoids, such that the following square commutes
    \[
    \begin{tikzcd}
        F ( a ) 
        \ar[r, "\varphi" , " \sim "']
        \ar[d, "F( a )"']
        &
        G ( b )
        \ar[d, " G ( v )"]
        \\
        F ( a' )
        \ar[r, "\varphi", "\sim"']
        &
        G(b')
    \end{tikzcd}
    \]
\end{construction}

Lecture 15.04

Let $ X ,  Y , Z \in \Set $ and consider a pullback diagram 
\[
\begin{tikzcd}
	X \times_Z Y 
	\ar[r, " \pi_Y "]
	\ar[d, " \pi_X "]
	&
	Y
	\ar[d, " g \circ ?"]
	\\
	X
	\ar[r, " f"]
	&
	Z    
\end{tikzcd}
\]
where $ X \times_Z Y \coloneqq \{ ( x , y ) \in X \times Y \mid f ( x ) = g ( y ) \}$ and 
\begin{comment}
\[
\begin{align*}
	\Hom_{\Set} ( \prescript{\forall}{}{W} , X \times_Z  Y ) 
	&\isomorphism
	\Hom_{ \Set } ( W , X ) \times_{\Hom_{\Set} ( W , Z ) } \Hom_{ \Set } (  W , Y ) 
	\\
	(\varphi \colon W \to X \times_Z Y ) 
	&\mapsto
	\left(
	\begin{tikzcd}
		W 
		\ar[r, " \pi_Y \circ \varphi "]
		\ar[d , " \pi_X \circ \varphi "]
		&
		Y
		\ar[d, "g"]
		\\
		X
		\ar[r, "f"]
		&
		Z        
	\end{tikzcd} 
	\right)
\end{align*}
\]
\end{comment}

\begin{construction}
	Let $ \mathcal{A} , \mathcal{B}, \mathcal{ C } $: groupoids
	\[
	\begin{tikzcd}
		\mathcal{A} \times_{\mathcal{ C }}^{(2)} \mathcal{B}
		\ar[r, " {\pi_{\mathcal{ A }}}"]
		\ar[d, " {\pi_{\mathcal{ B }}}"]
		&
		\mathcal{B}
		\ar[d ,"G"]
		\\
		\mathcal{A}
		\ar[r , "F"]
		&
		\mathcal{C}
	\end{tikzcd}
	\]
\end{construction}

Lecture 15.04

Let $ X , Y , Z \in \Set $ and consider a pullback diagram
\[
\begin{tikzcd}
	X \times_Z Y 
	\ar[r, " \pi_Y " ]
	\ar[d, " \pi_X"]
	&
	Y
	\ar[d, " g "]
	\\
	X
	\ar[r, " f "]
	&
	Z
\end{tikzcd}
\] 
Where the fiber product is given by $ X \times_Z Y \coloneqq \{ ( x , y ) \in  X \times Y \times f ( x ) = g ( x ) \} $ such that the following isomorphism holds.

%\begin{\align*}
%	\Hom_{ \Set } ( \prescript{\forall}{}{W} , X \times_Z Y ) 
%	&\isomorphism 
%	\Hom_{ \Set } ( W , X ) \times_{ \Hom_{ \Set } ( W , Z ) \Hom_{ \Set } ( W , Y )
%	\\
%	( \varphi \colon X \times_Z  Y )
%	&\mapsto
%\end{align*}

\begin{tikzcd}
	W 
	\ar[r, " \pi_Y \circ \varphi"]
	\ar[d, " \pi_X \circ \varphi"]
	&
	Y
	\ar[d, "g" ]
	\\
	X
	\ar[r, " f " ]
	&
	Z
\end{tikzcd}

\todo{do this properly}
Thus the case for pullback of a diagram of objects is clear, but what does the pullback of morphism sets look like?

\begin{construction}
	Let $ \mathcal{A} , \mathcal{B} , \mathcal{C} $ be groupoids
	and 
	\[
	\begin{tikzcd}
		\mathcal{A} \times_{\mathcal{ C }}^{(2)} \mathcal{B}
		\ar[r, " {\pi_{\mathcal{ A }}}"]
		\ar[d, " {\pi_{\mathcal{ B }}}"]
		&
		\mathcal{B}
		\ar[d ,"G"]
		\\
		\mathcal{A}
		\ar[r , "F"]
		&
		\mathcal{C}
	\end{tikzcd}
	\] 
	be the 2-pullback of $ \mathcal{ A } \xrightarrow{ F } \mathcal{ C }$.
	Its objects are triples $ X = ( a \in \mathcal{ A } , b \in \mathcal{ B } , \varphi_X \colon F ( a ) \isomorphism G ( b ) ) $ and for another triple $ X' = ( a' \in \mathcal{ A } , b' \in \mathcal{ B } , \varphi{X'} \colon F ( a' ) \isomorphism G ( b' ) ) $ the morphisms are given by tuples $  ( \mathcal{ A } \ni u \colon a \to a' , \mathcal{ v } \ni v \colon b \to b' ) $ such that 
	\[
	\begin{tikzcd}
		F ( a )
		\ar[r, " \sim "]
		\ar[d, " F ( u )"]
		&
		G ( b ) 
		\ar[ d , " G ( v ) "]
		\\
		F ( a ' )
		\ar[ r, " \sim "]
		&
		G ( b ' ) 
	\end{tikzcd}
	\] 
	that is $ \varphi_{X'} \circ F ( u ) = G ( v ) \circ \varphi_X.$
	For a groupoid $ \mathcal{ D } $ we may consider the induced cospan of groupoids 
	\[
		\begin{tikzcd}
			\Fun ( \mathcal{ D } , \mathcal{ A } ) \times_{ \Fun ( \mathcal{ D } , \mathcal{ C } )  } \Fun ( \mathcal{ D } , \mathcal{B } )
			\ar[d]
			\ar[r]
			&
			\Fun( \mathcal{ D } , \mathcal{ B } )
			\ar[d, " G \circ ? "]
			\\
			\Fun ( \mathcal{ D } , \mathcal{ A } )
			\ar[r, " F  \circ ? "]
			&
			\Fun( \mathcal{ D } , \mathcal{ C }) 
		\end{tikzcd}
	\]
	\begin{Interlude}
		Fix a groupoid $ G $.  
		Then, the construction $ \mathcal{ D } \in \Gpd \mapsto \Fun ( \mathcal{ D } , \mathcal{ G } )$ is suitably functorial, which means 
		\begin{itemize}
			\item 
			For all $ D \in \Gpd $ it holds that $ \Fun ( \mathcal{ D } , \mathcal{ G } ) $ is a groupoid.
			
			\item 
			For all $  F \colon \mathcal{ D }_ \to \mathcal{ D }_2 $ it holds that $ ? \circ F \colon \Fun ( \mathcal{D}_2 , \mathcal{G} ) \to \Fun ( \mathcal{ D }_1 , \mathcal{G} ) $ is a functor, which means that 
			\[
			\begin{tikzcd}
				\mathcal{ D }_2
				\arrow[bend left=50]{rr}[name=U',label=above:$G_1$]{}
				\arrow[bend right=50]{rr}[name=D',label=below:$G_2$]{}
				&&
				\mathcal{ G }
				\arrow[shorten <=10pt,shorten >=10pt,Rightarrow,to path={(U') -- node[label=right:$\beta$] {} (D')}]{} 
			\end{tikzcd}
		\mapsto
			\begin{tikzcd}
				\mathcal{ D }_1
				\arrow[bend left=50]{rr}[name=U',label=above:$G_1 \circ F $]{}
				\arrow[bend right=50]{rr}[name=D',label=below:$G_2 \circ F $]{}
				&&
				\mathcal{ G }
				\arrow[shorten <=10pt,shorten >=10pt,Rightarrow,to path={(U') -- node[label=right:$\beta_F$] {} (D')}]{} 
			\end{tikzcd}
			\] 
			
			\item 
			For a natural transformation between functors between groupoids.
			\[
			\begin{tikzcd}
				\mathcal{ D }_1
				\arrow[bend left=50]{rr}[name=U',label=above:$F_1$]{}
				\arrow[bend right=50]{rr}[name=D',label=below:$F_2$]{}
				&&
				\mathcal{ D }_2
				\arrow[shorten <=10pt,shorten >=10pt,Rightarrow,to path={(U') -- node[label=right:$\alpha$] {} (D')}]{} 
			\end{tikzcd}
			\]
			there is a natural transformation
			\[
			\begin{tikzcd}
				\Fun( \mathcal{ D }_2 , \mathcal{ G } )
				\arrow[bend left=50]{rr}[name=U',label=above:$? \circ F_1$]{}
				\arrow[bend right=50]{rr}[name=D',label=below:$? \circ F_2$]{}
				&&
				\Fun( \mathcal{ D }_1 , \mathcal{ G } ).
				\arrow[shorten <=10pt,shorten >=10pt,Rightarrow,to path={(U') -- node[label=right:$? * \alpha$] {} (D')}]{} 
			\end{tikzcd}
			\]
		\end{itemize} 
	\end{Interlude}
\end{construction}

Let us take the 2- pullback from above $ \mathbb{ F } \mathcal{D} \coloneqq \Fun ( \mathcal{ D } , \mathcal{ A } ) \times_{ \Fun ( \mathcal{ D } , \mathcal{ C } ) } \Fun ( \mathcal{ D } , \mathcal{ B } ) $ and analyze the map $ \mathcal{ D } \to \mathbb{ F } \mathcal{ D } $ 

\begin{itemize}
	\item 
	For all groupoids $ \mathcal{D} $ the category $ \mathbb{F} \mathcal{D} $ is a groupoid.
	
	\item 
	For all $ F \colon \mathcal{D}_1 \to \mathcal{D}_2 $ the attribution $ \mathbb{F} \mathcal{D}_2 \to \mathbb{ F } \mathcal{D}_1 $ is a functor.
	
	\item 
	The objects are given as $ ( p_{ \mathcal{ A } } \colon \mathcal{ D } \to \mathcal{A } , p_{ \mathcal{B } } \colon \mathcal{ D } \to \mathcal{ B } , \phi \colon F \circ p_{ \mathcal{ A } } \isomorphism  G \circ p_{ \mathcal{ B }} ) $,
	that is the 2-pullback, i.e. the datum of a pullback diagram
	\[
	\begin{tikzcd}	
		\mathcal{A} \times_{ \mathcal{ C } } \mathcal{ B}
		\ar[r, " \pi_{ \mathcal{ B } } "]
		\ar[r , shift right=4ex, shorten <=3pt, Rightarrow, shorten >=3pt, " \phi " ]
		\ar[d, " \pi_{ \mathcal{ A }}"']
		&
		\mathcal{B}
		\ar[d, " G "]
		\\ 
		\mathcal{A }
		\ar[r, " F "' ]
		&
		\mathcal{C}
	\end{tikzcd}
	\]
	
	\item 
	For all functors $ H \colon \mathcal{D}_1 \to \mathcal{D}_2$ we get $ \mathbb{F} \mathcal{ D }_2 \to \mathbb{F} \mathcal{D}_2$, that is for a second pullback square 
	\[
	\begin{tikzcd}	
		\mathcal{D}_2
		\ar[r, " q_{\mathcal{B } }"]
		\ar[d, " q_{ \mathcal{A} }"]
		\ar[r , shift right=4ex, shorten <=3pt, Rightarrow, shorten >=3pt, " \psi " ]
		&
		\mathcal{B}
		\ar[d, " G "]
		\\
		\mathcal{A} 
		\ar[r, " F "]
		&
		\mathcal{C}
	\end{tikzcd}
	\]
	with a functor between the fiberproducts, is mapped to the following.
	\[
	\begin{tikzcd}
		\mathcal{D_1}
		\ar[rrd, bend left, bend left, " q_{ \mathcal{ B } } \circ H "]
		\ar[rd, " H "]
		\ar[rdd, bend right," q_{ \mathcal{ A } } \circ H "]
		\\
		&
		\mathcal{D }_2 
		\ar[r, " q_{ \mathcal{ B } }"]
		\ar[d, " q_{ \mathcal{ A } }"]
		\ar[r , shift right=4ex, shorten <=3pt, Rightarrow, shorten >=3pt, " \psi " ]
		&
		\mathcal{B}
		\ar[d, " G "]
		\\
		&
		\mathcal{A}
		\ar[r, " F "]
		&
		\mathcal{ C }
	\end{tikzcd}
	\]
	The associated natural transformation is given by 
	$\psi_H \colon F \circ q_{ \mathcal{A} } \circ H \implies G \circ q_{ \mathcal{ B } } \circ H.$
\end{itemize}

The next question we can naturally ask is what morphisms in $ \mathbb{F } ( \mathcal{ D } )$ look like 
\[
\begin{tikzcd}
	( p_{ \mathcal{ A } }\colon \mathcal{D}_2 
	\ar[r]
	\ar[d, Rightarrow, shift left = 2ex, " \alpha"]
	&
	\mathcal{D} , p_{\mathcal{B}} \colon \mathcal{D}_2
	\ar[r]
	\ar[d, Rightarrow, shift left = 2ex, " \beta"]
	&
	\mathcal{B} , F \circ p_{ \mathcal{A} }
	\ar[r , Rightarrow, " \phi" , " \sim"']
	\ar[d, Rightarrow, " F_{\alpha}"]
	&
	G \circ p_{ \mathcal{ B }} )
	\ar[d, Rightarrow,  " G_\beta"]
	\\
	( q_{ \mathcal{A } }\colon \mathcal{D}_2 
	\ar[r]
	&
	\mathcal{A}, q_{ \mathcal{B} }: \mathcal{D}_2
	\ar[r]
	&
	\mathcal{B} , F \circ q_{ \mathcal{A }}
	\ar[r, Rightarrow, "\psi"' , "\sim"]
	&
	G \circ q_{ \mathcal{ B }})	
\end{tikzcd}
\]

For every 
$
\begin{tikzcd}
	\mathcal{ D }_1
	\arrow[bend left=50]{rr}[name=U',label=above:$H_1$]{}
	\arrow[bend right=50]{rr}[name=D',label=below:$H_2$]{}
	&&
	\mathcal{ D }_2
	\arrow[shorten <=10pt,shorten >=10pt,Rightarrow,to path={(U')-- node[label=right:$\gamma$] {} (D')}]{} 
\end{tikzcd}
$
there is a diagram 
$
\begin{tikzcd}
	\mathbb{F} \mathcal{ D }_1
	\arrow[bend left=50]{rr}[name=U',label=above:$\mathbb{F}H_1$]{}
	\arrow[bend right=50]{rr}[name=D',label=below:$\mathbb{F}H_2$]{}
	&&
	\mathbb{F}\mathcal{ D }_2
	\arrow[shorten <=10pt,shorten >=10pt,Rightarrow,to path={(U')-- node[label=right:$\mathbb{F}\gamma$] {} (D')}]{} 
\end{tikzcd}
$

Thus for any two pullback diagrams we obtain morphisms

\todo{maybe draw this huge diagram}

\begin{prop}
	Let $ \mathcal{ A } \xrightarrow{ F } \mathcal{ C } \xleftarrow{ G } \mathcal{ B } $ be a cospan of groupoids.
	Then for all groupoids $ \mathcal{ D } $, it holds that 
	\begin{align*}
		\mathbb{ X } \colon \mathbb{F} ( \mathcal{ A } \times^{ ( 2 ) }_{ \mathcal{C } }\mathcal{B })
		&\to
		\mathbb{F} ( \mathcal{D } \xleftarrow{ \mathbb{F} H } \mathbb{F } ( \mathcal{ A } \times_{ \mathcal{C }}^{ ( 2 ) }  \mathcal{B} ) )		
		\\
		( \mathcal{D} \xrightarrow{ H } \mathcal{ A } \times_{ \mathcal{ C }}^{ ( 2 ) } \mathcal{ B } )
		&\mapsto
		H^*( can )
	\end{align*}
	is an isomorphism. 
\end{prop}

\begin{exmp}
	Let $ A , B , C $ be groups and $ \mathbb{B} A , \mathbb{B} B , \mathbb{B} C $ their associated groupoids, then for group homomorphisms $ A \xrightarrow { f } C  \xleftarrow { g } B $,
	we get that the objects correspond to triples $ ( *_A , *_B , *_C \xrightarrow{c \in C} *_C ) $	
\end{exmp}



