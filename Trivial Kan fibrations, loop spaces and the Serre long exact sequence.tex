Lecture 17.12

\section{Trivial Kan fibrations, loop spaces and the Serre long exact sequence}

\underline{Construction}:
Let $X$ be a Kan complex \& $x \in X_0$
The \underline{(based) loop space} $\Omega(X,x)$ is given by the following pullback:
\[
\begin{tikzcd}
    \Omega(X,x)
    \ar[r]
    \ar[d]
    &
    \underline{\Hom(\Delta^1, X)}
    \ar[d, shift left=4, "i^*"]
    \ar[d, shift right=10]
    \\
    \Delta^0
    \ar[r, "{(x,x)}"]
    \ar[r, bend right, "c(x)"]
    &
    X\times X \cong \underline{\Hom(\partial \Delta^1 , X)}
\end{tikzcd}
\]
where $i^*$ is a Kan fibration and thus $\Omega(X,x)$ a Kan complex.
The \underline{path space} $\Pathspace X$ is given by the following diagram 
\[
\begin{tikzcd}
    \Omega(X,x) 
    \ar[d]
    \ar[r]
    &
    \Pathspace X
    \ar[r]
    \ar[d, "\pi"]
    &
    \underline{\Hom (\Delta^1,X)}
    \ar[d, "i^*"]
    \\
    \Delta^0
    \ar[rr, bend right, "{(x,x)}"']
    \ar[r]
    &
    X
    \ar[r,"{(i(x), \id_X)}"]
    &
    X \times X
    \cong 
    \underline{\Hom}(\partial \Delta^1, X)
\end{tikzcd}
\]

\underline{Aim}: Prove that for all $n \geq 0$ there is an isomorphism $\pi_{n+1}(X,x) \isomorphism \pi_n ( \Omega(X,x), 1_X)$ and that for all $n \geq 1$ the group $\pi_n(\Omega(X,x),1_X)$ is abelian and hence for all $n \geq 2$ the group $\pi_n(X,x)$ is abelian.
\begin{defi/prop}
\label{trivial_Kan_fibration}
    Let $p\colon X \to Y$ be a morphism in $\SetD$. The following are equivalent 
    \begin{enumerate}
        \item 
        $\{ \partial\Delta^n \hookrightarrow \Delta^n \mid n \geq 0 \} \lift{} p$
        \item 
        $\overline{\{ \partial\Delta^n \hookrightarrow \Delta^n \mid n  \geq 0\}} = \{\text{monomorphisms}\} \lift{} p$
    \end{enumerate}
    We call such a map a \underline{trivial Kan fibration}.
    Since $\Lambda_k^n \hookrightarrow \Delta^n$ is a monomorphism, all trivial Kan fibrations are especially Kan fibrations.
\end{defi/prop}

\begin{prop}
    Let $X \xrightarrow{p} \Delta^0$ be a trivial Kan fibration, then for all $x \in X$ and for all $n\geq 0$ the homotopy group is trivial, that is $\pi_n(X,x)=\{*\}$.
\end{prop}

\begin{proof}
    Exercise.    
\end{proof}

\begin{prop}
    Let $X \xrightarrow{p} Y$ be a Kan fibration and $L \xrightarrow{i}K$ be an anodyne extension, then
    $\underline{\Hom} (K,X) \to \underline{\Hom}(L,X) \times_{\underline{\Hom}(L,Y)} \underline{\Hom} (K,Y) $ is a trivial Kan fibration.    
\end{prop}

\begin{proof}
    The idea of the proof is the same as for 
    \todo{find where it is the same, too tired now, did alot today pat pat, no more focus left, gonna just tex now}
    \[
    \begin{tikzcd}
        (\partial \Lambda_k^n \times L) \cup (\Delta^n \times K)
        \ar[r]
        \ar[d]
        &
        X
        \ar[d, "p"]
        \\
        \Delta^n \times L 
        \ar[r]
        \ar[ru, dashed, "\exists"]
        &
        Y
    \end{tikzcd}
    \]
    \[
    \begin{tikzcd}
        \partial \Delta^n 
        \ar[r]
        \ar[d]
        &
        \underline{\Hom(K,X)}
        \ar[d]
        \\
        \Delta^n 
        \ar[r]
        \ar[ru, dashed]
        &
        \underline{\Hom}(L,X) \times_{\underline{\Hom}(L,Y)} \underline{\Hom} (K,Y)
    \end{tikzcd}
    \]
\end{proof}

\begin{cor}
    Let $X$ be a Kan complex then there is a pullback square:
    \[
    \begin{tikzcd}
        \Pathspace X
        \ar[r]
        \ar[d, "p"]
        &
        \underline{\Hom}(\Delta^1,X)
        \ar[d, "i^*"]
        \\
        \Delta^0
        \ar[r, "x"]
        &
        \underline{\Hom}(\Delta^{\{0\}},X) \cong X
    \end{tikzcd}
    \]
    where $i^*$ is a trivial Kan fibration by \cref{trivial_Kan_fibration} and we claim that $p$ is one as well.
\end{cor}

\begin{proof}
    Consider the following diagram
    \[
    \begin{tikzcd}
        \Pathspace X 
        \ar[r]
        \ar[d, "\pi"']
        &
        \underline{\Hom(\Delta^1, X)}
        \ar[d]
        \ar[dd, bend left=90, "i^*"]
        \\
        X
        \ar[r, "{(c(x), \id_X)}"]
        \ar[d]
        &
        X \times X \cong \underline{\Hom ( \partial \Delta^1, X)}
        \ar[d, "\pi_0"]
        \\
        \Delta^0
        \ar[r, "x"]
        &
        X 
        \cong
        \underline{\Hom ( \Delta^{\{0\}}, X)}
    \end{tikzcd}
    \]
    We know the top square is a pullback, thus is the bottom one were one as well, then we would be done by the pasting lemma. So let us take a closer look here.
    \[
    \begin{tikzcd}
        W 
        \ar[rdd, bend right]
        \ar[rrd, bend left]
        \ar[rd]
        &&
        \\
        &
        X
        \ar[r, "{(c(x), \id_X)}"]
        \ar[d]
        &
        X \times X
        \ar[d, "\pi_0"]
        \\
        &
        \Delta^0
        \ar[r]
        &
        X
    \end{tikzcd}
    \]
    \todo{I do not get the argument here}
    \begin{cor}
        For all $x \in \Pathspace X_0$ and for all $n \geq 0$ we have that $\pi_n(\Pathspace X, x)= \{*\}$. 
    \end{cor}
    Let $X \xrightarrow{p} Y$ be a Kan fibration, with $X$ as well as $Y$ Kan complexes and $x \in X_0$ as well as $y=p(x) \in Y_0$. 
    Define
    \[
    \begin{tikzcd}
        F 
        \ar[r, "i"]
        \ar[d]
        &
        X
        \ar[d, "p"]
        \\
        \Delta^0 
        \ar[r, "y"]
        &
        Y
    \end{tikzcd}
    \]
    Notice that $x \in F_0$ by construction.
    By the functoriality of the homotopy groups we get a sequence 
    \[
    \pi_n(F,x)
    \xrightarrow{i_*}
    \pi_n(X,x)
    \xrightarrow{p_*}
    \pi_n(Y,y)
    \]
    for all $n \geq 0$.
\end{proof}

\begin{construction}
    Let $n \geq 1$ and $\alpha \colon \Delta^n \to Y$ be a representative of a class in $\pi_n(Y,y)$.
    Consider the following lifting diagram:
    \[
    \begin{tikzcd}
        \Lambda_0^n 
        \ar[d, hook]
        \ar[r]
        &
        \Delta^0
        \ar[r, "x"]
        &
        X
        \ar[d,"p"]
        \\
        \Delta^n
        \ar[rru, "\exists v"]
        \ar[rr,"\alpha"]
        &&
        Y
    \end{tikzcd}
    \]
    Where the left horn-inclusion is an anodyne extension and the morphism $p$ is a Kan fibration. Since Kan fibrations are exactly the morphisms with the right lifting property with respect to the anodyne extensions we get a lift $v$.
    We define $\delta ( [\alpha] ) = [d_0(v)]$ for $d_0(v)\colon \Delta^{n-1} \to F$, notice that since $F$ is a pullback there is a unique morphism into $F$ given by the morphism $d_0(v)$ that goes to $X$ and the unique morphism into the terminal object $\Delta^0$.
    So it makes sense to take $d_0(v)$ as a morphism into $F$.
    This defines a map $\delta \colon \pi_n(Y,y) \to \pi_n(F,x)$.

    We thus get a map $\delta \colon \pi_n(Y,y) \to \pi_{n-1}(F,x)$.
    To see this map is well defined consider $h\colon \Delta^1 \times \Delta^n \to Y$ a homtopy of $\alpha$ to $\alpha'$ ($\rel \partial \Delta^n$).
    Then we choose lifts 
    \[
    \begin{tikzcd}
        \Lambda_0^n
        \ar[r, "c(x)"]
        \ar[d, hook]
        &
        X
        \ar[d,"p"]
        \\
        \Delta^n 
        \ar[ru, dashed, "v"]
        \ar[r, "\alpha"]
        &
        Y
    \end{tikzcd}
    \qquad
    \begin{tikzcd}
        \Lambda_0^n
        \ar[r, "c(x)"]
        \ar[d, hook]
        &
        X
        \ar[d,"p"]
        \\
        \Delta^n 
        \ar[ru, dashed, "v'"]
        \ar[r, "\alpha'"]
        &
        Y
    \end{tikzcd}
    \]
    to obtain for the tuple $\omega=(v,v',(\bullet,x \dotsc, x))$ a lift
    \[
    \begin{tikzcd}
        (\partial \Lambda_k^n \times L) \cup (\Delta^n \times K)
        \ar[r, "\omega"]
        \ar[d, "\An \ni"']
        &
        X
        \ar[d, "p"]
        \\
        \Delta^n \times L 
        \ar[r]
        \ar[ru, dashed, "\exists \Tilde{h}"]
        &
        Y
    \end{tikzcd}
    \]
    which again exists since the left vertical map is anodyne and the morphism $p$ a Kan fibration.
    This results in 
    \[
        \Delta^1 \times \Delta^{n-1} \xrightarrow{\id \times d^0} \Delta^1 \times \Delta^n \xrightarrow{\Tilde{h}} X
    \]
    Now this a homotpy of $d_0v$ to $d_0v'$ ($\rel \partial \Delta^{n+1}$ and thus $[d_0v]=[d_0v']$.
    \todo{tikz picture}
\end{construction}

\begin{thm}{Serre's Long exact sequence}
    Let $X \xrightarrow{p} Y$ be a Kan fibration such that $Y$ is a Kan complex. Let $x \in X_0$ and $y\coloneqq p(x) \in Y_0$. Then there is a long exact sequence of pointed sets.
    \[
    \begin{tikzcd}
        & 
        \dotsc
        \rar
        \arrow[d, phantom, ""{coordinate, name=Z1}]
        &
        \pi_2(Y,y)
        \arrow[ dll,
                        "\delta", pos=1,
                        rounded corners,
                        to path={ -- ([xshift=2ex]\tikztostart.east)
                                  |- (Z1) \tikztonodes
                                  -| ([xshift=-2ex]\tikztotarget.west)
                                  -- (\tikztotarget)}
                      ]
        \\
        \pi_1(F,x)
        \rar
        & 
        \pi_1(X,x)
        \rar
        \arrow[d, phantom, ""{coordinate, name=Z}]
        &
        \pi_1(Y,y)
        \arrow[ dll,
                        "\delta", pos=1,
                        rounded corners,
                        to path={ -- ([xshift=2ex]\tikztostart.east)
                                  |- (Z) \tikztonodes
                                  -| ([xshift=-2ex]\tikztotarget.west)
                                  -- (\tikztotarget)}
                      ]
        \\
        \pi_0(F,x)
        \rar
        &
        \pi_0(X,x)
        \rar
        &
        \pi_0(Y,y)
    \end{tikzcd}
    \]
    for a given pullback square
    \[
    \begin{tikzcd}
        F
        \ar[r, "i"]
        \ar[d]
        &
        X
        \ar[d, "p"]
        \\
        \Delta^0
        \ar[r,"x"]
        &
        Y
    \end{tikzcd}
    \]
    We call a sequence of pointed sets $((A,a) \xrightarrow{f} (B,b) \xrightarrow{g} (C,c))$ is called exact if $\ker g \coloneqq \{b \in B \mid g(b) = c\} = \im (f)$. 
    Moreover there is a natural action of $\pi_1(Y,y)$ on $\pi_0(F,x)$ such that the stabilizer of $[x] \in \pi_1(Y,y)$ is precisely $\im(p_*)$, this means that $i_*([x'])=i_*([x''])$ if and only if $[x']$ and $[x'']$ are in some $\pi_1(Y,y)$-orbit.
\end{thm}

\begin{cor}
    Let $X$ be a Kan complex and $x \in X_0$. Take the pullback 
    \[
    \begin{tikzcd}
        \Omega(X,x)
        \rar
        \dar
        &
        \Pathspace X
        \ar[d, "\pi"]
        \\
        \Delta^0
        \rar
        &
        X
    \end{tikzcd}
    \]
    Then for all $n \geq 0$ the morphism $\pi_{n+1}(X,x) \xrightarrow{\delta} \pi_n(\Omega(X,x),1_x)$ is bijective.
\end{cor}

\begin{proof}
    By the Serre long exact sequence we have for all $n \geq 1$
    \[
    \{*\}=\pi_{n+1}(\Pathspace X,1_x) \to \pi_{n+1}(X,x) \xrightarrow[\sim]{\delta} \pi_n(F,x) \to \pi_n(\Pathspace X, 1_x)=\{*\}
    \]
    \todo{warum brauchen wir hier orbit stabilzer, reicht es nicht, dass exakt,grp. hom und deswegen bij.}
\end{proof}

\begin{prop}
    The map $\delta\colon \pi_n(Y,y) \to \pi_{n-1}(F,x)$ is a group homomorphism for $n \geq 2$.
\end{prop}

\begin{proof}
    Let $\omega \colon \Delta^{n+1} \to Y$ be an (n+1)-simplex, such that
    $\partial \omega = ( y, \dotsc , y , \alpha_{n-1}, \alpha_n , \alpha_{n+1})$. Furthermore $\omega$ is a witness of $[\alpha_{n-1}][\alpha_{n+1}]=[\alpha_n]$ in $\pi_n(Y,y)$.
    Choose (for $i=n-1,n,n+1)$ witnesses of $\delta([\alpha_i])=[d_0v_i]$
    \[
    \begin{tikzcd}
        \Lambda_0^n
        \ar[r, "c(x)"]
        \ar[d]
        &
        X
        \ar[d, "p"]
        \\
        \Delta^n
        \ar[ru, dashed, "v_i"]
        \ar[r, "\alpha_i"']
        &
        Y
    \end{tikzcd}
    \]
    \[
    \begin{tikzcd}
        \Lambda^{n+1}_0
        \ar[rrr, "{(x,\dotsc,x, v_{n-1},v_n,v_{n+1})}"]
        \dar
        &&&
        X
        \dar["p"]
        \\
        \Delta^{n+1}
        \ar[rrru, dashed, "\exists \gamma"]
        \ar[rrr, "\omega"']
        &&&
        Y
    \end{tikzcd}
    \]
    We get the folowing equation $\partial([\alpha_{n-1}]) \partial([\alpha_{n+1}])=[d_0v_{n-1}][d_0v_{n+1}]=[d_0v_n]=\partial([\alpha_n])=\partial([\alpha_{n-1}][\alpha_{n+1}])$.
    Then $d_0\gamma$ is a witness of the above composition since $\partial(d_0\gamma)=(x,\dotsc,x, d_0v_{n-1}, d_0v_n, d_0v_{n+1})$.
    We now want to show (parts of) the exactness of Serre's long exact sequence.
    Let
    \[
        [\alpha] \in \pi_n(F,x) \xrightarrow{i_*} \pi_n(X,x) \xrightarrow{p_*}
        \pi_n(Y,y)
    \]
    be the sequence of homtopygroups.
    Then we have the following square
    \[
    \begin{tikzcd}
        \Delta^n
        \ar[r, "\alpha"]
        \ar[d]
        &
        X
        \ar[d, "p"]
        \\
        \Delta^0
        \ar[r,"y"]
        &
        Y
    \end{tikzcd}
    \]
    This shows that $ \im(i_*) \subseteq \ker(p_*)$.
    Conversely, if we have a commutative diagram
    \[
    \begin{tikzcd}
        \Delta^n
        \ar[r, "\alpha"]
        \ar[d]
        &
        \ar[rd, "p_*\alpha"']
        X
        \ar[d, "p"]
        \\
        \Delta^0
        \ar[r,"y"]
        &
        Y
    \end{tikzcd}
    \]
    Then $\alpha \colon \Delta^n \to  X$ is an $n$-simplex of the fibre
    $\pi_n(X,x) \xrightarrow{p_*} \pi_n(Y,y) \xrightarrow{\delta} \pi_{n-1}(F,x)$.
    To show $\im(p_*) \subseteq  \ker(\delta)$, take an $n$-simplex and extend it along $p$
    \[
    \begin{tikzcd}
        \Delta^n
        \ar[r,"\alpha"]
        \ar[rd, "p_{\alpha}"]
        &
        X
        \ar[d,"p"]
        \\
        &
        Y
    \end{tikzcd}
    \]
    This can be completed to a commutative square 
    \[
        \begin{tikzcd}
        \Lambda_0^n
        \dar
        \ar[r,"c(x)"]
        &
        X
        \ar[d,"p"]
        \\
        \Delta^n
        \ar[ru, "\alpha"]
        \ar[r, "p_{\alpha}"]
        &
        Y
        \end{tikzcd}
    \]
    where $\delta[p_*\alpha]=[d_0\alpha]=c(x)$.
    To show that $\im(p_*) \supseteq \ker \delta$, let $[\alpha] \in \pi_n(Y,y)$ such that $\delta([\alpha])=[d_0v]=[c(x)]$, that is $[\alpha] \in \ker \delta$.
    This means we have a homotopy from $[d_0v]$ to $[c(x)]$
    \[
    \begin{tikzcd}
        \Delta^{\{0\}} \times \Delta^{n-1}
        \ar[d, hook]
        \ar[dr, bend left, "d_0v"]
        &
        \\
        \Delta^1 \times \Delta^{n-1}
        \ar[r, "\exists h_0"]
        &
        X
        \\
        \Delta^{\{1\}} \times \Delta^n
        \ar[u, hook]
        \ar[ru, bend right, "c(x)"']
    \end{tikzcd}
    \]
    We thus obtain a commutative square
    \[
    \begin{tikzcd}
        &
        (\Delta^{\{1\}} \times \Delta^n) \cup ( \Delta^1 \times \partial \Delta^n)
        \ar[rrr, "{(v ( h_0, x , \dotsc , x ))}"]
        \ar[d, hook]
        &&&
        X
        \ar[d, "p"]
        \\
        \Delta^{\{1\}} \times \Delta^n
        \ar[r]
        &
        \Delta^1 \times \Delta^n
        \ar[rrru, "\exists \Tilde{h}"]
        \ar[rrr, "h\coloneqq p \Tilde{h}"]
        &&&
        Y
    \end{tikzcd}
    \]
    Then $h$ is a homotopy $\alpha \to p( \Tilde{h}d^1)$ $\rel \partial\Delta^n$
    The rest of the proof is given as exercise.
\end{proof}

\subsection{Interlude}

As we know, we can $\Kan \subseteq \SetD $ and pass to $\hKan$ the homotopy category of Kan complexes. 
Since we wish to interpret Kan complexes as models for $\infty$-groupoids, we have been studying "the homotopy category of $\infty$-groupoids."

\subsection{Challenge}

We wish to regard homotopy equivalent Kan complexes as being isomorphic, while having access to universal properties (limit/colimit constructions).

\underline{Aim}: Understand how to use $\SetD$ to study the $(\infty,1)$-category of Kan complexes in which instead of Hom sets we have "Hom Kan complexes" (="Hom $\infty$-groupoids").

\begin{defi}
    Let $X \xrightarrow{f} Y$ be a map of Kan complexes. Then $f$ is a \underline{weak homotopy equivalence} if $\forall x \in X$ and $\forall n \geq 0$, $\pi_n(f) \colon  \pi_n(X,x) \isomorphism \pi_n(Y,f(x))$.
\end{defi}

\begin{rmk}
    By Whitehead's theorem for a category $\mathcal{C}$ and $W \subseteq \Mor (\mathcal{C})$ a class of maps, we obtain:
    \[
    \begin{tikzcd}
        \SetD
        \rar
        \ar[rd]
        &
        \Gpd_{\infty}= \SetD[Weq^{-1}]_{(\infty, 1) cat}
        \ar[d]
        \\
        &
        \hKan
    \end{tikzcd}
    \]
\end{rmk}

We are not gonna detail the motivation here, since it is going to be extended in its formality in the nexte lectures.

\begin{cor}
    Let $X \xrightarrow{p} Y$ be a Kan fibration and $f\colon y_0 \to y_1$ be an edge in $Y$.
    Then there exist $X_{y_0} \xleftarrow{} \bullet \to X_{y_1}$ trivial Kan fibrations.
\end{cor}

\begin{proof}
    For 
    \[
    \begin{tikzcd}
        W
        \ar[r,"q"]
        \ar[d, "r"']
        &
        Y
        \\
        X
        \ar[ru, "p"']
    \end{tikzcd}
    \]
    define $\underline{\Hom}_Y((W,q)(X,p))$ by and take the pullback
    \[
    \begin{tikzcd}
        \underline{\Hom_Y}((W,q),(X,p)) 
        \ar[d]
        \rar
        &
        \underline{\Hom}(W,X)
        \ar[d, "p_*"]
        \\
        \Delta^0
        \ar[r, "q"]
        &
        \underline{\Hom}(W,Y)
    \end{tikzcd}
    \]
    Where both vertical maps are Kan fibrations.
    Consider for example a morphism from the trivial simplicial set $\Delta^0 \xrightarrow{y_e}Y$ then
    \[
    \begin{tikzcd}
        X_{y_e}=\underline{\Hom}_Y((\Delta^0,y_e)(X,\circ)) 
        \rar
        \dar
        &
        \underline{\Hom}(\Delta^0,X)
        \ar[r,"\sim"]
        \ar[d, "p_*"]
        &
        X
        \ar[d,"p"]
        \\
        \Delta^0
        \ar[r,"y_e"]
        &
        \underline{\Hom}(\Delta^0,Y)
        \ar[r, "\sim"]
        &
        Y
    \end{tikzcd}
    \]
    Consider now the pullback-square 
    \[
        \begin{tikzcd}
            \underline{\Hom}_Y((\Delta^1,f),(X,p))
            \rar
            \dar
            &
            \underline{\Hom}(\Delta^1,X)
            \dar
            \\
            \underline{\Hom}_Y((\Delta^{\{l\}},y_e),(X,p))
            &
            \underline{\Hom}(\Delta^1,Y) \times_{\underline{\Hom}}(\Delta^{\{l\}},Y) \underline{\Hom}(\Delta^{\{l\}},X)
        \end{tikzcd}
    \]
\end{proof}

Lecture 7.1

\begin{prop}
    Let $X$ be a Kan complex. 
    Then $\pi_1(\Omega(X,x),1_X)$ is abelian.
\end{prop}

\begin{proof}
    We have $\pi_1(\Omega(X,x),1_X)$ has 2 binary operations.
    \todo{why tho?}
    By the Eckmann Hilton argument the result follows.
    The details of that are to be worked out in the exercises.
\end{proof}

Let now $\alpha\colon \Delta^1 \to \Omega(X,x) $ be a 1-simplex of the loop space.
Take the pullback diagram:
\[
\begin{tikzcd}
    \Delta^1
    \ar[rd,"\alpha"]
    \ar[rrd, bend left]
    \ar[rdd, bend right]
    &&
    \\
    &
    \Omega(X,x)
    \rar
    \dar
    &
    \underline{\Hom}(\Delta^1,X)
    \ar[d, "{(s,t)}"]
    \\
    &
    \Delta^0
    \ar[r, "{(X,x)}"]
    &
    X \times X \cong \underline{\Hom}(\partial \Delta^1, X)
\end{tikzcd}
\]

\[
    \begin{tikzcd}
        x
        \ar[d,"1_x"']
        \ar[r,"1_x"]
        &
        x
        \ar[d,"1_x"]
        \\
        x
        \ar[r,"1_x"']
        &
        x
    \end{tikzcd}
\]

We see that binary operations correspond to "vertical stacking" and "horizontal stacking".
\[
\begin{tikzcd}
    x
    \ar[r]
    \ar[d]
    \ar[r, phantom, shift right =3.4ex, "\beta" ]
    \ar[rr, bend left]
    \ar[dd, bend right]
    &
    x
    \ar[r]
    \ar[d]
    \ar[r, phantom, shift right =3.4ex, "\delta" ]
    &
    x
    \dar
    \ar[dd, bend left]
    \\
    x
    \ar[r]
    \ar[d]
    \ar[r, phantom, shift right =3.4ex, "\alpha" ]
    \ar[rr, bend left]
    &
    x
    \ar[r]
    \ar[d]
    \ar[r, phantom, shift right =3.4ex, "\gamma" ]
    &
    x
    \dar
    \\
    x
    \rar
    \ar[rr,bend right]
    &
    x
    \rar
    &
    x
\end{tikzcd}
\begin{tikzcd}[column sep=4ex,row sep=1ex, ampersand replacement=\&]
    \ar[r, "{\begin{pmatrix}
        \beta  & \delta \\
        \alpha & \gamma
        \end{pmatrix}}"]
    \&
    X
    \&
    \\
    \ar[r, hook, "{\in \An}"']
    \&
    (\Lambda_1^2 \times \Delta^2) \cup ( \Delta^2 \times \Lambda_1^2)
    \ar[r, hook, "{\in \An}"']
    \&
    \Delta^2 \times \Delta^2
    \ar[lu, "w"']
\end{tikzcd}
\]


    $([\alpha] \circ [\beta]) \bullet ( [\gamma] \circ [\delta])$
    
     $=( \left[
    \begin{tikzcd}
        00
        \rar
        \dar
        &
        01
        \dar
        \\
        10
        \rar
        &
        21
    \end{tikzcd}
    \right] ) 
    \circ ( \left[ 
    \begin{tikzcd}
        01
        \rar
        \dar
        &
        02
        \dar
        \\
        21 
        \rar
        &
        22
    \end{tikzcd}
    \right] )
    =
    ( \left[
    \begin{tikzcd}
        00
        \rar
        \dar
        &
        02
        \dar
        \\
        20
        \rar
        &
        22
    \end{tikzcd}
    \right] )$ 

    
    $=
    ( \left[
    \begin{tikzcd}
        10
        \rar
        \dar
        &
        12
        \dar
        \\
        20
        \rar
        &
        22
    \end{tikzcd}
    \right] ) \circ
    ( \left[
    \begin{tikzcd}
        00
        \rar
        \dar
        &
        02
        \dar
        \\
        10
        \rar
        &
        12
    \end{tikzcd}
    \right] ) 
$
